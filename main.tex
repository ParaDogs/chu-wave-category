\documentclass[a4paper,12pt]{article}
\usepackage{framed}
\usepackage[top=2cm, bottom=2cm]{geometry}
\usepackage[utf8]{inputenc}
\usepackage[english,russian]{babel}
\usepackage{amsthm,amssymb,amsfonts,amsmath,cite,enumerate}
\usepackage[all]{xy}

\newtheorem{statement}{Утверждение}
\newtheorem{lemma}{Лемма}
\newtheorem{theorem}{Теорема}
\newtheorem{consequence}{Следствие}
\newtheorem{remark}{Замечание}

\newcommand{\SAct}{S\text{-}Act}
\newcommand{\GAct}{G\text{-}Act}
\newcommand{\FGAct}{FG\text{-}Act}
\newcommand{\fo}{\widetilde{f^0}}

\begin{document}

\section*{Определения}

Зафиксируем категорию $C$ и объект $\star \in Ob(C)$. Определим пунктированную категорию $C_\star$ категории $C$ следующим образом:
\begin{itemize}
    \item Объектами являются пары $(k,u)$, где $k \in Ob(C)$, $u \in Hom(\star,k)$;
    \item Морфизмом из $(k,u)$ в $(l,v)$ является элемент $f \in Hom(k,l)$ такой, что следующая диаграмма
    $$\xymatrix@R=2em@C=2em{
        &\star \ar[ld]_u \ar[rd]^v&\\
        k \ar[rr]^f && l
    }$$
    коммутативна в $C$, где композиция наследуется из $C$.
\end{itemize}

Пусть $(C,\otimes,\star)$ --- моноидальная, замкнутая, биполная категория с терминальным объектом $\star$, и пусть $C_\star$ --- пунктированная категория категории $C$. В категории $C_\star$ объекты $(k,u)$ будем обозначать через $k_\star$ или просто $k$. Для объектов $k$ и $l$ категории $C_\star$ определим их скрещенное произведение $k \wedge l$ как объект категории $C_\star$ задаваемый следующим корасслоенным произведением в категории $C$:
$$\xymatrix@R=2em@C=2em{
    (k \otimes \star) \sqcup (\star \otimes l) \ar[r] \ar[d] & k \otimes l \ar[d]\\
    \star \ar[r] & k \wedge l.
}$$

Всюду ниже $(C, \times, \star)$ будет означать моноидальную, замкнутую, биполную категорию с терминальным объектом $\star$, объектами $C$ являются алгебраические системы, и для любой подсистемы $B$ произвольной алгебраической системы $A \in Ob(C)$ существует подсистема $C$ системы $A$ такая, что $A = B \sqcup C$. Скрещенное произведение $A \wedge B$ объектов $A, B \in Ob(C_\star)$ изоморфно объекту $((A \times B)/\Theta,w)$, где $\Theta = \{(a,\star) \mid a \in A\} \cup \{(\star,b) \mid b \in B\} \cup \{(c,c) \mid c \in A \cup B\}$ и $w(\star)/\Theta = \{(a,\star) \mid a \in A\} \cup \{(\star, b) \mid b \in B\}$. В пунктированной категории $C_\star$ элементы $(a,b)/w$ скрещенного произведения $A \wedge B$ будем обозначать через $(a,b)$ и $\star$ будем рассматривать как элемент $A$.

Например:
\begin{itemize}
    \item Пунктированная категория $Set_\star$ категории множеств $Set$. Объектом $\star$ в этой категории является одноэлементное множество, тензоным произведением является декартово произведение, копроизведением является дизъюнктное объединение.
    \item Пунктированная категория $(\GAct)_*$ категории $\GAct$ полигонов над группой $G$. Объектом $\star$ в этой категории является одноэлементный полигон, тензорным произведением является декартово произведение, копроизведением является дизъюнктное объединение. Поскольку любой связный полигон над группой $G$ является циклическим полигоном без чистого подполигона и любой полигон является копроизведением циклических подполигонов, тогда для любого подполигона $B$ произвольного полигона $A$ существует подполигон $C$ полигона $A$ такой, что $A = B \sqcup C$.
    \item Пунктированная категория $(\FGAct)_\star$ категории $\FGAct$, где $G$ --- группа и категория $\FGAct$ определяется следующим образом: $Ob(\FGAct) = Ob(\GAct)$ и
    $$
    u \in Hom_{\FGAct}(A,B) \Leftrightarrow u \in Hom_{\GAct}(G \times A, B), u(st,a) = u(s,a)
    $$
    для любых $s,t \in G$, $a \in A$. В работе \ref{2021} доказано, что категория $\FGAct$ естественно изоморфна категории $\GAct$, объект $\star$, тензорное произведение, копроизведение полигонов в этой категории определяются также как в категории $\GAct$. Таким образом, для любого подполигона $B$ произвольного полигона $A$ существует подполигон $C$ полигона $A$ такой, что $A = B \sqcup C$.

    Пространство Чу над категорией $C$ обозначается как четверка $(A,X,D,r)$, где $A,X,D \in Ob(C)$, $r \in Hom_C(A \times X, D)$. Если это не вызывает путаницы, то для пространства Чу $(A,X,D,r)$ будем использовать обозначение $r \in Hom_C(A \wedge X,D)$ или, поскольку категория $C$ фиксирована, $r: A \wedge X \to D$. Определим категорию $Chu_\sim(C_\star)$ следующим образом:
    \begin{itemize}
        \item $Ob(Chu_\sim(C_\star))$ --- это множество всех пространств Чу над категорией $C_\star$;
        \item если $r: A \wedge X \to D$, $r: A' \wedge X' \to D'$, то морфизмом или преобразованием Чу $(f,g,\widetilde{h}): r \to r'$ из $r$ в $r'$ является тройка $(f,g,\widetilde{h})$, где $f: A \to A'$, $g: X' \to X$, $h: D \to D'$ --- морфизмы категории $C_\star$, отношение $\sim$ --- это отношение эквивалентности на множестве
        $$
            \{h' \mid h': D \to D',\ \forall a \in A\ \forall x' \in X'\ h'(r(a,g(x'))) = r'(f(a),x')\}
        $$
        такое, что
        $$
            h' \sim h'' \Leftrightarrow \forall a \in A\ \forall x' \in X'\ h'(r(a,g(x'))) = h''(r(a,g(x')))
        $$
        и $\widetilde{h} = h/\sim$;
        \item если $(f,g,\widetilde{h}): r \to r'$ и $(f',g',\widetilde{h'}): r' \to r''$, то композиция преобразований Чу определяется следующим образом:
        $$
            (f',g',\widetilde{h'}) \circ (f,g,\widetilde{h}) = (f' \circ f, g \circ g', \widetilde{(h' \circ h)}): r \to r''.
        $$
    \end{itemize}
\end{itemize}

Поскольку категория $C$ алгебраических систем сигнатуры $\Sigma$ является замкнутой моноидальной, то для любых $A,B \in Ob(C)$ существуют $A \multimap B \in Ob(C)$ и морфизм $ev_{A,B}: (A \multimap B) \times A \to B$ такие, что для любого морфизма $f: X \times A \to B$ существует единственный морфизм $h: X \to A \multimap B$ такой, что $f = ev_{A,B} \circ (h \times id_A)$. Покажем, что в этом случае моноидальная категория $C_\star$ с тензорным произведением $\wedge$ является замкнутой моноидальной. Пусть $A_\star,B_\star \in Ob(C_\star)$, тогда $A_\star = (A,w_{A_\star})$, $A_\star = (B,w_{B_\star})$, где $A,B \in Ob(C)$, $w_{A_\star}: \{\star\} \to A$, $w_{B_\star}: \{\star\} \to B$, $\{\star\}$ --- терминальный объект категории $C$. Положим $A_\star \multimap B_\star = (A \multimap B, w_{A_\star \multimap B_\star})$, где $w_{A_\star \multimap B_\star}: \{\star\} \to (A \multimap B)$ определим следующим образом: $ev_{A,B}(w_{A_\star \multimap B_\star}(\star),a) = w_{B_\star}(\star)$ для любых $a \in A$. Покажем, что $w_{A_\star \multimap B_\star}$ определен однозначно. Пусть $w': \{\star\} \to (A_\star \multimap B_\star)$ такой морфизм, что $ev_{A,B}(w'(\star),a) = w_{B_\star}(\star)$ для любого $a \in A$. В этом случае для морфизма $\widetilde{w}: \{\star\} \times A \to B$ такого, что $\widetilde{w}(\star,a) = w_{B_\star}(\star)$ для любого $a \in A$, существует единственный морфизм $k: \{\star\} \to (A \multimap B)$ такой, что $\widetilde{w} = ev_{A,B} \circ (k \times id_A)$, то есть $ev_{A,B}(w_{A_\star \multimap B_\star}(\star),a) = ev_{A,B}(w'(\star),a) = w_{B_\star}(\star) = \widetilde{w}(\star,a) = ev_{A,B}(k(\star),a)$ для любого $a \in A$. Следовательно, $w_{A_\star \multimap B_\star} = k = w'$. Определим $ev_{A_\star,B_\star}: (A_\star \multimap B_\star) \wedge A_\star \to B_\star$ следующим образом: $ev_{A_\star,B_\star}(u,a) = ev_{A,B}(u,a)$ для любых $u \in A_\star \multimap B_\star$, $a \in A_\star$. Корректность определения $ev_{A_\star,B_\star}$ следует из равенств
$$
    (ev_{A_\star,B_\star} \circ w_{(A_\star \multimap B_\star) \wedge A_\star})(\star) = ev_{A_\star,B_\star}(w_{(A_\star \multimap B_\star) \wedge A_\star}(\star)) = ev_{A_\star,B_\star}(w_{A_\star \multimap B_\star}(\star),a) = w_{B_\star}(\star)
$$
для любого $a \in A_\star$. Пусть $f: X_\star \wedge A_\star \to B_\star$ --- морфизм категории $C_\star$. По морфизму $f$ построим морфизм $\overline{f}: X \times A \to B$ следующим образом: $\overline{f}(x,a) = 
\begin{cases}
    f(x,a),&\text{если } x \ne w_{X_\star}(\star), a \ne w_{A_\star}(\star),\\
    w_{B_\star}(\star),&\text{иначе}
\end{cases}    
$. Заметим, что $f(x,a) = \overline{f}(x,a)$ для любых $(x,a) \in X_\star \wedge A_\star$. В  этом случае из замкнутости категории $C$ следует, что существует единственный морфизм $\overline{h}: X \to (A \multimap B)$ такой, что $\overline{f} = ev_{A,B} \circ (\overline{h} \times id_A)$, то есть $\overline{f}(x,a) = ev_{A,B}(\overline{h}(x),a)$ для любых $x \in X$, $a \in A$. Следовательно,  $f(x,a) = \overline{f}(x,a) = ev_{A,B}(\overline{h}(x),a) = ev_{A_\star,B_\star}(h,a)$ для любых $x \in X_\star$, $a \in A_\star$, где $h: X_\star \to (A_\star \multimap B_\star)$ определим следующим образом: $h(x) = \overline{h}(x)$ для любых $x \in X_\star$. Таким образом, $f = ev_{A_\star,B_\star} \circ (h \times id_{A_\star})$. Корректность определения $h$, то есть $h(w_{X_\star}(\star)) = w_{A_\star \multimap B_\star}(\star)$ следует из равенств:
$$
    ev_{A,B}(h(w_{X_\star}(\star)),a) = f(w_{X_\star}(\star),a) = w_{B_\star}(\star)
$$
для любых $a \in A_\star$, поскольку $w_{A_\star \multimap B_\star}(\star)$ единственный морфизм такой, что $ev_{A,B}(w_{A_\star \multimap B_\star}(\star),a) = w_{B_\star}(\star)$.

Определим функторы $F: Chu_\wedge(C_\star) \to Chu_\sim(C_\star)$ и $G: Chu_\sim(C_\star) \to Chu_\wedge(C_\star)$ следующим образом:
$$
    F(r_1) = r_1, F(f^+,f^-,f^0) = (f^+,f^-,\widetilde{f^0}),
$$
$$
    G(r_1') = r_1', 
    G(f'^+,f'^-,\widetilde{f'^0}) = (f'^+,f'^-,\overline{\widetilde{f'^0}}),
$$
где $r_1: A_1 \wedge X_1 \to D_1 \in Ob(Chu_\wedge(C_\star))$, $r_1': A_1' \wedge X_1' \to D_1' \in Ob(Chu_\sim(C_\star))$, $(f^+,f^-,f^0) \in Hom_{Chu_\wedge(C_\star)}(r_1,r_2)$, $(f'^+,f'^-,\widetilde{f'^0}) \in Hom_{Chu_\sim(C_\star)}(r_1',r_2')$, $$\overline{\widetilde{f'^0}}(d) = 
\begin{cases}
    r_2(f^+(a),x),&\text{если } \exists a \in A_1, x \in X_2: d = r_1(a,f^-(x))\\
    w_{D_2}(\star),&\text{иначе.}
\end{cases}$$
Покажем, что $G$ --- правый сопряженный функтор к функтору $F$.

\section*{Категория $Chu_\sim(C_\star)$}

\begin{theorem}\label{epimorphism-c}
    Пусть $r_1: A_1 \wedge X_1 \to D_1$ и $r_2: A_2 \wedge X_2 \to D_2$ --- объекты категории $Chu_\sim(C_\star)$. Преобразование $(f^+,f^-,\fo): r_1 \to r_2$ категории $Chu_\sim(C_\star)$ является эпиморфизмом тогда и только тогда, когда $f^+: A_1 \to A_2$ --- эпиморфизм, $f^-: X_2 \to X_1$ --- мономорфизм категории $C_{\star}$.
\end{theorem}
\begin{proof}
    \textbf{Необходимость.} Пусть $(f^+,f^-,\fo): r_1 \to r_2$ --- эпиморфизм категории $Chu_\sim(C_\star)$.

    Покажем, что $f^+$ --- эпиморфизм категории $C_\star$. Предположим, что $B_1 \ne A_2$, где $B_1 = f^+(A_1)$. Тогда существует $B_2$ --- подсистема $A_2$ такая, что $A_2 = B_1 \sqcup B_2$. Обозначим через $B_0 = \{\star\} \sqcup B_2$ алгебраическую систему сигнатуры $\Sigma$, где любая $n$-местная операция $g^{(n)}$ задается как $g^{(n)}(\varepsilon_1,\ldots,\varepsilon_n) = 
    \begin{cases}
        \star, & \text{если } \exists i: \varepsilon_i = \star,\\
        g^{(n)}(\varepsilon_1,\ldots,\varepsilon_n) \in B_2, & \text{если } \forall i\ \varepsilon_i \in B_2.
    \end{cases}$ Определим объект $w: B_0 \wedge \{\star\} \to \{\star\}$ и морфизмы $(f_1,\star_{X_2},\widetilde{T_{D_2}}), (f_2,\star_{X_2},\widetilde{T_{D_2}}): r_2 \to w$ категории $Chu_\sim(C_\star)$ следующим образом: $f_1(a) = \star$, $f_2(a) = 
    \begin{cases}
        \star, & \text{если } a \in B_1,\\
        a, & \text{если } a \in B_2,
    \end{cases}$, для любых $a \in A_2$, $\star_{X_2}: \{\star\} \to X_2$ --- морфизм из определения объекта $(X_2,\star_{X_2})$ категории $C_\star$, $T_{D_2}: D_2 \to \{\star\}$ --- единственный морфизм в $\{\star\}$ поскольку $\{\star\}$ --- терминальный объект категории $C$. Корректность определения морфизмов $(f_1,\star_{X_2},\widetilde{T_{D_2}})$, $(f_2,\star_{X_2},\widetilde{T_{D_2}})$ следует из равенств:
    $$
        T_{D_2}(r_2(a,\star_{X_2}(\star))) = \star = w(f_1(a),\star) = w(f_2(a),\star)
    $$
    для любых $a \in A_2$. Если $a \in A_1$, то $f^+(a) \in B_1$ и $f_1(f^+(a)) = f_2(f^+(a))$, то есть $f_1 \circ f^+ = f_2 \circ f^+$. Тогда $(f_1,\star,\widetilde{T_{D_2}}) \circ (f^+,f^-,\fo) = (f_2,\star,\widetilde{T_{D_2}}) \circ (f^+,f^-,\fo)$. Поскольку $(f^+,f^-,\fo)$ --- эпиморфизм категории $Chu_\sim(C_\star)$, то $f_1 = f_2$. Противоречие.

    Покажем, что $f^-$ --- мономорфизм категории $C_\star$. Предположим, что существуют различные $x_1, x_2 \in X_2$ такие, что $f^-(x_1) = f^-(x_2)$. Пусть $a \in A_2$. Покажем, что $r_2(a,x_1) = r_2(a,x_2)$. Так как $f^+$ --- эпиморфизм, то существует $a' \in A_1$ такой, что $f^+(a') = a$. Тогда
    $$
        r_2(a,x_1) = r_2(f^+(a'),x_1) = f^0(r_1(a',f^-(x_1))) = f^0(r_1(a',f^-(x_2))) = r_2(f^+(a'),x_2) = r_2(a,x_2).
    $$
    Определим объект $w: A_2 \wedge \{\star,x_0\} \to D_2$ и морфизмы $(e_{A_2},g_1,\widetilde{e_{D_2}}), (e_{A_2},g_2,\widetilde{e_{D_2}}): r_2 \to w$ категории $Chu_\sim(C_\star)$ следующим образом: $w(a,\star) = \star$, $w(a,x_0) = r_2(a,x_1)$, $g_1(\star) = g_2(\star) = \star$, $g_1(x_0) = x_1$, $g_2(x_0) = x_2$ для любых $a \in A_2$. Корректность определения морфизмов $(e_{A_2},g_1,\widetilde{e_{D_2}}), (e_{A_2},g_2,\widetilde{e_{D_2}})$ следует из равенств:
    $$
        w(a,\star) = r_2(a,g_1(\star)) = r_2(a,g_2(\star)) = r_2(a,\star) = \star,
    $$
    $$
        w(a,x_0) = r_2(a,g_1(x_0)) = r_2(a,x_1) = r_2(a,x_2) = r_2(a,g_2(x_0))
    $$
    для любого $a \in A_2$. Так как
    $$
        (f^- \circ g_1)(\star) = f^-(g_1(\star)) = \star = f^-(g_2(\star)) = (f^- \circ g_2)(\star),
    $$
    $$
        (f^- \circ g_1)(x_0) = f^-(g_1(x_0)) = f^-(x_1) = f^-(x_2) = f^-(g_2(x_0)) = (f^- \circ g_2)(x_0),
    $$
    то $(e_{A_2},g_1,\widetilde{e_{D_2}}) \circ (f^+,f^-,\fo) = (e_{A_2},g_2,\widetilde{e_{D_2}}) \circ (f^+,f^-,\fo)$. Поскольку $(f^+,f^-,\fo)$ --- эпиморфизм категории $Chu_\sim(C_\star)$, то $g_1 = g_2$. Противоречие.

    \textbf{Достаточность.} Пусть $(f^+,f^-,\fo): r_1 \to r_2$ --- преобразование Чу категории $Chu_\sim(C_\star)$, где $f^+$ --- эпиморфизм, $f^-$ --- мономорфизм категории $C_{\star}$. Предположим, что $(f_1,g_1,\widetilde{h_1}), (f_2,g_2,\widetilde{h_2}): r_2 \to w$ --- преобразования Чу категории $Chu_\sim(C_\star)$ такие, что $(f_1,g_1,\widetilde{h_1}) \circ (f^+,f^-,\fo) = (f_2,g_2,\widetilde{h_2}) \circ (f^+,f^-,\fo)$, где $w: E \wedge Z \to P$ --- пространство Чу категории $Chu_\sim(C_\star)$. Тогда $f_1 \circ f^+ = f_2 \circ f^+$, $f^- \circ g_1 = f^- \circ g_2$. Поскольку $f^+$ --- эпиморфизм, $f^-$ --- мономорфизм категории $C_\star$, то $f_1 = f_2$, $g_1 = g_2$. Для любого $d \in D_1$ если $d = r_1(a,(f^- \circ g_1)(z)) = r_2(a,(f^- \circ g_2)(z))$, то $(h_1 \circ f^0)(d) = (h_2 \circ f^0)(d) = w((f_1 \circ f^+)(a), z) = w((f_2 \circ f^+)(a), z)$. Покажем, что для любого $d \in D_2$ если $d = r_2(a,g_1(z))$, то $h_1(d) = h_2(d) = w(f_1(a),z)$. Пусть $d \in D_2$ и $d = r_2(a,g_1(z))$ для некоторых $a \in A_2$, $z \in Z$. Так как $f^+$ --- эпиморфизм, то $a = f(a')$ для некоторого $a' \in A_1$. Тогда 
    \begin{multline*}
        h_1(d) = h_1(r_2(a,g_1(z))) = h_1(r_2(f(a'),g_1(z))) = w(f_1(f^+(a')),z) =\\=
        w(f_2(f^+(a')),z) = h_2(r_2(f^+(a'),g_2(z))) = h_1(r_2(a,g_1(z))) = h_2(d).
    \end{multline*}
    Следовательно, $(f_1,g_1,\widetilde{h_1}) = (f_2,g_2,\widetilde{h_2})$ в категории $Chu_\sim(C_\star)$ и преобразование $(f^+,f^-,\fo): r_1 \to r_2$ является эпиморфизмом категории $Chu_\sim(C_\star)$.
\end{proof}

\begin{theorem}\label{monomorphism-c}
    Пусть $r_1: A_1 \wedge X_1 \to D_1$ и $r_2: A_2 \wedge X_2 \to D_2$ --- объекты категории $Chu_\sim(C_\star)$. Преобразование $(f^+,f^-,\fo): r_1 \to r_2$ категории $Chu_\sim(C_\star)$ является мономорфизмом тогда и только тогда, когда $f^+: A_1 \to A_2$ --- мономорфизм, $f^-: X_2 \to X_1$ --- эпиморфизм категории $C_\star$.
\end{theorem}
\begin{proof}
    \textbf{Необходимость.} Пусть $(f^+,f^-,\fo): r_1 \to r_2$ --- мономорфизм категории $Chu_\sim(C_\star)$.

    Покажем, что $f^-$ --- эпиморфизм категории $C_\star$. Предположим, что $Y_1 \ne X_1$, где $Y_1 = f^-(X_2)$. Тогда существует $Y_2$ --- подсистема $X_1$ такая, что $X_1 = Y_1 \sqcup Y_2$. Через $Y_0$ обозначим факторсистему алгебраической системы $X_1$ по конгруэнции $\theta$ такой, что $(y,y') \in \theta \Leftrightarrow y, y' \in Y_1$. Определим объект $w: A_1 \wedge (Y_0 \times X_1) \to D_1$ и морфизмы $(e_{A_1},g_1,\widetilde{e_{D_1}}), (e_{A_1},g_2,\widetilde{e_{D_1}}): w \to r_1$ категории $Chu_\sim(C_\star)$ следующим образом: $w(a,(x_0,x)) = r_1(a,x)$, $g_1(x) = (Y_1,x)$, $g_2(x) = (x/\theta, x)$, для любых $a \in A_1$, $x \in X_1$, $x_0 \in X_0$. Из определения объекта $w$ категории $Chu_\sim(C_\star)$ следует равенство
    $$
        w(a,g_1(x)) = w(a,g_2(x)) = r_1(a,x),
    $$ 
    для любых $a \in A_1$, $x \in X_1$, что доказывает корректность определения морфизмов $(e_{A_1},g_1,\widetilde{e_{D_1}})$, $(e_{A_1},g_2,\widetilde{e_{D_1}})$. Если $x \in X_2$, то $f^-(x) \in X_1$ и $g_1(f^-(x)) = g_2(f^-(x))$, то есть $g_1 \circ f^- = g_2 \circ f^-$. Тогда $(f^+,f^-,\fo) \circ (e_{A_1},g_1,\widetilde{e_{D_1}}) = (f^+,f^-,\fo) \circ (e_{A_1},g_2,\widetilde{e_{D_1}})$. Поскольку $(f^+,f^-,\fo)$ --- мономорфизм категории $Chu_\sim(C_\star)$, то $g_1 = g_2$. Противоречие.

    Покажем, что $f^+$ --- мономорфизм категории $C_\star$. Предположим, что существуют различные $a_1, a_2 \in A_1$ такие, что $f^+(a_1) = f^+(a_2)$. Определим объект $w: \{\star,a',a''\} \wedge X_1 \to D_1'$, где $D_1' = r_1(\{a_1,a_2\} \wedge X_1)$ и морфизмы $(f_1,e_{X_1},\widetilde{h_1}), (f_2,e_{X_1},\widetilde{h_2}): w \to r_1$ категории $Chu_\sim(C_\star)$ следующим образом $w(\star,x) = \star$, $w(a',x) = r_1(a_1,x)$, $w(a'',x) = r_1(a_2,x)$, $f_1(\star) = f_2(\star) = \star$, $f_1(a') = f_2(a'') = a_1$, $f_1(a'') = f_2(a') = a_2$, $h_1(r_1(a_1,x)) = h_2(r_1(a_2,x)) = r_1(a_1,x)$, $h_1(r_1(a_2,x)) = h_2(r_1(a_1,x)) = r_1(a_2,x)$ для любых $x \in X_1$. Корректность определения морфизмов $(f_1,e_X,\widetilde{h_1}), (f_2,e_X,\widetilde{h_2})$ следует из равенств
    $$
        h_1(w(\star,x)) = h_2(w(\star,x)) = \star = r_1(f_1(\star),x) = r_1(f_2(\star),x),
    $$
    $$
        h_1(w(a',x)) = h_1(r_1(a_1,x)) = r_1(a_1,x) = r_1(f_1(a'),x),
    $$
    $$
        h_2(w(a',x)) = h_2(r_1(a_1,x)) = r_1(a_2,x) = r_1(f_2(a'),x),
    $$
    $$
        h_1(w(a'',x)) = h_1(r_1(a_2,x)) = r_1(a_2,x) = r_1(f_1(a''),x),
    $$
    $$
        h_2(w(a'',x)) = h_2(r_1(a_2,x)) = r_1(a_1,x) = r_1(f_2(a''),x)
    $$
    для любых $x \in X_1$.
    Так как $(f^+ \circ f_1)(\star) = \star = (f^+ \circ f_2)(\star)$ и $f^+(f_1(a')) = f^+(f_2(a'')) = f^+(a_1) = f^+(a_2) = f^+(f_1(a'')) = f^+(f_2(a'))$,
    то $f^+ \circ f_1 = f^+ \circ f_2$. Из того, что $f^-$ --- эпиморфизм, следует, что $x = f^-(y)$ для некоторого $y \in X_2$ для любого $x \in X_1$. Тогда из равенств
    $$
        (f^0 \circ h_1)(w(\star,x)) = \star = (f^0 \circ h_2)(w(\star,x)),
    $$
    \begin{multline*}
        (f^0 \circ h_1)(w(a',x)) = f^0(h_1(r_1(a_1,x))) = f^0(r_1(a_1,f^-(y))) = r_2(f^+(f_1(a')),x)  =\\= 
        r_2(f^+(a_1),y) = r_2(f^+(a_2),y) = f^0(r_1(a_2,f^-(y))) = f^0(h_2(r_1(a_1,x))) = (f^0 \circ h_2)(w(a',x)),
    \end{multline*}
    \begin{multline*}
        (f^0 \circ h_1)(w(a'',x)) = f^0(h_1(r_1(a_2,x))) = f^0(r_1(a_2,f^-(y))) = r_2(f^+(f_1(a'')),x)  =\\= 
        r_2(f^+(a_2),y) = r_2(f^+(a_1),y) = f^0(r_1(a_1,f^-(y))) = f^0(h_2(r_1(a_2,x))) = (f^0 \circ h_2)(w(a'',x))
    \end{multline*}
    для любых $x \in X_1$ следует, что для любого $d' \in D_1'$ если $d' = w(a,x)$, то $(f^0 \circ h_1)(d') = (f^0 \circ h_2)(d') = r_2((f^+ \circ f_1)(a),x)$. Следовательно, $(f^+,f^-,\fo) \circ (f_1,e_X,\widetilde{h_1}) = (f^+,f^-,\fo) \circ (f_2,e_X,\widetilde{h_2})$. Поскольку $(f^+,f^-,\fo)$ --- мономорфизм категории $Chu_\sim(C_\star)$, то $f_1 = f_2$, то есть $a_1 = a_2$. Противоречие.

    \textbf{Достаточность.} Пусть $(f^+,f^-,\fo): r_1 \to r_2$ --- преобразование Чу категории $Chu_\sim(C_\star)$, где $f^+$ --- мономорфизм, $f^-$ --- эпиморфизм категории $C_\star$. Предположим, что $(f_1,g_1,\widetilde{h_1}), (f_2,g_2,\widetilde{h_2}): w \to r_1$ --- преобразования Чу категории $Chu_\sim(C_\star)$ такие, что $(f^+,f^-,\fo) \circ (f_1,g_1,\widetilde{h_1}) = (f^+,f^-,\fo) \circ (f_2,g_2,\widetilde{h_2})$, где $w: E \wedge Z \to P$ --- пространство Чу категории $Chu_\sim(C_\star)$. Тогда $f^+ \circ f_1 = f^+ \circ f_2$, $g_1 \circ f^- = g_1 \circ f^-$ и для любого $p \in P$ если $p = w(e,(g_1 \circ f^-)(y)) = w(e,(g_2 \circ f^-)(y))$, то $(f^0 \circ h_1)(p) = (f^0 \circ h_2)(p) = r_2((f^+ \circ f_1)(e), y) = r_2((f^+ \circ f_2)(e), y)$. Покажем, что для любого $p \in P$ если $p = w(e,g_1(x))$, то $h_1(p) = h_2(p) = r_1(f_1(e),x)$. Пусть $p \in P$ и $p = w(e,g_1(x))$ для некоторых $e \in E$, $x \in X_1$. Поскольку $f^+$ --- мономорфизм, $f^-$ --- эпиморфизм категории $C_\star$, то $f_1 = f_2$, $g_1 = g_2$. Тогда 
    $$
        h_1(p) = h_1(w(e,g_1(x))) = r_1(f_1(e),x) = r_1(f_2(e),x) = h_2(w(e,g_2(x))) = h_2(p).
    $$
    Следовательно, $(f_1,g_1,\widetilde{h_1}) = (f_2,g_2,\widetilde{h_2})$ в категории $Chu_\sim(C_\star)$ и преобразование $(f^+,f^-,\fo): r_1 \to r_2$ является мономорфизмом категории $Chu_\sim(C_\star)$.
\end{proof}

\begin{theorem}\label{coproduct-c}
    Пусть $r_i: A_i \wedge X_i \to D_i$ --- объекты категории $Chu_\sim(C_\star)$, $i \in I$. Объект $r: \coprod_{i \in I} A_i \wedge \prod_{i \in I} X_i \to \coprod_{i \in I} D_i$ категории $Chu_\sim(C_\star)$ вместе с семейством $\{(q^+_i,q^-_i,\widetilde{q^0_i}): r \to r_i \mid i \in I\}$ преобразований Чу категории $Chu_\sim(C_\star)$, где $r(a,x) = r_i(a,q^-_i(x)) = r_i(a, x_i)$, $q^+_i: A_i \to \coprod_{i \in I} A_i$, $q^0_i: D_i \to \coprod_{i \in I} D_i$ --- естественные вложения, $q^-_i: \prod_{i \in I} X_i \to X_i$ --- каноническая проекция для любых $i \in I$, $a \in A_i$, $x \in \prod_{i \in I} X_i$, является копроизведением объектов $r_i$, $i \in I$, категории $Chu_\sim(C_\star)$.
\end{theorem}
\begin{proof}
    Пусть условия теоремы выполнены. Поскольку $q^0_i(r_i(a,q^-_i(x))) = r_i(a,x_i) = r(a,x) = r(q^+_i(a),x)$ для любых $a \in A_i$, $x \in \prod_{i \in I} X_i$, $i \in I$, то $(q^+_i, q^-_i, \widetilde{q^0_i})$ --- преобразование Чу категории $Chu_\sim(C_\star)$ для любого $i \in I$.

    Пусть $t: B \wedge Y \to D$ --- объект категории $Chu_\sim(C_\star)$, $(f^+_i,f^-_i,\widetilde{f^0_i}): r_i \to t$ --- преобразование Чу категории $Chu_\sim(C_\star)$ для любого $i \in I$. Так как $\coprod_{i \in I} A_i$ и $\coprod_{i \in I} D_i$ --- копроизведения $A_i$, $ i \in I$, и $D_i$, $i \in I$, соответственно, а $\prod_{i \in I} X_i$ --- произведение $A_i$, $i \in I$ в категории $C_\star$, то существуют единственные морфизмы $f^+: \coprod_{i \in I} A_i \to B$, $f^-: Y \to \prod_{i \in I} X_i$, $f^0: \coprod_{i \in I} D_i \to D$ категории $C_\star$ такие, что $f^+_i = f^+ \circ q^+_i$, $f^-_i = q^-_i \circ f^-$ и $f^0_i = f^0 \circ q^0_i$ для любого $i \in I$, то есть $f^+(a) = f^+_i(a)$, $f^-(y)_i = f^-_i(y)$ и $f^0(d) = f^0_i(d)$ для любых $a \in A_i$, $y \in Y$, $d \in D_i$. 

    Покажем, что $(f^+,f^-,\widetilde{f^0})$ --- преобразование Чу категории $Chu_\sim(C_\star)$, то есть для любых $a \in \coprod_{i \in I} A_i$, $y \in Y$ имеет место равенство $f^0(r(a,f^-(y))) = t(f^+(a),y)$. Так как $(q^+_i,q^-_i,\widetilde{q^0_i})$ --- преобразование Чу категории $Chu_\sim(C_\star)$, то $r(q^+_i(a),f^-(y)) = q^0_i(r_i(a,(q^-_i \circ f^-)(y)))$ для любых $a \in A_i$, $y \in Y$. Из того, что $f^0_i = f^0 \circ q^0_i$ и $f^-_i = q^-_i \circ f^-$ следует, что $f^0(r(a,f^-(y))) = (f^0 \circ q^0_i)(r_i(a,(q^-_i \circ f^-)(y))) = f^0_i(r_i(a,f^-_i(y)))$ для любых $a \in A_i$, $y \in Y$. Так как $(f^+_i,f^-_i,\widetilde{f^0_i})$ --- преобразование Чу категории $Chu_\sim(C_\star)$, то $f^0_i(r_i(a,f^-_i(y))) = t(f^+_i(a),y)$ для любых $a \in A_i$, $y \in Y$. Поскольку $f^+(a) = f^+_i(a)$, то $t(f^+_i(a),y) = t(f^+(a),y)$ для любых $a \in A_i$, $y \in Y$. Таким образом, имеет место равенство $f^0(r(a,f^-(y))) = t(f^+(a),y)$ для любых $a \in \coprod_{i \in I} A_i$, $y \in Y$.
    Следовательно, $(f^+,f^-,\widetilde{f^0}) \circ (q^+_i, q^-_i, \widetilde{q^0_i}) = (f^+_i, f^-_i, \widetilde{f^0_i})$ для любого $i \in I$ в категории $Chu_\sim(C_\star)$.
\end{proof}

\begin{theorem}\label{iso-c}
    Обекты $r: A \wedge X \to D$ и $\tilde{r}: A \wedge X \to r(A \wedge X)$ категории $Chu_\sim(C_\star)$, где $\tilde{r}(a,x) = r(a,x)$ для любых $a \in A$, $x \in X$, изоморфны.
\end{theorem}
\begin{proof}
    Построим морфизм $(f^+,f^-,\widetilde{f^0}): r \to \tilde{r}$ следующим образом: $f^+ = e_A$, $f^- = e_X$, $f^0(d) = 
    \begin{cases}
        r(a,x),& \text{если } d \in r(A \wedge X),\\
        \star,& \text{иначе}.
    \end{cases}$
    Из равенства $f^0(r(a,x)) = r(a,x)$ для любых $a \in A$, $x \in X$ следует, что $(f^+,f^-,\widetilde{f^0})$ --- преобразование Чу категории $Chu_\sim(C_\star)$. Построим морфизм $(f'^+,f'^-,\widetilde{f'^0}): \tilde{r} \to r$ следующим образом: $f'^+ = e_A$, $f'^- = e_X$, $f'^0$ --- естественное вложение. Из равенства $f'^0(r(a,x)) = r(a,x)$ для любых $a \in A$, $x \in X$ следует, что $(f'^+,f'^-,\widetilde{f'^0})$ --- преобразование Чу категории $Chu_\sim(C_\star)$.

    Поскольку $(f'^+,f'^-,\widetilde{f'^0}) \circ (f^+,f^-,\widetilde{f^0}) = (e_A,e_X,\widetilde{f'^0 \circ f^0})$ и для любого $d \in D$ если $d = r(a,x)$, то $f'^0(d) = f^0(d) = r(a,x)$, то $(f'^+,f'^-,\widetilde{f'^0}) \circ (f^+,f^-,\widetilde{f^0}) = e_r$. Поскольку $(f^+,f^-,\widetilde{f^0}) \circ (f'^+,f'^-,\widetilde{f'^0}) = (e_A,e_X,\widetilde{f^0 \circ f'^0})$ и для любого $d \in r(A \wedge X)$ если $d = r(a,x)$, то $f^0(d) = f'^0(d) = r(a,x)$, то $(f^+,f^-,\widetilde{f^0}) \circ (f'^+,f'^-,\widetilde{f'^0}) = e_{\tilde{r}}$.
\end{proof}

\begin{theorem}\label{product-c}
    Пусть $r_i: A_i \wedge X_i \to D_i$ --- объекты категории $Chu_\sim(C_\star)$, $i \in I$. Объект $r: \prod_{i \in I} A_i \wedge \coprod_{i \in I} X_i \to \prod_{i \in I} r_i(A_i \wedge X_i)$ категории $Chu_\sim(C_\star)$ вместе с семейством $\{(p^+_i,p^-_i,\widetilde{p^0_i}): r \to r_i \mid i \in I\}$ преобразований Чу категории $Chu_\sim(C_\star)$, где $\pi_j(r(a,x_i)) = 
    \begin{cases}
        r_i(p^+_i(a),x_i),& \text{если } i = j,\\
        \star,& \text{если } i \ne j,
    \end{cases}$
    $\pi_i: \prod_{i \in I} r_i(A_i \wedge X_i) \to r_i(A_i \wedge X_i)$ --- каноническая проекция, $p^+_i: \prod_{i \in I} A_i \to A_i$ --- каноническая проекция, $p^-_i: X_i \to \coprod_{i \in I} X_i$ --- естественное вложение, $p^0_i = m_i \circ \pi_i$, $m_i: r_i(A_i \wedge X_i) \to D_i$ --- естественное вложение, для любых $a \in \prod_{i \in I} A_i$, $x_i \in X_i$, $i,j \in I$, является произведением объектов $r_i$, $i \in I$, категории $Chu_\sim(C_\star)$.
\end{theorem}
\begin{proof}
    Пусть условия теоремы выполнены. Поскольку $p^0_i(r(a,p^-_i(x_i))) = m_i(\pi_i(r(a,p^-_i(x_i)))) = \pi_i(r(a,p^-_i(x_i))) = r_i(p^+_i(a),x_i)$ для любых $a \in \prod_{i \in I} A_i$, $x_i \in X_i$, $i \in I$, то $(p^+_i,p^-_i,\widetilde{p^0_i})$ --- преобразование Чу категории $Chu_\sim(C_\star)$ для любого $i \in I$.

    Пусть $t: B \wedge Y \to D$ --- объект категории $Chu_\sim(C_\star)$, $(f^+_i,f^-_i,\widetilde{f^0_i}): t \to r_i$ --- преобразование Чу категории $Chu_\sim(C_\star)$ для любого $i \in I$. Для любого $f^0_i: D \to D_i$, $i \in I$, построим морфизм $\overline{f^0_i}: D \to r_i(A_i \wedge X_i)$ категории $C_\star$ следующим образом: $\overline{f^0_i}(d) = 
    \begin{cases}
        f^0_i(d),& \text{если } d = t(b,f^-_i(x_i)) \text{ для некоторых } b \in B, x_i \in X_i,\\
        \star,& \text{иначе},
    \end{cases}$
    для любого $d \in D$. Поскольку $(f^+_i,f^-_i,\widetilde{f^0_i})$ --- преобразование Чу категории $Chu_\sim(C_\star)$ для любого $i \in I$, то $m_i(\overline{f^0_i}(t(b,f^-_i(x_i)))) = f^0_i(t(b,f^-_i(x_i))) = r_i(f^+_i(b),x_i)$ для любых $b \in B$, $x_i \in X_i$. Следовательно, $(f^+_i,f^-_i,\widetilde{m_i \circ \overline{f^0_i}}): t \to r_i$ --- преобразование Чу категории $Chu_\sim(C_\star)$ для любого $i \in I$.

    Так как $\prod_{i \in I} A_i$ и $\prod_{i \in I} r_i(A_i \wedge X_i)$ --- произведения $A_i$, $ i \in I$, и $r_i(A_i \wedge X_i)$, $i \in I$, соответственно, а $\coprod_{i \in I} X_i$ --- копроизведение $X_i$, $i \in I$, в категории $C_\star$, то существуют единственные морфизмы $f^+: B \to \prod_{i \in I} A_i$, $f^-: \coprod_{i \in I} X_i \to Y$, $\overline{f^0}: D \to \prod_{i \in I} r_i(A_i \wedge X_i)$ категории $C_\star$ такие, что $f^+_i = p^+_i \circ f^+$, $f^-_i = f^- \circ p^-_i$ и $\overline{f^0_i} = \pi_i \circ \overline{f^0}$ для любого $i \in I$, то есть $p^+_i(f^+(b)) = f^+_i(b)$, $f^-(x_i) = f^-_i(x_i)$ и $\pi_i(\overline{f^0}(d)) = \overline{f^0_i}(d)$ для любых $b \in  B$, $x_i \in X_i$, $d \in D$, $i \in I$.

    Покажем, что $(f^+,f^-,\widetilde{\overline{f^0}})$ --- преобразование Чу категории $Chu_\sim(C_\star)$, то есть для любых $b \in B$, $x_i \in X_i$, $i \in I$, имеет место равенство $\overline{f^0}(t(b,f^-(x_i))) = r(f^+(b),x_i)$. Поскольку $\prod_{i \in I} r_i(A_i \wedge X_i)$ вместе с семейством $\{\pi_i: \prod_{i \in I} r_i(A_i \wedge X_i) \to r_i(A_i \wedge X_i) \mid i \in I\}$ морфизмов категории $C_\star$ является произведением объектов $r_i(A_i \wedge X_i)$, $i \in I$, то для любых $b \in B$, $x \in X_i$ равенство $\overline{f^0}(t(b,f^-(x_i))) = r(f^+(b),x_i)$ равносильно равенству $\pi_j(\overline{f^0}(t(b,f^-(x_i)))) = \pi_j(r(f^+(b),x_i))$ для любых $b \in B$, $x_i \in I$, $i,j \in I$. Тогда из равенств
    \begin{multline*}
        \pi_j(\overline{f^0}(t(b,f^-(x_i)))) = \overline{f^0}_j(t(b,f^-_i(x_i))) =\\= 
        \begin{cases}
            f^0_i(t(b,f^-_i(x_i))) = r_i(f^+_i(b),x_i) = r_j(f_j(b),x_i),& \text{если } i = j\\
            \star,& \text{если } i \ne j
        \end{cases} =\\=
        \pi_j(r(f^+(b),x_i))
    \end{multline*}
    для любых $b \in B$, $x_i \in X_i$, $i,j \in I$ следует, что $(f^+,f^-,\widetilde{\overline{f^0}})$ --- преобразование Чу категории $Chu_\sim(C_\star)$.

    Покажем, что $(p^+_i,p^-_i,\widetilde{p^0_i}) \circ (f^+,f^-,\widetilde{\overline{f^0}}) = (f^+_i,f^-_i,\widetilde{f^0_i})$ для любого $i \in I$, то есть $p^+_i \circ f^+ = f^+_i$, $f^- \circ p^-_i = f^-_i$ и для любого $d \in D$ если $d = t(b,f^-_i(x_i))$, то $(p^0_i \circ \overline{f^0})(d) = f^0_i(d) = r_i(f^+_i(b),x_i)$. Из определения $f^+$ и $f^-$ следует, что $p^+_i \circ f^+ = f^+_i$ и $f^- \circ p^-_i = f^-_i$ для любого $i \in I$. Пусть $d \in D$ и $d = t(b,f^-_i(x_i))$ для некоторых $b \in B$, $x_i \in X_i$, $i \in I$. Тогда
    \begin{multline*}
        (p^0_i \circ \overline{f^0})(d) = (m_i \circ \pi_i \circ \overline{f^0})(d) = (m_i \circ \overline{f^0_i})(d) = m_i(\overline{f^0_i}(d)) = \overline{f^0_i}(d) =\\=
        \overline{f^0_i}(t(b,f^-_i(x_i))) = f^0_i(t(b,f^-_i(x_i))) = f^0_i(d) = r_i(f^+_i(b),x_i).
    \end{multline*}
    Следовательно, $(p^+_i,p^-_i,\widetilde{p^0_i}) \circ (f^+,
    f^-,\widetilde{\overline{f^0}}) = (f^+_i,f^-_i,\widetilde{f^0_i})$ для любого $i \in I$.

    Покажем, что $(f^+,f^-,\widetilde{\overline{f^0}})$ --- единственный морфизм такой, что $(p^+_i,p^-_i,p^0_i) \circ (f^+,f^-,\widetilde{\overline{f^0}}) = (f^+_i,f^-_i,\widetilde{f^0_i})$ для любого $i \in I$. Пусть $(f'^+,f'^-,\widetilde{f'^0}): t \to r$ --- преобразование Чу категории $Chu_\sim(C_\star)$ такое, что $(p^+_i,p^-_i,\widetilde{p^0_i}) \circ (f'^+,f'^-,\widetilde{f'^0}) = (f^+_i,f^-_i,\widetilde{f^0_i})$ для любого $i \in I$, то есть $p^+_i \circ f'^+ = f^+_i$, $f'^- \circ p^-_i = f^-_i$ и для любого $d \in D$ если $d = t(b,f^-_i(x_i))$, то $(p^0_i \circ f'^0)(d) = f^0_i(d) = r_i(f^+_i(b),x_i)$ для любого $i \in I$. Из определения $f^+$ и $f^-$ следует, что $f'^+ = f^+$ и $f'^- = f^-$. Осталось показать, что $(f^+,f^-,\widetilde{f'^0}) = (f^+,f^-,\widetilde{\overline{f^0}})$, то есть для любого $d \in D$ если $d = t(b,f^-_i(x_i))$, то $f'^0(d) = \overline{f^0}(d) = r(f^+,x_i)$. Поскольку $\prod_{i \in I} r_i(A_i \wedge X_i)$ вместе с семейством $\{\pi_i: \prod_{i \in I} r_i(A_i \wedge X_i) \to r_i(A_i \wedge X_i) \mid i \in I\}$ морфизмов категории $C_\star$ является произведением объектов $r_i(A_i \wedge X_i)$, $i \in I$, то для любых $b \in B$, $x \in X_i$ равенства $f'^0(d) = \overline{f^0}(d) = r(f^+(b),x_i)$ равносильны равенствам $\pi_j(f'^0(d)) = \pi_j(\overline{f^0}(d)) = \pi_j(r(f^+(b),x_i))$ для любых $b \in B$, $x_i \in I$, $i,j \in I$.  Пусть $d \in D$ и $d = t(b,f^-_i(x_i))$. Тогда если $i = j$, то
    $$
        \pi_j(f'^0(d)) = m_i(\pi_i(f'^0(d))) = p^0_i(f'^0(d)) = p^0_i(\overline{f^0}(d)) = f^0_i(d) = r_i(f^+_i(b),x_i),
    $$
    иначе, если $i \ne j$, то
    $$
        \pi_j(f'^0(d)) = \pi_j(f'^0(t(b,f'^-(x_i)))) = \pi_j(r(f'^+(b),x_i)) = \star = \pi_j(r(f^+(b),x_i)) = \pi_j(\overline{f^0}(t(b,f^-(x_i)))).
    $$
    Следовательно, $(f^+,f^-,\widetilde{\overline{f^0}}) = (f'^+,f'^-,\widetilde{f'^0})$ в категории $Chu_\sim(C_\star)$.
\end{proof}

\begin{remark}
    Пусть $\varphi_1, \varphi_2: U \to V$ --- морфизмы категории $C_\star$. Тогда отношение $\nu(\varphi_1,\varphi_2)$ на $U$ определяется следующим образом: $(u_1,u_2) \in \nu(\varphi_1,\varphi_2) \Leftrightarrow \varphi_1(u_1) = \varphi_2(u_2)$.
\end{remark}

\begin{theorem}
    Пусть $r_1: A_1 \wedge X_1 \to D_1$, $r_2: A_2 \wedge X_2 \to D_2$ --- объекты категории $Chu_\sim(C_\star)$ и $(f_1^+,f_1^-,\widetilde{f_1^0}), (f_2^+,f_2^-,\widetilde{f_2^0}): r_1 \to r_2$ --- преобразования Чу категории $Chu_\sim(C_\star)$. Тогда коуравнителем морфизмов $(f_1^+,f_1^-,\widetilde{f_1^0}), (f_2^+,f_2^-,\widetilde{f_2^0})$ является пространство Чу $t: Q \wedge E \to W$, где $Q = A_2/\nu(f_1^+,f_2^+)$, $W = D_2/\nu(f_1^0,f_2^0)$, $E = \{x \in X_2 \mid f_1^-(x) = f_2^-(x)\}$, $t(a/\nu(f_1^+,f_2^+),x) = r_2(a,x)/\nu(f_1^0,f_2^0)$ для любых $a \in A_2$, $x \in X_2$, вместе с морфизмом $(f^+,f^-,\widetilde{f^0}): r_2 \to t$, где $f^+$, $f^0$ --- канонические эпиморфизмы, $f^-$ --- естественное вложение. 
\end{theorem}
\begin{proof}
    Пусть условия теоремы выполнены. Введем обозначения $\overline{a} = a/\nu(f_1^+,f_2^+)$, $\overline{d} = d/\nu(f_1^0,f_2^0)$ для любых $a \in A_2$, $d \in D_2$. Корректность определения $t$ следует из равенств $f_1^0(r_1(a,x)) = r_2(f_1^+(a),x)$, $f_2^0(r_1(a,x)) = r_2(f_2^+(a),x)$ для любых $a \in A_1$, $x \in E$.

    Покажем, что $(f^+,f^-,\widetilde{f^0}): r_2 \to t$ является преобразованием Чу категории $Chu_\sim(C_\star)$. Так как имеют место равенства $t(f^+(a),x) = t(\overline{a},x) = \overline{r_2(a,x)}$ и $f^0(r_2(a,f^-(x))) = f^0(r_2(a,x)) = \overline{r_2(a,x)}$ для любых $a \in A_2$, $x \in E$, то $t(f^+(a),x) = f^0(r_2(a,f^-(x)))$ для любых $a \in A_2$, $x \in X$. Таким образом, $(f^+,f^-,\widetilde{f^0})$ является преобразованием Чу категории $Chu_\sim(C_\star)$.

    Пусть $t': Q' \wedge E' \to W'$ --- объект категории $Chu_\sim(C_\star)$ и $(f'^+,f'^-,\widetilde{f'^0}): r_2 \to t'$ --- преобразование Чу категории $Chu_\sim(C_\star)$ такое, что $(f'^+,f'^-,\widetilde{f'^0}) \circ (f_1^+,f_1^-,\widetilde{f_1^0}) = (f'^+,f'^-,\widetilde{f'^0}) \circ (f_2^+,f_2^-,\widetilde{f_2^0})$. Тогда $f'^+ \circ f_1^+ = f'^+ \circ f_2^+$, $f_1^- \circ f'^- = f_2^- \circ f'^-$ и для любого $d \in D_1$ если $d = r_1(a,(f_1^- \circ f'^-)(x))$, то $(f'^0 \circ f_1^0)(d) = (f'^0 \circ f_2^0)(d) = t((f'^+ \circ f_1^+)(a),x)$. Так как $Q,W$ являются коуравнителями, $E$ является уравнителем в категории $C_\star$, то существуют единственные морфизмы $u: Q \to Q'$, $w: W \to W'$, $E' \to E$ категории $C_\star$ такие, что $f'^+ = u \circ f^+$, $f'^0 = w \circ f^0$, $f'^- = f^- \circ v$. Покажем, что $(u,v,\widetilde{w}): t \to t'$ --- преобразование Чу категории $Chu_\sim(C_\star)$. Пусть $a \in A_2$, $x' \in E'$. Так как $u(\overline{a}) = (u \circ f^+)(a)$ и $f'^+ = u \circ f^+$, то $t'(u(\overline{a}),x') = t'((u \circ f^+)(a),x')$. Из того, что $(f'^+,f'^-,\widetilde{f'^0})$ --- преобразование Чу категории $Chu_\sim(C_\star)$ следует, что $t'(f'^+(a),x') = f'^0(r_2(a,f'^-(x')))$. С другой стороны, из того, что $\overline{a} = f^+(a)$ следует, что $w(t(\overline{a},v(x'))) = w(t(f^+(a),v(x')))$. Так как $(f^+,f^-,\widehat{f^0})$ --- преобразование Чу категории $Chu_\sim(C_\star)$, то $w(t(f^+(a),v(x'))) = (w \circ f^0)(r_2(a,(f^- \circ v)(x')))$. Поскольку $f'^0 = w \circ f^0$ и $f'^- = f^- \circ v$,, то $(w \circ f^0)(r_2(a,(f^- \circ v)(x'))) = f'^0(r_2(a,f'^-(x')))$. Таким образом, $t'(u(a),x') = w(t(\overline{a},v(x')))$. Следовательно, $(u,v,w)$ --- преобразование Чу категории $Chu_\sim(C_\star)$. Ясно, что $(u,v,w)$ --- единственное такое преобразование Чу, что $(f'^+,f'^-.\widetilde{f'^0}) = (u,v,w) \circ (f^+,f^-.\widetilde{f^0})$ 
\end{proof}

\begin{theorem}
    Пусть $r_1: A_1 \wedge X_1 \to D_1$, $r_2: A_2 \wedge X_2 \to D_2$ --- объекты категории $Chu_\sim(C_\star)$ и $(f_1^+,f_1^-,\widetilde{f_1^0}), (f_2^+,f_2^-,\widetilde{f_2^0}): r_1 \to r_2$ --- преобразования Чу категории $Chu_\sim(C_\star)$. Тогда уравнителем морфизмов $(f_1^+,f_1^-,\widetilde{f_1^0}), (f_2^+,f_2^-,\widetilde{f_2^0})$ является пространство Чу $t: E \wedge Q \to W$, где $Q = X_1/\nu(f_1^-,f_2^-)$, $w = D_1/\nu(f_1^0,f_2^0)$, $E = \{a \in A_1 \mid f_1^+(a) = f_2^+(a)\}$, $t(a/\nu(f_1^+,f_2^+),x) = r_2(a,x)/\nu(f_1^0,f_2^0)$ для любых $a \in A_2$, $x \in X_2$, вместе с морфизмом $(f+,f^-,\widetilde{f^0}): t \to r_1$, где $f^+$, $f^0$ ---  --- естественные вложения, $f^-$ --- канонический эпиморфизм. 
\end{theorem}

\end{document}

\section*{Категория $Chu_{\widetilde{Set}}$}

\begin{theorem}\label{epimorphism}
    Пусть $r: A \times X \to D$ и $t: B \times Y \to C$ --- объекты категории $Chu_{\widetilde{Set}}$. Преобразование $(f,g,h): r \to t$ категории $Chu_{\widetilde{Set}}$ является эпиморфизмом тогда и только тогда, когда $f: A \to B$ --- эпиморфизм, $g: Y \to X$ --- мономорфизм категории $Set_{\star}$.
\end{theorem}
\begin{proof}
    \textbf{Необходимость.} Пусть $(f,g,h): r \to t$ --- эпиморфизм категории $Chu_{\widetilde{Set}}$.

    Покажем, что $f$ --- эпиморфизм категории $Set_{\star}$. Предположим, что $B_1 \ne B$, где $B_1 = f(A)$. Через $B_0$ обозначим фактормножество множества $B$ по отношению эквивалентности $\sim$ такому, что $b \sim b' \Leftrightarrow b, b' \in B_1$. Определим объект $w: (B_0 \times B) \times Y \to C$ и морфизмы $(f_1,1_Y,1_C), (f_2,1_Y,1_C): t \to w$ категории $Chu_{\widetilde{Set}}$ следующим образом: $w((b_0,b),y) = t(b,y)$, $f_1(b) = (B_1,b)$, $f_2(b) = (b/\sim,b)$ для любых $b \in B$, $b_0 \in B_0$, $y \in Y$. Корректность определения морфизмов $(f_1,1_Y,1_C)$, $(f_2,1_Y,1_C)$ следует из равенств:
    $$
        t(b,y) = w(f_1(b),y) = w(f_2(b),y),
    $$
    где $b \in B$, $y \in Y$. Если $a \in A$, то $f(a) \in B_1$ и $f_1(f(a)) = f_2(f(a))$, то есть $f_1 \circ f = f_2 \circ f$. Тогда $(f_1,1_Y,1_C) \circ (f,g,h) = (f_2,1_Y,1_C) \circ (f,g,h)$. Поскольку $(f,g,h)$ --- эпиморфизм категории $Chu_{\widetilde{Set}}$, то $f_1 = f_2$. Противоречие.

    Покажем, что $g$ --- мономорфизм категории $Set_{\star}$. Предположим, что существуют различные $y_1, y_2 \in Y$ такие, что $g(y_1) = g(y_2)$. Пусть $b \in B$. Покажем, что $t(b,y_1) = t(b,y_2)$. Так как $f$ --- эпиморфизм, то существует $a \in A$ такой, что $f(a) = b$. Тогда
    $$
        t(b,y_1) = t(f(a),y_1) = h(r(a,g(y_1))) = h(r(a,g(y_2))) = t(f(a),y_2) = t(b,y_2).
    $$
    Определим объект $w: B \times \{\star,y_0\} \to C$ и морфизмы $(1_B,g_1,1_C), (1_B,g_2,1_C): t \to w$ категории $Chu_{\widetilde{Set}}$ следующим образом: $w(b,\star) = \star$, $w(b,y_0) = t(b,y_1)$, $g_1(\star) = g_2(\star) = \star$, $g_1(y_0) = y_1$, $g_2(y_0) = y_2$ для любых $b \in B$. Корректность определения морфизмов $(1_B,g_1,1_C), (1_B,g_2,1_C)$ следует из равенств:
    $$
        w(b,\star) = t(b,g_1(\star)) = t(b,g_2(\star)) = t(b,\star) = \star,
    $$
    $$
        w(b,y_0) = t(b,g_1(y_0)) = t(b,y_1) = t(b,y_2) = t(b,g_2(y_0))
    $$
    для любого $b \in B$. Так как
    $$
        (g \circ g_1)(\star) = g(g_1(\star)) = \star = g(g_2(\star)) = (g \circ g_2)(\star),
    $$
    $$
        (g \circ g_1)(y_0) = g(g_1(y_0)) = g(y_1) = g(y_2) = g(g_2(y_0)) = (g \circ g_2)(y_0),
    $$
    то $(1_B,g_1,1_C) \circ (f,g,h) = (1_B,g_2,1_C) \circ (f,g,h)$. Поскольку $(f,g,h)$ --- эпиморфизм категории $Chu_{\widetilde{Set}}$, то $g_1 = g_2$. Противоречие.

    \textbf{Достаточность.} Пусть $(f,g,h): r \to t$ --- преобразование Чу категории $Chu_{\widetilde{Set}}$, где $f$ --- эпиморфизм, $g$ --- мономорфизм категории $Set_{\star}$. Предположим, что $(f_1,g_1,h_1), (f_2,g_2,h_2): t \to w$ --- преобразования Чу категории $Chu_{\widetilde{Set}}$ такие, что $(f_1,g_1,h_1) \circ (f,g,h) = (f_2,g_2,h_2) \circ (f,g,h)$, где $w: E \times Z \to P$ --- пространство Чу категории $Chu_{\widetilde{Set}}$. Тогда $f_1 \circ f = f_2 \circ f$, $g \circ g_1 = g \circ g_2$ и для любого $d \in D$ если $d = r(a,(g \circ g_1)(z)) = r(a,(g \circ g_2)(z))$, то $(h_1 \circ h)(d) = (h_2 \circ h)(d) = w((f_1 \circ f)(a), z) = w((f_2 \circ f)(a), z)$. Покажем, что для любого $c \in C$ если $c = t(b,g_1(z))$, то $h_1(c) = h_2(c) = w(f_1(b),z)$. Пусть $c \in C$ и $c = t(b,g_1(z))$ для некоторых $b \in B$, $z \in Z$. Поскольку $f$ --- эпиморфизм, $g$ --- мономорфизм категории $Set_{\star}$, то $f_1 = f_2$, $g_1 = g_2$. Так как $f$ --- эпиморфизм, то $b = f(a)$ для некоторого $a \in A$. Тогда 
    \begin{multline*}
        h_1(c) = h_1(t(b,g_1(z))) = h_1(t(f(a),g_1(z))) = w(f_1(f(a)),z) =\\=
        w(f_2(f(a)),z) = h_2(t(f(a),g_2(z))) = h_1(t(b,g_1(z))) = h_2(c).
    \end{multline*}
    Следовательно, $(f_1,g_1,h_1) = (f_2,g_2,h_2)$ в категории $Chu_{\widetilde{Set}}$ и преобразование $(f,g,h): r \to t$ является эпиморфизмом категории $Chu_{\widetilde{Set}}$.
\end{proof}

\begin{theorem}\label{monomorphism}
    Пусть $r: A \times X \to D$ и $t: B \times Y \to C$ --- объекты категории $Chu_{\widetilde{Set}}$. Преобразование $(f,g,h): r \to t$ категории $Chu_{\widetilde{Set}}$ является мономорфизмом тогда и только тогда, когда $f: A \to B$ --- мономорфизм, $g: Y \to X$ --- эпиморфизм категории $Set_{\star}$.
\end{theorem}
\begin{proof}
    \textbf{Необходимость.} Пусть $(f,g,h): r \to t$ --- мономорфизм категории $Chu_{\widetilde{Set}}$.

    Покажем, что $g$ --- эпиморфизм категории $Set_{\star}$. Предположим, что $X_1 \ne X$, где $X_1 = g(Y)$. Через $X_0$ обозначим фактормножество множества $X$ по отношению эквивалентности $\sim$ такому, что $x \sim x' \Leftrightarrow x,x' \in X_1$. Определим объект $w: A \times (X_0 \times X) \to D$ и морфизмы $(1_A,g_1,1_D), (1_A,g_2,1_D): w \to r$ категории $Chu_{\widetilde{Set}}$ следующим образом: $w(a,(x_0,x)) = r(a,x)$, $g_1(x) = (X_1,x)$, $g_2(x) = (x/\sim, x)$, для любых $a \in A$, $x \in X$, $x_0 \in X_0$. Из определения объекта $w$ категории $Chu_{\widetilde{Set}}$ следует равенство
    $$
        w(a,g_1(x)) = w(a,g_2(x)) = r(a,x),
    $$ 
    где $a \in A$, $x \in X$, что доказывает корректность определения морфизмов $(1_A,g_1,1_D)$, $(1_A,g_2,1_D)$. Если $y \in Y$, то $g(y) \in X_1$ и $g_1(g(y)) = g_2(g(y))$, то есть $g_1 \circ g = g_2 \circ g$. Тогда $(f,g,h) \circ (1_A,g_1,1_D) = (f,g,h) \circ (1_A,g_2,1_D)$. Поскольку $(f,g,h)$ --- мономорфизм категории $Chu_{\widetilde{Set}}$, то $g_1 = g_2$. Противоречие.

    Покажем, что $f$ --- мономорфизм категории $Set_{\star}$. Предположим, что существуют различные $a_1, a_2 \in A$ такие, что $f(a_1) = f(a_2)$. Определим объект $w: \{\star,a',a''\} \times X \to D'$, где $D' = r(\{a_1,a_2\} \times X)$ и морфизмы $(f_1,1_X,h_1), (f_2,1_X,h_2): w \to r$ категории $Chu_{\widetilde{Set}}$ следующим образом $w(\star,x) = \star$, $w(a',x) = r(a_1,x)$, $w(a'',x) = r(a_2,x)$, $f_1(\star) = f_2(\star) = \star$, $f_1(a') = f_2(a'') = a_1$, $f_1(a'') = f_2(a') = a_2$, $h_1(r(a_1,x)) = h_2(r(a_2,x)) = r(a_1,x)$, $h_1(r(a_2,x)) = h_2(r(a_1,x)) = r(a_2,x)$ для любых $x \in X$. Корректность определения морфизмов $(f_1,1_X,h_1), (f_2,1_X,h_2)$ следует из равенств
    $$
        h_1(w(\star,x)) = h_2(w(\star,x)) = \star = r(f_1(\star),x) = r(f_2(\star),x),
    $$
    $$
        h_1(w(a',x)) = h_1(r(a_1,x)) = r(a_1,x) = r(f_1(a'),x),
    $$
    $$
        h_2(w(a',x)) = h_2(r(a_1,x)) = r(a_2,x) = r(f_2(a'),x),
    $$
    $$
        h_1(w(a'',x)) = h_1(r(a_2,x)) = r(a_2,x) = r(f_1(a''),x),
    $$
    $$
        h_2(w(a'',x)) = h_2(r(a_2,x)) = r(a_1,x) = r(f_2(a''),x)
    $$
    для любых $x \in X$.
    Так как $(f \circ f_1)(\star) = \star = (f \circ f_2)(\star)$ и $f(f_1(a')) = f(f_2(a'')) = f(a_1) = f(a_2) = f(f_1(a'')) = f(f_2(a'))$,
    то $f \circ f_1 = f \circ f_2$. Из того, что $g$ --- эпиморфизм, следует, что $x = g(y)$ для некоторого $y \in Y$ для любого $x \in X$. Тогда из равенств
    $$
        (h \circ h_1)(w(\star,x)) = \star = (h \circ h_2)(w(\star,x)),
    $$
    \begin{multline*}
        (h \circ h_1)(w(a',x)) = h(h_1(r(a_1,x))) = h(r(a_1,g(y))) = t(f(f_1(a')),x)  =\\= 
        t(f(a_1),y) = t(f(a_2),y) = h(r(a_2,g(y))) = h(h_2(r(a_1,x))) = (h \circ h_2)(w(a',x)),
    \end{multline*}
    \begin{multline*}
        (h \circ h_1)(w(a'',x)) = h(h_1(r(a_2,x))) = h(r(a_2,g(y))) = t(f(f_1(a'')),x)  =\\= 
        t(f(a_2),y) = t(f(a_1),y) = h(r(a_1,g(y))) = h(h_2(r(a_2,x))) = (h \circ h_2)(w(a'',x))
    \end{multline*}
    для любых $x \in X$ следует, что для любого $d' \in D'$ если $d' = w(a,x)$, то $(h \circ h_1)(d') = (h \circ h_2)(d') = t((f \circ f_1)(a),x)$. Следовательно, $(f,g,h) \circ (f_1,1_X,h_1) = (f,g,h) \circ (f_2,1_X,h_2)$. Поскольку $(f,g,h)$ --- мономорфизм категории $Chu_{\widetilde{Set}}$, то $f_1 = f_2$, то есть $a_1 = a_2$. Противоречие.

    \textbf{Достаточность.} Пусть $(f,g,h): r \to t$ --- преобразование Чу категории $Chu_{\widetilde{Set}}$, где $f$ --- мономорфизм, $g$ --- эпиморфизм категории $Set_{\star}$. Предположим, что $(f_1,g_1,h_1), (f_2,g_2,h_2): w \to r$ --- преобразования Чу категории $Chu_{\widetilde{Set}}$ такие, что $(f,g,h) \circ (f_1,g_1,h_1) = (f,g,h) \circ (f_2,g_2,h_2)$, где $w: E \times Z \to P$ --- пространство Чу категории $Chu_{\widetilde{Set}}$. Тогда $f \circ f_1 = f \circ f_2$, $g_1 \circ g = g_1 \circ g$ и для любого $p \in P$ если $p = w(e,(g_1 \circ g)(y)) = w(e,(g_2 \circ g)(y))$, то $(h \circ h_1)(p) = (h \circ h_2)(p) = t((f \circ f_1)(e), y) = t((f \circ f_2)(e), y)$. Покажем, что для любого $p \in P$ если $p = w(e,g_1(x))$, то $h_1(p) = h_2(p) = r(f_1(e),x)$. Пусть $p \in P$ и $p = w(e,g_1(x))$ для некоторых $e \in E$, $x \in X$. Поскольку $f$ --- мономорфизм, $g$ --- эпиморфизм категории $Set_{\star}$, то $f_1 = f_2$, $g_1 = g_2$. Тогда 
    $$
        h_1(p) = h_1(w(e,g_1(x))) = r(f_1(e),x) = r(f_2(e),x) = h_2(w(e,g_2(x))) = h_2(p).
    $$
    Следовательно, $(f_1,g_1,h_1) = (f_2,g_2,h_2)$ в категории $Chu_{\widetilde{Set}}$ и преобразование $(f,g,h): r \to t$ является мономорфизмом категории $Chu_{\widetilde{Set}}$.
\end{proof}

\begin{theorem}\label{coproduct}
    Пусть $r_i: A_i \times X_i \to D_i$ --- объекты категории $Chu_{\widetilde{Set}}$, $i \in I$. Объект $r: \coprod_{i \in I} A_i \times \prod_{i \in I} X_i \to \coprod_{i \in I} D_i$ категории $Chu_{\widetilde{Set}}$ вместе с семейством $\{(q_{A_i},p_{X_i},q_{D_i}): r \to r_i \mid i \in I\}$ преобразований Чу категории $Chu_{\widetilde{Set}}$, где $r(a,x) = r_i(a,p_{X_i}(x)) = r_i(a, x_i)$, $q_{A_i}: A_i \to \coprod_{i \in I} A_i$, $q_{D_i}: D_i \to \coprod_{i \in I} D_i$ --- естественные вложения, $p_{X_i}: \prod_{i \in I} X_i \to X_i$ --- каноническая проекция для любых $i \in I$, $a \in A_i$, $x \in \prod_{i \in I} X_i$, является копроизведением объектов $r_i$, $i \in I$, категории $Chu_{\widetilde{Set}}$.
\end{theorem}
\begin{proof}
    Пусть условия теоремы выполнены. Поскольку $q_{D_i}(r_i(a,p_{X_i}(x))) = r_i(a,x_i) = r(a,x) = r(q_{A_i}(a),x)$ для любых $a \in A_i$, $x \in \prod_{i \in I} X_i$, $i \in I$, то $(q_{A_i}, p_{X_i}, q_{D_i})$ --- преобразование Чу категории $Chu_{\widetilde{Set}}$ для любого $i \in I$.

    Пусть $t: B \times Y \to D$ --- объект категории $Chu_{\widetilde{Set}}$, $(f_i,g_i,h_i): r_i \to t$ --- преобразование Чу категории $Chu_{\widetilde{Set}}$ для любого $i \in I$. Так как $\coprod_{i \in I} A_i$ и $\coprod_{i \in I} D_i$ --- копроизведения $A_i$, $ i \in I$, и $D_i$, $i \in I$, соответственно, а $\prod_{i \in I} X_i$ --- произведение $A_i$, $i \in I$ в категории $Set_{\star}$, то существуют единственные морфизмы $\tilde{f}: \coprod_{i \in I} A_i \to B$, $\tilde{g}: Y \to \prod_{i \in I} X_i$, $\tilde{h}: \coprod_{i \in I} D_i \to D$ категории $Set_{\star}$ такие, что $f_i = \tilde{f} \circ q_{A_i}$, $g_i = p_{X_i} \circ \tilde{g}$ и $h_i = \tilde{h} \circ q_{D_i}$ для любого $i \in I$, то есть $\tilde{f}(a) = f_i(a)$, $\tilde{g}(y)_i = g_i(y)$ и $\tilde{h}(d) = h_i(d)$ для любых $a \in A_i$, $y \in Y$, $d \in D_i$. Следовательно, $(\tilde{f},\tilde{g},\tilde{h}) \circ (q_{A_i}, p_{X_i}, q_{D_i}) = (f_i, g_i, h_i)$ для любого $i \in I$ в категории $Chu_{\widetilde{Set}}$.

    Покажем, что $(\tilde{f},\tilde{g},\tilde{h})$ --- преобразование Чу категории $Chu_{\widetilde{Set}}$, то есть для любых $a \in \coprod_{i \in I} A_i$, $y \in Y$ имеет место равенство $\tilde{h}(r(a,\tilde{g}(y))) = t(\tilde{f}(a),y)$. Так как $(q_{A_i},p_{X_i},q_{D_i})$ --- преобразование Чу категории $Chu_{\widetilde{Set}}$, то $r(q_{A_i}(a),\tilde{g}(y)) = q_{D_i}(r_i(a,(p_{X_i} \circ \tilde{g})(y)))$ для любых $a \in A_i$, $y \in Y$. Из того, что $h_i = \tilde{h} \circ q_{D_i}$ и $g_i = p_{X_i} \circ \tilde{g}$ следует, что $\tilde{h}(r(a,\tilde{g}(y))) = (\tilde{h} \circ q_{D_i})(r_i(a,(p_{X_i} \circ \tilde{g})(y))) = h_i(r_i(a,g_i(y)))$ для любых $a \in A_i$, $y \in Y$. Так как $(f_i,g_i,h_i)$ --- преобразование Чу категории $Chu_{\widetilde{Set}}$, то $h_i(r_i(a,g_i(y))) = t(f_i(a),y)$ для любых $a \in A_i$, $y \in Y$. Поскольку $\tilde{f}(a) = f_i(a)$, то $t(f_i(a),y) = t(\tilde{f}(a),y)$ для любых $a \in A_i$, $y \in Y$. Таким образом, для любых $a \in \coprod_{i \in I} A_i$, $y \in Y$ имеет место равенство $\tilde{h}(r(a,\tilde{g}(y))) = t(\tilde{f}(a),y)$.
\end{proof}

\begin{theorem}\label{iso}
    Обекты $r: A \times X \to D$ и $\tilde{r}: A \times X \to r(A \times X)$ категории $Chu_{\widetilde{Set}}$, где $\tilde{r}(a,x) = r(a,x)$ для любых $a \in A$, $x \in X$, изоморфны.
\end{theorem}
\begin{proof}
    Построим морфизм $(f,g,h): r \to \tilde{r}$ следующим образом: $f = id_A$, $g = id_X$, $h(d) = 
    \begin{cases}
        r(a,x),& \text{если } d \in r(A \times X),\\
        \star,& \text{иначе}.
    \end{cases}$
    Из равенства $h(r(a,x)) = r(a,x)$ для любых $a \in A$, $x \in X$ следует, что $(f,g,h)$ --- преобразование Чу категории $Chu_{\widetilde{Set}}$. Построим морфизм $(\tilde{f},\tilde{g},\tilde{h}): \tilde{r} \to r$ следующим образом: $\tilde{f} = id_A$, $\tilde{g} = id_X$, $\tilde{h}$ --- естественное вложение. Из равенства $\tilde{h}(r(a,x)) = r(a,x)$ для любых $a \in A$, $x \in X$ следует, что $(\tilde{f},\tilde{g},\tilde{h})$ --- преобразование Чу категории $Chu_{\widetilde{Set}}$.

    Поскольку $(\tilde{f},\tilde{g},\tilde{h}) \circ (f,g,h) = (1_A,1_X,\tilde{h} \circ h)$ и для любого $d \in D$ если $d = r(a,x)$, то $\tilde{h}(d) = h(d) = r(a,x)$, то $(\tilde{f},\tilde{g},\tilde{h}) \circ (f,g,h) = 1_r$. Поскольку $(f,g,h) \circ (\tilde{f},\tilde{g},\tilde{h}) = (1_A,1_X,h \circ \tilde{h})$ и для любого $d \in r(A \times X)$ если $d = r(a,x)$, то $h(d) = \tilde{h}(d) = r(a,x)$, то $(f,g,h) \circ (\tilde{f},\tilde{g},\tilde{h}) = 1_{\tilde{r}}$.
\end{proof}

\begin{theorem}\label{product}
    Пусть $r_i: A_i \times X_i \to D_i$ --- объекты категории $Chu_{\widetilde{Set}}$, $i \in I$. Объект $r: \prod_{i \in I} A_i \times \coprod_{i \in I} X_i \to \prod_{i \in I} r_i(A_i \times X_i)$ категории $Chu_{\widetilde{Set}}$ вместе с семейством $\{(p_{A_i},q_{X_i},p_i): r \to r_i \mid i \in I\}$ преобразований Чу категории $Chu_{\widetilde{Set}}$, где $\tilde{p}_j(r(a,x_i)) = 
    \begin{cases}
        r_i(p_{A_i}(a),x_i),& \text{если } i = j,\\
        \star,& \text{если } i \ne j,
    \end{cases}$
    $\tilde{p}_i: \prod_{i \in I} r_i(A_i \times X_i) \to r_i(A_i \times X_i)$ --- каноническая проекция, $p_{A_i}: \prod_{i \in I} A_i \to A_i$ --- каноническая проекция, $q_{X_i}: X_i \to \coprod_{i \in I} X_i$ --- естественное вложение, $p_i = m_i \circ \tilde{p}_i$, $m_i: r_i(A_i \times X_i) \to D_i$ --- естественное вложение, для любых $a \in \prod_{i \in I} A_i$, $x_i \in X_i$, $i,j \in I$, является произведением объектов $r_i$, $i \in I$, категории $Chu_{\widetilde{Set}}$.
\end{theorem}
\begin{proof}
    Пусть условия теоремы выполнены. Поскольку $p_i(r(a,q_{X_i}(x_i))) = m_i(\tilde{p}_i(r(a,q_{X_i}(x_i)))) = \tilde{p}_i(r(a,q_{X_i}(x_i))) = r_i(p_{A_i}(a),x_i)$ для любых $a \in \prod_{i \in I} A_i$, $x_i \in X_i$, $i \in I$, то $(p_{A_i},q_{X_i},p_i)$ --- преобразование Чу категории $Chu_{\widetilde{Set}}$ для любого $i \in I$.

    Пусть $t: B \times Y \to D$ --- объект категории $Chu_{\widetilde{Set}}$, $(f_i,g_i,h_i): t \to r_i$ --- преобразование Чу категории $Chu_{\widetilde{Set}}$ для любого $i \in I$. Для любого $h_i: D \to D_i$, $i \in I$, построим морфизм $\tilde{h}_i: D \to r_i(A_i \times X_i)$ категории $Set_\star$ следующим образом: $\tilde{h}_i(d) = 
    \begin{cases}
        h_i(d),& \text{если } d = t(b,g_i(x_i)) \text{ для некоторых } b \in B, x_i \in X_i,\\
        \star,& \text{иначе},
    \end{cases}$
    для любого $d \in D$. Поскольку $(f_i,g_i,h_i)$ --- преобразование Чу категории $Chu_{\widetilde{Set}}$ для любого $i \in I$, то $m_i(\tilde{h}_i(t(b,g_i(x_i)))) = h_i(t(b,g_i(x_i))) = r_i(f_i(b),x_i)$ для любых $b \in B$, $x_i \in X_i$. Следовательно, $(f_i,g_i,m_i \circ \tilde{h}_i): t \to r_i$ --- преобразование Чу категории $Chu_{\widetilde{Set}}$ для любого $i \in I$.

    Так как $\prod_{i \in I} A_i$ и $\prod_{i \in I} r_i(A_i \times X_i)$ --- произведения $A_i$, $ i \in I$, и $r_i(A_i \times X_i)$, $i \in I$, соответственно, а $\coprod_{i \in I} X_i$ --- копроизведение $X_i$, $i \in I$, в категории $Set_{\star}$, то существуют единственные морфизмы $\tilde{f}: B \to \prod_{i \in I} A_i$, $\tilde{g}: \coprod_{i \in I} X_i \to Y$, $\tilde{h}: D \to \prod_{i \in I} r_i(A_i \times X_i)$ категории $Set_{\star}$ такие, что $f_i = p_{A_i} \circ \tilde{f}$, $g_i = \tilde{g} \circ q_{X_i}$ и $\tilde{h}_i = \tilde{p}_i \circ \tilde{h}$ для любого $i \in I$, то есть $p_{A_i}(\tilde{f}(b)) = f_i(b)$, $\tilde{g}(x_i) = g_i(x_i)$ и $\tilde{p}_i(\tilde{h}(d)) = \tilde{h}_i(d)$ для любых $b \in  B$, $x_i \in X_i$, $d \in D$, $i \in I$.

    Покажем, что $(\tilde{f},\tilde{g},\tilde{h})$ --- преобразование Чу категории $Chu_{\widetilde{Set}}$, то есть для любых $b \in B$, $x_i \in X_i$, $i \in I$, имеет место равенство $\tilde{h}(t(b,\tilde{g}(x_i))) = r(\tilde{f}(b),x_i)$. Поскольку $\prod_{i \in I} r_i(A_i \times X_i)$ вместе с семейством $\{\tilde{p}_i: \prod_{i \in I} r_i(A_i \times X_i) \to r_i(A_i \times X_i) \mid i \in I\}$ морфизмов категории $Set_\star$ является произведением объектов $r_i(A_i \times X_i)$, $i \in I$, то для любых $b \in B$, $x \in X_i$ равенство $\tilde{h}(t(b,\tilde{g}(x_i))) = r(\tilde{f}(b),x_i)$ равносильно равенству $\tilde{p}_j(\tilde{h}(t(b,\tilde{g}(x_i)))) = \tilde{p}_j(r(\tilde{f}(b),x_i))$ для любых $b \in B$, $x_i \in I$, $i,j \in I$. Тогда из равенств
    \begin{multline*}
        \tilde{p}_j(\tilde{h}(t(b,\tilde{g}(x_i)))) = \tilde{h}_j(t(b,g_i(x_i))) =\\= 
        \begin{cases}
            h_i(t(b,g_i(x_i))) = r_i(f_i(b),x_i) = r_j(f_j(b),x_i),& \text{если } i = j\\
            \star,& \text{если } i \ne j
        \end{cases} =\\=
        \tilde{p}_j(r(\tilde{f}(b),x_i))
    \end{multline*}
    для любых $b \in B$, $x_i \in X_i$, $i,j \in I$ следует, что $(\tilde{f},\tilde{g},\tilde{h})$ --- преобразование Чу категории $Chu_{\widetilde{Set}}$.

    Покажем, что $(p_{A_i},q_{X_i},p_i) \circ (\tilde{f},\tilde{g},\tilde{h}) = (f_i,g_i,h_i)$ для любого $i \in I$, то есть $p_{A_i} \circ \tilde{f} = f_i$, $\tilde{g} \circ q_{X_i} = g_i$ и для любого $d \in D$ если $d = t(b,g_i(x_i))$, то $(p_i \circ \tilde{h})(d) = h_i(d) = r_i(f_i(b),x_i)$. Из определения $\tilde{f}$ и $\tilde{g}$ следует, что $p_{A_i} \circ \tilde{f} = f_i$ и $\tilde{g} \circ q_{X_i} = g_i$ для любого $i \in I$. Пусть $d \in D$ и $d = t(b,g_i(x_i))$ для некоторых $b \in B$, $x_i \in X_i$, $i \in I$. Тогда
    \begin{multline*}
        (p_i \circ \tilde{h})(d) = (m_i \circ \tilde{p}_i \circ \tilde{h})(d) = (m_i \circ \tilde{h}_i)(d) = m_i(\tilde{h}_i(d)) = \tilde{h}_i(d) =\\=
        \tilde{h}_i(t(b,g_i(x_i))) = h_i(t(b,g_i(x_i))) = h_i(d) = r_i(f_i(b),x_i).
    \end{multline*}
    Следовательно, $(p_{A_i},q_{X_i},p_i) \circ (\tilde{f},
    \tilde{g},\tilde{h}) = (f_i,g_i,h_i)$ для любого $i \in I$.

    Покажем, что $(\tilde{f},\tilde{g},\tilde{h})$ --- единственный морфизм такой, что $(p_{A_i},q_{X_i},p_i) \circ (\tilde{f},\tilde{g},\tilde{h}) = (f_i,g_i,h_i)$ для любого $i \in I$. Пусть $(f',g',h'): t \to r$ --- преобразование Чу категории $Chu_{\widetilde{Set}}$ такое, что $(p_{A_i},q_{X_i},p_i) \circ (f',g',h') = (f_i,g_i,h_i)$ для любого $i \in I$, то есть $p_{A_i} \circ f' = f_i$, $g' \circ q_{X_i} = g_i$ и для любого $d \in D$ если $d = t(b,g_i(x_i))$, то $(p_i \circ h')(d) = h_i(d) = r_i(f_i(b),x_i)$ для любого $i \in I$. Из определения $\tilde{f}$ и $\tilde{g}$ следует, что $f' = \tilde{f}$ и $g' = \tilde{g}$. Осталось показать, что $(\tilde{f},\tilde{g},h') = (\tilde{f},\tilde{g},\tilde{h})$, то есть для любого $d \in D$ если $d = t(b,g_i(x_i))$, то $h'(d) = \tilde{h}(d) = r(\tilde{f},x_i)$. Поскольку $\prod_{i \in I} r_i(A_i \times X_i)$ вместе с семейством $\{\tilde{p}_i: \prod_{i \in I} r_i(A_i \times X_i) \to r_i(A_i \times X_i) \mid i \in I\}$ морфизмов категории $Set_\star$ является произведением объектов $r_i(A_i \times X_i)$, $i \in I$, то для любых $b \in B$, $x \in X_i$ равенства $h'(d) = \tilde{h}(d) = r(\tilde{f}(b),x_i)$ равносильны равенствам $\tilde{p}_j(h'(d)) = \tilde{p}_j(\tilde{h}(d)) = \tilde{p}_j(r(\tilde{f}(b),x_i))$ для любых $b \in B$, $x_i \in I$, $i,j \in I$.  Пусть $d \in D$ и $d = t(b,g_i(x_i))$. Тогда если $i = j$, то
    $$
        \tilde{p}_j(h'(d)) = m_i(\tilde{p}_i(h'(d))) = p_i(h'(d)) = p_i(\tilde{h}(d)) = h_i(d) = r_i(f_i(b),x_i),
    $$
    иначе, если $i \ne j$, то
    $$
        \tilde{p}_j(h'(d)) = \tilde{p}_j(h'(t(b,g'(x_i)))) = \tilde{p}_j(r(f'(b),x_i)) = \star = \tilde{p}_j(r(\tilde{f}(b),x_i)) = \tilde{p}_j(\tilde{h}(t(b,\tilde{g}(x_i)))).
    $$
    Следовательно, $(\tilde{f},\tilde{g},\tilde{h}) = (f',g',h')$ в категории $Chu_{\widetilde{Set}}$.
\end{proof}



\section*{Категория $Chu(\GAct_{\star})$}

\begin{theorem}\label{epimorphism-gact}
    Пусть $r: A \otimes X \to D$ и $t: B \otimes Y \to C$ --- объекты категории $Chu_{\widetilde{\GAct}}$. Преобразование $(f,g,h): r \to t$ категории $Chu_{\widetilde{\GAct}}$ является эпиморфизмом тогда и только тогда, когда $f: A \to B$ --- эпиморфизм, $g: Y \to X$ --- мономорфизм категории $\GAct_{\star}$.
\end{theorem}
\begin{proof}
    \textbf{Необходимость.} Пусть $(f,g,h): r \to t$ --- эпиморфизм категории $Chu_{\widetilde{\GAct}}$.

    Покажем, что $f$ --- эпиморфизм категории $\GAct_{\star}$. Предположим, что $B_1 \ne B$, где $B_1 = f(A)$. Через $B_0$ обозначим фактор-полигон полигона $B$ по конгруэнции Риса $\rho_{D_1}$. Определим объект $w: (B_0 \times B) \otimes Y \to C$ и морфизмы $(f_1,1_Y,1_C), (f_2,1_Y,1_C): t \to w$ категории $Chu_{\widetilde{\GAct}}$ следующим образом: $w((b_0,b) \otimes y) = t(b \otimes y)$, $f_1(b) = (B_1,b)$, $f_2(b) = (b/\rho_{B_1},b)$ для любых $b \in B$, $b_0 \in B_0$, $y \in Y$. Корректность определения морфизмов $(f_1,1_Y,1_C)$, $(f_2,1_Y,1_C)$ следует из равенств:
    $$
        t(b \otimes y) = w(f_1(b) \otimes y) = w(f_2(b) \otimes y),
    $$
    где $b \in B$, $y \in Y$. Если $a \in A$, то $f(a) \in B_1$ и $f_1(f(a)) = f_2(f(a))$, то есть $f_1 \circ f = f_2 \circ f$. Тогда $(f_1,1_Y,1_C) \circ (f,g,h) = (f_2,1_Y,1_C) \circ (f,g,h)$. Поскольку $(f,g,h)$ --- эпиморфизм категории $Chu_{\widetilde{\GAct}}$, то $f_1 = f_2$. Противоречие.

    Покажем, что $g$ --- мономорфизм категории $\GAct_{\star}$. Предположим, что существуют различные $y_1, y_2 \in Y$ такие, что $g(y_1) = g(y_2)$. Пусть $b \in B$. Покажем, что $t(b \otimes y_1) = t(b \otimes y_2)$. Так как $f$ --- эпиморфизм, то существует $a \in A$ такой, что $f(a) = b$. Тогда
    $$
        t(b \otimes y_1) = t(f(a) \otimes y_1) = h(r(a \otimes g(y_1))) = h(r(a \otimes g(y_2))) = t(f(a) \otimes y_2) = t(b \otimes y_2).
    $$
    Определим объект $w: B \otimes (G \sqcup \{\star\}) \to C$ и морфизмы $(1_B,g_1,1_C), (1_B,g_2,1_C): t \to w$ категории $Chu_{\widetilde{\GAct}}$ следующим образом: $w(b \otimes \star) = \star$, $w(b \otimes s) = st(b \otimes y_1)$, $g_1(\star) = g_2(\star) = \star$, $g_1(s) = sy_1$, $g_2(s) = sy_2$ для любых $b \in B$, $s \in G$. Корректность определения морфизмов $(1_B,g_1,1_C), (1_B,g_2,1_C)$ следует из равенств:
    $$
        w(b \otimes \star) = t(b \otimes g_1(\star)) = t(b \otimes g_2(\star)) = t(b \otimes \star) = \star,
    $$
    $$
        w(b \otimes s) = t(b \otimes g_1(s)) = st(b \otimes y_1) = st(b \otimes y_2) = t(b \otimes g_2(s))
    $$
    для любого $b \in B$, $s \in G$. Так как
    $$
        (g \circ g_1)(\star) = g(g_1(\star)) = \star = g(g_2(\star)) = (g \circ g_2)(\star),
    $$
    $$
        (g \circ g_1)(s) = g(g_1(s)) = sg(y_1) = sg(y_2) = g(g_2(s)) = (g \circ g_2)(s),
    $$
    то $(1_B,g_1,1_C) \circ (f,g,h) = (1_B,g_2,1_C) \circ (f,g,h)$. Поскольку $(f,g,h)$ --- эпиморфизм категории $Chu_{\widetilde{\GAct}}$, то $g_1 = g_2$. Противоречие.

    \textbf{Достаточность.} Пусть $(f,g,h): r \to t$ --- преобразование Чу категории $Chu_{\widetilde{\GAct}}$, где $f$ --- эпиморфизм, $g$ --- мономорфизм категории $\GAct_{\star}$. Предположим, что $(f_1,g_1,h_1), (f_2,g_2,h_2): t \to w$ --- преобразования Чу категории $Chu_{\widetilde{\GAct}}$ такие, что $(f_1,g_1,h_1) \circ (f,g,h) = (f_2,g_2,h_2) \circ (f,g,h)$, где $w: E \otimes Z \to P$ --- пространство Чу категории $Chu_{\widetilde{\GAct}}$. Тогда $f_1 \circ f = f_2 \circ f$, $g \circ g_1 = g \circ g_2$ и для любого $d \in D$ если $d = r(a \otimes (g \circ g_1)(z)) = r(a \otimes (g \circ g_2)(z))$, то $(h_1 \circ h)(d) = (h_2 \circ h)(d) = w((f_1 \circ f)(a) \otimes z) = w((f_2 \circ f)(a) \otimes z)$. Покажем, что для любого $c \in C$ если $c = t(b \otimes g_1(z))$, то $h_1(c) = h_2(c) = w(f_1(b) \otimes z)$. Пусть $c \in C$ и $c = t(b \otimes g_1(z))$ для некоторых $b \in B$, $z \in Z$. Поскольку $f$ --- эпиморфизм, $g$ --- мономорфизм категории $\GAct_{\star}$, то $f_1 = f_2$, $g_1 = g_2$. Так как $f$ --- эпиморфизм, то $b = f(a)$ для некоторого $a \in A$. Тогда 
    \begin{multline*}
        h_1(c) = h_1(t(b \otimes g_1(z))) = h_1(t(f(a) \otimes g_1(z))) = w(f_1(f(a)) \otimes z) =\\=
        w(f_2(f(a)) \otimes z) = h_2(t(f(a) \otimes g_2(z))) = h_1(t(b \otimes g_1(z))) = h_2(c).
    \end{multline*}
    Следовательно, $(f_1,g_1,h_1) = (f_2,g_2,h_2)$ в категории $Chu_{\widetilde{\GAct}}$ и преобразование $(f,g,h): r \to t$ является эпиморфизмом категории $Chu_{\widetilde{\GAct}}$.
\end{proof}

\begin{theorem}\label{monomorphism-gact}
    Пусть $r: A \otimes X \to D$ и $t: B \otimes Y \to C$ --- объекты категории $Chu_{\widetilde{\GAct}}$. Преобразование $(f,g,h): r \to t$ категории $Chu_{\widetilde{\GAct}}$ является мономорфизмом тогда и только тогда, когда $f: A \to B$ --- мономорфизм, $g: Y \to X$ --- эпиморфизм категории $\GAct^{\star}$.
\end{theorem}
\begin{proof}
    \textbf{Необходимость.} Пусть $(f,g,h): r \to t$ --- мономорфизм категории $Chu_{\widetilde{\GAct}}$.

    Покажем, что $g$ --- эпиморфизм категории $\GAct_{\star}$. Предположим, что $X_1 \ne X$, где $X_1 = g(Y)$. Через $X_0$ обозначим фактор-полигон полигона $X$ по конгруэнции Риса $\rho_{X_1}$. Определим объект $w: A \otimes (X_0 \times X) \to D$ и морфизмы $(1_A,g_1,1_D), (1_A,g_2,1_D): w \to r$ категории $Chu_{\widetilde{\GAct}}$ следующим образом: $w(a \otimes (x_0,x)) = r(a \otimes x)$, $g_1(x) = (X_1,x)$, $g_2(x) = (x/\rho_{X_1}, x)$, для любых $a \in A$, $x \in X$, $x_0 \in X_0$. Из определения объекта $w$ категории $Chu_{\widetilde{\GAct}}$ следует равенство
    $$
        w(a \otimes g_1(x)) = w(a \otimes g_2(x)) = r(a \otimes x),
    $$ 
    где $a \in A$, $x \in X$, что доказывает корректность определения морфизмов $(1_A,g_1,1_D)$, $(1_A,g_2,1_D)$. Если $y \in Y$, то $g(y) \in X_1$ и $g_1(g(y)) = g_2(g(y))$, то есть $g_1 \circ g = g_2 \circ g$. Тогда $(f,g,h) \circ (1_A,g_1,1_D) = (f,g,h) \circ (1_A,g_2,1_D)$. Поскольку $(f,g,h)$ --- мономорфизм категории $Chu_{\widetilde{\GAct}}$, то $g_1 = g_2$. Противоречие.

    Покажем, что $f$ --- мономорфизм категории $\GAct_{\star}$. Предположим, что существуют различные $a_1, a_2 \in A$ такие, что $f(a_1) = f(a_2)$. Определим объект $w: (\{\star\} \sqcup G \sqcup G') \otimes X \to D'$, где $G'$ --- копия $G$, $D' = r(\{a_1,a_2\} \otimes X)$ и морфизмы $(f_1,1_X,1_D), (f_2,1_X,1_D): w \to r$ категории $Chu_{\widetilde{\GAct}}$ следующим образом $w(\star \otimes x) = \star$, $w(s \otimes x) = sr(a_1 \otimes x)$, $w(s' \otimes x) = s'r(a_2 \otimes x)$, $f_1(\star) = f_2(\star) = \star$, $f_1(s) = sa_1$, $f_1(s') = s'a_2$, $f_2(s) = sa_2$, $f_2(s') = s'a_1$, $h_1(r(a_1 \otimes x)) = h_2(r(a_2 \otimes x)) = r(a_1 \otimes x)$, $h_1(r(a_2 \otimes x)) = h_2(r(a_1 \otimes x)) = r(a_2 \otimes x)$ для любых $x \in X$, $s \in G$, $s' \in G'$. Корректность определения морфизмов $(f_1,1_X,h_1), (f_2,1_X,h_2)$ следует из равенств
    $$
        h_1(w(\star \otimes x)) = h_2(w(\star \otimes x)) = \star = r(f_1(\star) \otimes x) = r(f_2(\star) \otimes x),
    $$
    $$
        h_1(w(s \otimes x)) = h_1(sr(a_1 \otimes x)) = r(sa_1 \otimes x) = r(f_1(s) \otimes x),
    $$
    $$
        h_2(w(s \otimes x)) = h_2(sr(a_1 \otimes x)) = r(sa_2 \otimes x) = r(f_2(s) \otimes x),
    $$
    $$
        h_1(w(s' \otimes x)) = h_1(s'r(a_2 \otimes x)) = r(s'a_2 \otimes x) = r(f_1(s') \otimes x),
    $$
    $$
        h_2(w(s' \otimes x)) = h_2(s'r(a_2 \otimes x)) = r(s'a_1 \otimes x) = r(f_2(s') \otimes x)
    $$
    для любых $x \in X$, $s \in G$, $s' \in G'$.
    Так как $f(f_1(s)) = f(sa_1) = sf(a_1) = sf(a_2) = f(sa_2) = f(f_2(s))$, $f(f_1(s')) = f(s'a_2) = s'f(a_2) = s'f(a_1) = f(s'a_1) = f(f_2(s'))$ и $(f \circ f_1)(\star) = \star = (f \circ f_2)(\star)$ для любых $s \in G$, $s' \in G'$, то $f \circ f_1 = f \circ f_2$. Из того, что $g$ --- эпиморфизм, следует, что $x = g(y)$ для некоторого $y \in Y$ для любого $x \in X$. Тогда из равенств
    $$
        (h \circ h_1)(w(\star \otimes x)) = \star = (h \circ h_2)(w(\star \otimes x)),
    $$
    \begin{multline*}
        (h \circ h_1)(w(s \otimes x)) = h(sh_1(r(a_1 \otimes x))) = sh(r(a_1 \otimes g(y))) = st(f(f_1(s)) \otimes x)  =\\= 
        st(f(a_1) \otimes y) = st(f(a_2) \otimes y) = sh(r(a_2 \otimes g(y))) = h(sh_2(r(a_1 \otimes x))) = (h \circ h_2)(w(s \otimes x)),
    \end{multline*}
    \begin{multline*}
        (h \circ h_1)(w(s' \otimes x)) = h(s'h_1(r(a_2 \otimes x))) = s'h(r(a_2 \otimes g(y))) = s't(f(f_1(s')) \otimes x)  =\\= 
        s't(f(a_2) \otimes y) = s't(f(a_1) \otimes y) = s'h(r(a_1 \otimes g(y))) = h(s'h_2(r(a_2 \otimes x))) = (h \circ h_2)(w(s' \otimes x))
    \end{multline*}
    для любых $x \in X$, $s \in G$, $s' \in G'$ следует, что для любого $d' \in D'$ если $d' = w(a \otimes x)$, то $(h \circ h_1)(d') = (h \circ h_2)(d') = t((f \circ f_1)(a) \otimes x)$. Следовательно, $(f,g,h) \circ (f_1,1_X,h_1) = (f,g,h) \circ (f_2,1_X,h_2)$. Поскольку $(f,g,h)$ --- мономорфизм категории $Chu_{\widetilde{\GAct}}$, то $f_1 = f_2$, то есть $a_1 = a_2$. Противоречие.

    \textbf{Достаточность.} Пусть $(f,g,h): r \to t$ --- преобразование Чу категории $Chu_{\widetilde{\GAct}}$, где $f$ --- мономорфизм, $g$ --- эпиморфизм категории $\GAct_{\star}$. Предположим, что $(f_1,g_1,h_1), (f_2,g_2,h_2): w \to r$ --- преобразования Чу категории $Chu_{\widetilde{\GAct}}$ такие, что $(f,g,h) \circ (f_1,g_1,h_1) = (f,g,h) \circ (f_2,g_2,h_2)$, где $w: E \otimes Z \to P$ --- пространство Чу категории $Chu_{\widetilde{\GAct}}$. Тогда $f \circ f_1 = f \circ f_2$, $g_1 \circ g = g_1 \circ g$ и для любого $p \in P$ если $p = w(e \otimes (g_1 \circ g)(y)) = w(e \otimes (g_2 \circ g)(y))$, то $(h \circ h_1)(p) = (h \circ h_2)(p) = t((f \circ f_1)(e) \otimes y) = t((f \circ f_2)(e) \otimes y)$. Покажем, что для любого $p \in P$ если $p = w(e \otimes g_1(x))$, то $h_1(p) = h_2(p) = r(f_1(e) \otimes x)$. Пусть $p \in P$ и $p = w(e \otimes g_1(x))$ для некоторых $e \in E$, $x \in X$. Поскольку $f$ --- мономорфизм, $g$ --- эпиморфизм категории $\GAct_{\star}$, то $f_1 = f_2$, $g_1 = g_2$. Тогда 
    $$
        h_1(p) = h_1(w(e \otimes g_1(x))) = r(f_1(e) \otimes x) = r(f_2(e) \otimes x) = h_2(w(e \otimes g_2(x))) = h_2(p).
    $$
    Следовательно, $(f_1,g_1,h_1) = (f_2,g_2,h_2)$ в категории $Chu_{\widetilde{\GAct}}$ и преобразование $(f,g,h): r \to t$ является мономорфизмом категории $Chu_{\widetilde{\GAct}}$.
\end{proof}

\begin{theorem}\label{coproduct-gact}
    Пусть $r_i: A_i \otimes X_i \to D_i$ --- объекты категории $Chu_{\widetilde{\GAct}}$, $i \in I$. Объект $r: \coprod_{i \in I} A_i \otimes \prod_{i \in I} X_i \to \coprod_{i \in I} D_i$ категории $Chu_{\widetilde{\GAct}}$ вместе с семейством $\{(q_{A_i},p_{X_i},q_{D_i}): r \to r_i \mid i \in I\}$ преобразований Чу категории $Chu_{\widetilde{\GAct}}$, где $r(a,x) = r_i(a,p_{X_i}(x)) = r_i(a, x_i)$, $q_{A_i}: A_i \to \coprod_{i \in I} A_i$, $q_{D_i}: D_i \to \coprod_{i \in I} D_i$ --- естественные вложения, $p_{X_i}: \prod_{i \in I} X_i \to X_i$ --- каноническая проекция для любых $i \in I$, $a \in A_i$, $x \in \prod_{i \in I} X_i$, является копроизведением объектов $r_i$, $i \in I$, категории $Chu_{\widetilde{\GAct}}$.
\end{theorem}
\begin{proof}
    Пусть условия теоремы выполнены. Поскольку $q_{D_i}(r_i(a \otimes p_{X_i}(x))) = r_i(a \otimes x_i) = r(a \otimes x) = r(q_{A_i}(a) \otimes x)$ для любых $a \in A_i$, $x \in \prod_{i \in I} X_i$, $i \in I$, то $(q_{A_i}, p_{X_i}, q_{D_i})$ --- преобразование Чу категории $Chu_{\widetilde{\GAct}}$ для любого $i \in I$.

    Пусть $t: B \otimes Y \to D$ --- объект категории $Chu_{\widetilde{\GAct}}$, $(f_i,g_i,h_i): r_i \to t$ --- преобразование Чу категории $Chu_{\widetilde{\GAct}}$ для любого $i \in I$. Так как $\coprod_{i \in I} A_i$ и $\coprod_{i \in I} D_i$ --- копроизведения $A_i$, $ i \in I$, и $D_i$, $i \in I$, соответственно, а $\prod_{i \in I} X_i$ --- произведение $A_i$, $i \in I$ в категории $\GAct_{\star}$, то существуют единственные морфизмы $\tilde{f}: \coprod_{i \in I} A_i \to B$, $\tilde{g}: Y \to \prod_{i \in I} X_i$, $\tilde{h}: \coprod_{i \in I} D_i \to D$ категории $\GAct_{\star}$ такие, что $f_i = \tilde{f} \circ q_{A_i}$, $g_i = p_{X_i} \circ \tilde{g}$ и $h_i = \tilde{h} \circ q_{D_i}$ для любого $i \in I$, то есть $\tilde{f}(a) = f_i(a)$, $\tilde{g}(y)_i = g_i(y)$ и $\tilde{h}(d) = h_i(d)$ для любых $a \in A_i$, $y \in Y$, $d \in D_i$. Следовательно, $(\tilde{f},\tilde{g},\tilde{h}) \circ (q_{A_i}, p_{X_i}, q_{D_i}) = (f_i, g_i, h_i)$ для любого $i \in I$ в категории $Chu_{\widetilde{\GAct}}$.

    Покажем, что $(\tilde{f},\tilde{g},\tilde{h})$ --- преобразование Чу категории $Chu_{\widetilde{\GAct}}$, то есть для любых $a \in \coprod_{i \in I} A_i$, $y \in Y$ имеет место равенство $\tilde{h}(r(a \otimes \tilde{g}(y))) = t(\tilde{f}(a) \otimes y)$. Так как $(q_{A_i},p_{X_i},q_{D_i})$ --- преобразование Чу категории $Chu_{\widetilde{\GAct}}$, то $r(q_{A_i}(a) \otimes \tilde{g}(y)) = q_{D_i}(r_i(a \otimes (p_{X_i} \circ \tilde{g})(y)))$ для любых $a \in A_i$, $y \in Y$. Из того, что $h_i = \tilde{h} \circ q_{D_i}$ и $g_i = p_{X_i} \circ \tilde{g}$ следует, что $\tilde{h}(r(a \otimes \tilde{g}(y))) = (\tilde{h} \circ q_{D_i})(r_i(a \otimes (p_{X_i} \circ \tilde{g})(y))) = h_i(r_i(a \otimes g_i(y)))$ для любых $a \in A_i$, $y \in Y$. Так как $(f_i,g_i,h_i)$ --- преобразование Чу категории $Chu_{\widetilde{\GAct}}$, то $h_i(r_i(a \otimes g_i(y))) = t(f_i(a) \otimes y)$ для любых $a \in A_i$, $y \in Y$. Поскольку $\tilde{f}(a) = f_i(a)$, то $t(f_i(a) \otimes y) = t(\tilde{f}(a) \otimes y)$ для любых $a \in A_i$, $y \in Y$. Таким образом, для любых $a \in \coprod_{i \in I} A_i$, $y \in Y$ имеет место равенство $\tilde{h}(r(a \otimes \tilde{g}(y))) = t(\tilde{f}(a) \otimes y)$.
\end{proof}

\begin{theorem}\label{iso-gact}
    Обекты $r: A \otimes X \to D$ и $\tilde{r}: A \otimes X \to r(A \otimes X)$ категории $Chu_{\widetilde{Set}}$, где $\tilde{r}(a,x) = r(a,x)$ для любых $a \in A$, $x \in X$, изоморфны.
\end{theorem}
\begin{proof}
    Построим морфизм $(f,g,h): r \to \tilde{r}$ следующим образом: $f = id_A$, $g = id_X$, $h(d) = 
    \begin{cases}
        r(a \otimes x),& \text{если } d \in r(A \otimes X),\\
        \star,& \text{иначе}.
    \end{cases}$
    Из равенства $h(r(a \otimes x)) = r(a \otimes x)$ для любых $a \in A$, $x \in X$ следует, что $(f,g,h)$ --- преобразование Чу категории $Chu_{\widetilde{Set}}$. Построим морфизм $(\tilde{f},\tilde{g},\tilde{h}): \tilde{r} \to r$ следующим образом: $\tilde{f} = id_A$, $\tilde{g} = id_X$, $\tilde{h}$ --- естественное вложение. Из равенства $\tilde{h}(r(a \otimes x)) = r(a \otimes x)$ для любых $a \in A$, $x \in X$ следует, что $(\tilde{f},\tilde{g},\tilde{h})$ --- преобразование Чу категории $Chu_{\widetilde{Set}}$.

    Поскольку $(\tilde{f},\tilde{g},\tilde{h}) \circ (f,g,h) = (1_A,1_X,\tilde{h} \circ h)$ и для любого $d \in D$ если $d = r(a,x)$, то $\tilde{h}(d) = h(d) = r(a \otimes x)$, то $(\tilde{f},\tilde{g},\tilde{h}) \circ (f,g,h) = 1_r$. Поскольку $(f,g,h) \circ (\tilde{f},\tilde{g},\tilde{h}) = (1_A,1_X,h \circ \tilde{h})$ и для любого $d \in r(A \otimes X)$ если $d = r(a \otimes x)$, то $h(d) = \tilde{h}(d) = r(a \otimes x)$, то $(f,g,h) \circ (\tilde{f},\tilde{g},\tilde{h}) = 1_{\tilde{r}}$.
\end{proof}

\begin{theorem}\label{product-gact}
    Пусть $r_i: A_i \otimes X_i \to D_i$ --- объекты категории $Chu_{\widetilde{Set}}$, $i \in I$. Объект $r: \prod_{i \in I} A_i \otimes \coprod_{i \in I} X_i \to \prod_{i \in I} r_i(A_i \otimes X_i)$ категории $Chu_{\widetilde{Set}}$ вместе с семейством $\{(p_{A_i},q_{X_i},p_i): r \to r_i \mid i \in I\}$ преобразований Чу категории $Chu_{\widetilde{Set}}$, где $\tilde{p}_j(r(a \otimes x_i)) = 
    \begin{cases}
        r_i(p_{A_i}(a) \otimes x_i),& \text{если } i = j,\\
        \star,& \text{если } i \ne j,
    \end{cases}$
    $\tilde{p}_i: \prod_{i \in I} r_i(A_i \otimes X_i) \to r_i(A_i \otimes X_i)$ --- каноническая проекция, $p_{A_i}: \prod_{i \in I} A_i \to A_i$ --- каноническая проекция, $q_{X_i}: X_i \to \coprod_{i \in I} X_i$ --- естественное вложение, $p_i = m_i \circ \tilde{p}_i$, $m_i: r_i(A_i \otimes X_i) \to D_i$ --- естественное вложение, для любых $a \in \prod_{i \in I} A_i$, $x_i \in X_i$, $i,j \in I$, является произведением объектов $r_i$, $i \in I$, категории $Chu_{\widetilde{Set}}$.
\end{theorem}
\begin{proof}
    Пусть условия теоремы выполнены. Поскольку $p_i(r(a,q_{X_i}(x_i))) = m_i(\tilde{p}_i(r(a,q_{X_i}(x_i)))) = \tilde{p}_i(r(a,q_{X_i}(x_i))) = r_i(p_{A_i}(a),x_i)$ для любых $a \in \prod_{i \in I} A_i$, $x_i \in X_i$, $i \in I$, то $(p_{A_i},q_{X_i},p_i)$ --- преобразование Чу категории $Chu_{\widetilde{Set}}$ для любого $i \in I$.

    Пусть $t: B \times Y \to D$ --- объект категории $Chu_{\widetilde{Set}}$, $(f_i,g_i,h_i): t \to r_i$ --- преобразование Чу категории $Chu_{\widetilde{Set}}$ для любого $i \in I$. Для любого $h_i: D \to D_i$, $i \in I$, построим морфизм $\tilde{h}_i: D \to r_i(A_i \times X_i)$ категории $Set_\star$ следующим образом: $\tilde{h}_i(d) = 
    \begin{cases}
        h_i(d),& \text{если } d = t(b,g_i(x_i)) \text{ для некоторых } b \in B, x_i \in X_i,\\
        \star,& \text{иначе},
    \end{cases}$
    для любого $d \in D$. Поскольку $(f_i,g_i,h_i)$ --- преобразование Чу категории $Chu_{\widetilde{Set}}$ для любого $i \in I$, то $m_i(\tilde{h}_i(t(b,g_i(x_i)))) = h_i(t(b,g_i(x_i))) = r_i(f_i(b),x_i)$ для любых $b \in B$, $x_i \in X_i$. Следовательно, $(f_i,g_i,m_i \circ \tilde{h}_i): t \to r_i$ --- преобразование Чу категории $Chu_{\widetilde{Set}}$ для любого $i \in I$.

    Так как $\prod_{i \in I} A_i$ и $\prod_{i \in I} r_i(A_i \times X_i)$ --- произведения $A_i$, $ i \in I$, и $r_i(A_i \times X_i)$, $i \in I$, соответственно, а $\coprod_{i \in I} X_i$ --- копроизведение $X_i$, $i \in I$, в категории $Set_{\star}$, то существуют единственные морфизмы $\tilde{f}: B \to \prod_{i \in I} A_i$, $\tilde{g}: \coprod_{i \in I} X_i \to Y$, $\tilde{h}: D \to \prod_{i \in I} r_i(A_i \times X_i)$ категории $Set_{\star}$ такие, что $f_i = p_{A_i} \circ \tilde{f}$, $g_i = \tilde{g} \circ q_{X_i}$ и $\tilde{h}_i = \tilde{p}_i \circ \tilde{h}$ для любого $i \in I$, то есть $p_{A_i}(\tilde{f}(b)) = f_i(b)$, $\tilde{g}(x_i) = g_i(x_i)$ и $\tilde{p}_i(\tilde{h}(d)) = \tilde{h}_i(d)$ для любых $b \in  B$, $x_i \in X_i$, $d \in D$, $i \in I$.

    Покажем, что $(\tilde{f},\tilde{g},\tilde{h})$ --- преобразование Чу категории $Chu_{\widetilde{Set}}$, то есть для любых $b \in B$, $x_i \in X_i$, $i \in I$, имеет место равенство $\tilde{h}(t(b,\tilde{g}(x_i))) = r(\tilde{f}(b),x_i)$. Поскольку $\prod_{i \in I} r_i(A_i \times X_i)$ вместе с семейством $\{\tilde{p}_i: \prod_{i \in I} r_i(A_i \times X_i) \to r_i(A_i \times X_i) \mid i \in I\}$ морфизмов категории $Set_\star$ является произведением объектов $r_i(A_i \times X_i)$, $i \in I$, то для любых $b \in B$, $x \in X_i$ равенство $\tilde{h}(t(b,\tilde{g}(x_i))) = r(\tilde{f}(b),x_i)$ равносильно равенству $\tilde{p}_j(\tilde{h}(t(b,\tilde{g}(x_i)))) = \tilde{p}_j(r(\tilde{f}(b),x_i))$ для любых $b \in B$, $x_i \in I$, $i,j \in I$. Тогда из равенств
    \begin{multline*}
        \tilde{p}_j(\tilde{h}(t(b,\tilde{g}(x_i)))) = \tilde{h}_j(t(b,g_i(x_i))) =\\= 
        \begin{cases}
            h_i(t(b,g_i(x_i))) = r_i(f_i(b),x_i) = r_j(f_j(b),x_i),& \text{если } i = j\\
            \star,& \text{если } i \ne j
        \end{cases} =\\=
        \tilde{p}_j(r(\tilde{f}(b),x_i))
    \end{multline*}
    для любых $b \in B$, $x_i \in X_i$, $i,j \in I$ следует, что $(\tilde{f},\tilde{g},\tilde{h})$ --- преобразование Чу категории $Chu_{\widetilde{Set}}$.

    Покажем, что $(p_{A_i},q_{X_i},p_i) \circ (\tilde{f},\tilde{g},\tilde{h}) = (f_i,g_i,h_i)$ для любого $i \in I$, то есть $p_{A_i} \circ \tilde{f} = f_i$, $\tilde{g} \circ q_{X_i} = g_i$ и для любого $d \in D$ если $d = t(b,g_i(x_i))$, то $(p_i \circ \tilde{h})(d) = h_i(d) = r_i(f_i(b),x_i)$. Из определения $\tilde{f}$ и $\tilde{g}$ следует, что $p_{A_i} \circ \tilde{f} = f_i$ и $\tilde{g} \circ q_{X_i} = g_i$ для любого $i \in I$. Пусть $d \in D$ и $d = t(b,g_i(x_i))$ для некоторых $b \in B$, $x_i \in X_i$, $i \in I$. Тогда
    \begin{multline*}
        (p_i \circ \tilde{h})(d) = (m_i \circ \tilde{p}_i \circ \tilde{h})(d) = (m_i \circ \tilde{h}_i)(d) = m_i(\tilde{h}_i(d)) = \tilde{h}_i(d) =\\=
        \tilde{h}_i(t(b,g_i(x_i))) = h_i(t(b,g_i(x_i))) = h_i(d) = r_i(f_i(b),x_i).
    \end{multline*}
    Следовательно, $(p_{A_i},q_{X_i},p_i) \circ (\tilde{f},
    \tilde{g},\tilde{h}) = (f_i,g_i,h_i)$ для любого $i \in I$.

    Покажем, что $(\tilde{f},\tilde{g},\tilde{h})$ --- единственный морфизм такой, что $(p_{A_i},q_{X_i},p_i) \circ (\tilde{f},\tilde{g},\tilde{h}) = (f_i,g_i,h_i)$ для любого $i \in I$. Пусть $(f',g',h'): t \to r$ --- преобразование Чу категории $Chu_{\widetilde{Set}}$ такое, что $(p_{A_i},q_{X_i},p_i) \circ (f',g',h') = (f_i,g_i,h_i)$ для любого $i \in I$, то есть $p_{A_i} \circ f' = f_i$, $g' \circ q_{X_i} = g_i$ и для любого $d \in D$ если $d = t(b,g_i(x_i))$, то $(p_i \circ h')(d) = h_i(d) = r_i(f_i(b),x_i)$ для любого $i \in I$. Из определения $\tilde{f}$ и $\tilde{g}$ следует, что $f' = \tilde{f}$ и $g' = \tilde{g}$. Осталось показать, что $(\tilde{f},\tilde{g},h') = (\tilde{f},\tilde{g},\tilde{h})$, то есть для любого $d \in D$ если $d = t(b,g_i(x_i))$, то $h'(d) = \tilde{h}(d) = r(\tilde{f},x_i)$. Поскольку $\prod_{i \in I} r_i(A_i \times X_i)$ вместе с семейством $\{\tilde{p}_i: \prod_{i \in I} r_i(A_i \times X_i) \to r_i(A_i \times X_i) \mid i \in I\}$ морфизмов категории $Set_\star$ является произведением объектов $r_i(A_i \times X_i)$, $i \in I$, то для любых $b \in B$, $x \in X_i$ равенства $h'(d) = \tilde{h}(d) = r(\tilde{f}(b),x_i)$ равносильны равенствам $\tilde{p}_j(h'(d)) = \tilde{p}_j(\tilde{h}(d)) = \tilde{p}_j(r(\tilde{f}(b),x_i))$ для любых $b \in B$, $x_i \in I$, $i,j \in I$.  Пусть $d \in D$ и $d = t(b,g_i(x_i))$. Тогда если $i = j$, то
    $$
        \tilde{p}_j(h'(d)) = m_i(\tilde{p}_i(h'(d))) = p_i(h'(d)) = p_i(\tilde{h}(d)) = h_i(d) = r_i(f_i(b),x_i),
    $$
    иначе, если $i \ne j$, то
    $$
        \tilde{p}_j(h'(d)) = \tilde{p}_j(h'(t(b,g'(x_i)))) = \tilde{p}_j(r(f'(b),x_i)) = \star = \tilde{p}_j(r(\tilde{f}(b),x_i)) = \tilde{p}_j(\tilde{h}(t(b,\tilde{g}(x_i)))).
    $$
    Следовательно, $(\tilde{f},\tilde{g},\tilde{h}) = (f',g',h')$ в категории $Chu_{\widetilde{Set}}$.
\end{proof}