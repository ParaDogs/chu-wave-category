\documentclass[a4paper,12pt]{article}
\usepackage{framed}
\usepackage[top=2cm, bottom=2cm]{geometry}
\usepackage[utf8]{inputenc}
\usepackage[english,russian]{babel}
\usepackage{amsthm,amssymb,amsfonts,amsmath,cite,enumerate}
\usepackage[all]{xy}

\newtheorem{statement}{Утверждение}
\newtheorem{lemma}{Лемма}
\newtheorem{theorem}{Теорема}
\newtheorem{consequence}{Следствие}
\newtheorem{remark}{Замечание}

\newcommand{\SAct}{S\text{-}Act}
\newcommand{\GAct}{G\text{-}Act}

\begin{document}

\section*{Предварительные сведения}

Зафиксируем категорию $C$ и объект $\star \in Ob(C)$. Определим следующую категорию, называемую \textit{пунктированной категорией $C$}:
\begin{itemize}
    \item Объектами категории являются пары $(k,u)$, где $k \in Ob(C)$, $u \in Mor(\star,k)$;
    \item Морфизмом из $(k,u)$ в $(l,v)$ является элемент $f \in Mor(k,l)$ такой, что $f \circ u = v$,
\end{itemize}
где комозиция наследуется из категории $C$.

Пусть $(C,\otimes,\star)$ --- моноидальная, замкнутая, биполная категория с терминальным объектом $\star$ и пусть $C_\star$ --- пунктированная категория $C$. Для объектов $k$ и $l$ из $C_*$ определим их \textit{скрещенное произведение} $k \wedge l$ как объект из $C_\star$, задаваемый следующим расслоенным произведением в $C$:
$$\xymatrix{
    (k \otimes \star) \sqcup (\star \otimes l) \ar[r] \ar[d] & k \otimes l \ar[d]\\
    \star \ar[r] & k \wedge l.
}$$

Везде ниже $(C,\times,\star)$ означает моноидальную замкнутую биполную категорию с терминальным объектом $\star$, объектами $C$ являются алгебраические системы и для любой подсистемы $B$ произвольной алгебраической системы $A \in Ob(C)$ существует подсистема $C$ системы $A$ такая, что $A = B \sqcup C$. Скрещенное произведение объектов $(A,u), (B,v) \in Ob(C_\star)$ изоморфино объекту $((A \sqcup B)/\Theta, w)$, где $\Theta = \{(a,\star) \mid a \in A\} \cup \{(\star,b) \mid a \in A\} \cup \{(c,c) \mid c \in A \cup B\}$ и $w(\star)/\Theta = \{(a,\star) \mid a \in A\} \cup \{(\star,b) \mid b \in B\}$. В пунктированной категории $C_\star$ объекты $(A,u)$ будет обозначаться через $A_\star$ или просто $A$ и $\star$ будет рассматриваться как элемент $A$.

НАПОМНИМ, ЧТО В КАТЕГОРИИ МНОЖЕСТВ... ПРОИЗВЕДЕНИЕ, КОПРОИЗВЕДЕНИЕ, РАССЛОЕННОЕ ПРОИЗВЕДЕНИЕ, УРАВНИТЕЛЬ, КОУРАВНИТЕЛЬ, ИНИЦИАЛЬНЫЙ И ТЕРМИНАЛЬНЫЙ ОБЪЕКТЫ.

\subsection*{Сведения из категории $\SAct$ полигонов над моноидом $S$}

Пусть $S$ --- моноид, $1$ --- единица моноида $S$. Множество $A$ (возможно пустое) называется левым $S$-полигоном (или полигоном над $S$, или просто полигоном), если существует отображение $S \times A \to A, (s,a) \mapsto sa$, такое, что для любых $a \in A$ и $s,t \in S$
$$
    1a = a\ \text{и}\ s(ta) = (st)a.
$$
Иными словами, полигон --- это множество $A$, на котором определено действие моноида $S$, причем единица $S$ действует на $A$ тождественно. Элемент $\theta \in A$ такой, что $s\theta = \theta$ для любого $s \in S$, называется нулем полигона $A$. Одноэлементный полигон $\Theta = \{\theta\}$ называется нулевым полигоном.

Подмножество $B$ множества $A$, замкнутое относительно действия $S$?, называется подполигоном полигона $A$. Ясно, что $S$, а также любой его левый идеал, можно рассматривать как полигон. Полигон $A$ называется конечно порожденным, если существуют $a_1,\ldots,a_n \in A$ такие, что $A = \bigcup_{i=1}^n Sa_i$; причем, если $n=1$, то $A$ называется циклическим полигоном. Конгруэнцией полигона $A$ называется отношение эквивалентности $\varepsilon$ на $A$ такое, что 
$$
    (a,b) \in \varepsilon \Rightarrow (sa,sb) \in \varepsilon 
$$
для любых $a,b \in A$, $s \in S$. Класс конгруэнции $\varepsilon$ полигона $A$ с представителем $a$ обозначается через $a/\varepsilon$. Конгруэнцией полигона $A$, порожденной множеством $I \subseteq A^2$, называется наименьшая конгруэнция полигона $A$, содержащая $I$. Пусть $B$ --- подполигон полигона $A$. Конгруэнция $\rho_B = \{(b_1,b_1) \mid b_1, b_2 \in B\} \cup \{(a, a) \mid a \in A\}$ полигона $A$ называется конгруэнцией Риса.

Отображение $f: A \to B$ такое, что $f(sa) = sf(a)$ для любых $a \in A$, $s \in S$, называется $S$-морфизмом (гомоморфизмом или морфизмом) из полигона $A$ в полигон $B$, причем если $A = \varnothing$, то $f = \varnothing$. Через $Hom(A,B)$ обозначим множество всех морфизмов из полигона $A$ в полигон $B$. Тождественное отображение из полигона $A$ в полигон $A$ обозначается через $1_A$. Класс всех полигонов над моноидом $S$ с морфизмами образуют категорию $\SAct$.

Пустой полигон является инициальным объектом категории $\SAct$, нулевой полигон является терминальным объектом категории $\SAct$.

В категории $\SAct$ произведением полигонов $A_i$, $i \in I$, является их декартово произведение $\prod_{i \in I}A_i$, действие моноида $S$ на котором определяется покомпонентно; копроизведением полигонов $A_i$, $i \in I$, является их дизъюнктное объединение $\coprod_{i \in I}A_i$.

Расслоенным произведением пары $(g_1,g_2)$, где $g_i: A_i \to B$ --- морфизм полигонов $(i \in \{1,2\})$ в категории $\SAct$, является полигон $A_1 \underset{B}{\times} A_2 = \{(a_1,a_2 \in A_1 \times A_2 \mid g_1(a_1) = g_2(a_2))\}$ с морфизмами $p_i: A_1 \underset{B}{\times} A_2 \to A_i$, $i \in \{1,2\}$, являющимися проекциями на $i$-ую координату $(i \in \{1,2\})$. Расслоенной суммой пары $(f_1,f_2)$, где $f_i: B \to A_i$ --- морфизм полигонов $(i \in \{1,2\})$ в категории $\SAct$, является полигон $A_1 \underset{B}{\sqcup} A_2 = (A_1 \sqcup A_2)/\mu(f_1,f_2)$ с морфизмами $q_i: A_i \to A_1 \underset{B}{\sqcup} A_2$, $i \in \{1,2\}$, где $\mu(f_1,f_2)$ --- конгруэнция полигона $A_1 \sqcup A_2$, порожденная множеством $\{(f_1(b),f_2(b)) \mid b \in B\}$, $q_i = \pi u_i$, $u_i: A_i \to A_1 \sqcup A_2$ --- вложение, $\pi$ --- канонический эпиморфизм $A_1 \sqcup A_2$ на $A_1 \underset{B}{\sqcup} A_2$ $(i \in \{1,2\})$. Заметим, что $A_2 \underset{\Theta}{\times} A_2 = A_1 \times A_2$ и $A_1 \underset{\varnothing}{\sqcup} A_2 = A_1 \sqcup A_2$.

Уравнителем морфизмов $f_1,f_2: A \to B$ в категории $\SAct$ является полигон $E = \{a \in A \mid f_1(a) = f_2(a)\}$ с естественным вложением $i: E \to A$. Коуравнителем морфизмов $g_1,g_2: B \to A$ в категории $\SAct$ является полигон $A/\nu(g_1,g_2)$ с каноническим эпиморфизмом $\pi: A \to A/\nu(g_1,g_2)$, где $\nu(g_1,g_2)$ --- конгруэнция полигона $A$, порожденная множеством $\{(g_1(b), g_2(b)) \mid b \in B\}$.

Для коммутативного моноида $S$ тензорное произведение $A \otimes B$ полигонов $A$ и $B$ определяется как фактормножество множества $A \times B$ по отношению эквивалентности, порожденному множеством $\{(sa,b), (a,sb) \mid a \in A, b \in B, s \in S\}$. Для $a \in A$ и $b \in B$ класс эквивалентности с представителем $(a,b)$ будем обозначать $a \otimes b$. Тензорное произведение $A \otimes B$ полигонов $A$ и $B$ является полигоном относительно действия моноида $S$, определенного следующим образом:
$$
    s(a \otimes b) = sa \otimes b = a \otimes sb,
$$
для любых $a \in A$, $b \in B$, $s \in S$.

\subsection*{Определение категории $Chu_{\widetilde{V}}$}

Ниже под $V$ будем понимать либо пунктированную категорию множеств $Set_\star$, где объектом $\star$ является одноэлементное множество $\{\star\}$, а тензорным произведением $\otimes$ --- декартово произведение множеств, либо пунктированную категорию $\GAct_{\star}$ полигонов над абелевой группой $G$, где объектом $\star$ является нулевой полигон $\Theta$, а тензорным произведением $\otimes$ --- тензорное произведение полигонов (поскольку группа $G$ является коммутативным моноидом).

Категория пространств Чу $Chu_{\widetilde{V}}$, где $V$ --- это $Set$ или $\GAct$, определяется следующим образом: объектами категории выступают всевозможные морфизмы категории $V_{\star}$ вида $r: A \otimes X \to D$, где $A,X,D \in Ob(V_{\star})$, где $\otimes$ --- тензорное произведение в категории $V_{\star}$, называемые пространствами Чу категории $Chu_{\widetilde{V}}$. Пусть $r: A \otimes X \to D_1$, $s: B \otimes Y \to D_2$ --- пространства Чу категории $Chu_{\widetilde{V}}$. Определим множество $M(r,s)$ как множество троек $(f,g,h)$ морфизмов категории $V_{\star}$, где $f: A \to B$, $g: Y \to X$ и $h: D_1 \to D_2$ таких, что $h(r(a,g(y))) = s(f(a),y)$ для любых $a \in A$, $y \in Y$. Определим отношение эквивалентности $\sim$ на множестве $M(r,s)$ следующим образом: $(f_1,g_1,h_1) \sim (f_2,g_2,h_2)$ тогда и только тогда $f_1 = f_2$, $g_1 = g_2$ и для любого $d_1 \in D_1$ если $d_1 = r(a,g_1(y))$, то $h_1(d_1) = h_2(d_2) = s(f_1(a),y)$. Фактормножество $M(r,s)/\sim$ называется множеством морфизмов между $r$ и $s$ (преобразований Чу категории $Chu_{\widetilde{V}}$). Пусть $t: C \otimes Z \to D_3$ --- пространство Чу категории $Chu_{\widetilde{V}}$, $(f_1,g_1,h_1): r \to s$, $(f_2,g_2,h_2): s \to t$, тогда композиция этих морфизмов задается равенством $(f_2,g_2,h_2) \circ (f_1,g_1,h_1) = (f_2 \circ f_1, g_1 \circ g_2, h_2 \circ h_1)$.

\section*{Категория $Chu_{\widetilde{Set}}$}

\begin{theorem}\label{epimorphism}
    Пусть $r: A \times X \to D$ и $t: B \times Y \to C$ --- объекты категории $Chu_{\widetilde{Set}}$. Преобразование $(f,g,h): r \to t$ категории $Chu_{\widetilde{Set}}$ является эпиморфизмом тогда и только тогда, когда $f: A \to B$ --- эпиморфизм, $g: Y \to X$ --- мономорфизм категории $Set_{\star}$.
\end{theorem}
\begin{proof}
    \textbf{Необходимость.} Пусть $(f,g,h): r \to t$ --- эпиморфизм категории $Chu_{\widetilde{Set}}$.

    Покажем, что $f$ --- эпиморфизм категории $Set_{\star}$. Предположим, что $B_1 \ne B$, где $B_1 = f(A)$. Через $B_0$ обозначим фактормножество множества $B$ по отношению эквивалентности $\sim$ такому, что $b \sim b' \Leftrightarrow b, b' \in B_1$. Определим объект $w: (B_0 \times B) \times Y \to C$ и морфизмы $(f_1,1_Y,1_C), (f_2,1_Y,1_C): t \to w$ категории $Chu_{\widetilde{Set}}$ следующим образом: $w((b_0,b),y) = t(b,y)$, $f_1(b) = (B_1,b)$, $f_2(b) = (b/\sim,b)$ для любых $b \in B$, $b_0 \in B_0$, $y \in Y$. Корректность определения морфизмов $(f_1,1_Y,1_C)$, $(f_2,1_Y,1_C)$ следует из равенств:
    $$
        t(b,y) = w(f_1(b),y) = w(f_2(b),y),
    $$
    где $b \in B$, $y \in Y$. Если $a \in A$, то $f(a) \in B_1$ и $f_1(f(a)) = f_2(f(a))$, то есть $f_1 \circ f = f_2 \circ f$. Тогда $(f_1,1_Y,1_C) \circ (f,g,h) = (f_2,1_Y,1_C) \circ (f,g,h)$. Поскольку $(f,g,h)$ --- эпиморфизм категории $Chu_{\widetilde{Set}}$, то $f_1 = f_2$. Противоречие.

    Покажем, что $g$ --- мономорфизм категории $Set_{\star}$. Предположим, что существуют различные $y_1, y_2 \in Y$ такие, что $g(y_1) = g(y_2)$. Пусть $b \in B$. Покажем, что $t(b,y_1) = t(b,y_2)$. Так как $f$ --- эпиморфизм, то существует $a \in A$ такой, что $f(a) = b$. Тогда
    $$
        t(b,y_1) = t(f(a),y_1) = h(r(a,g(y_1))) = h(r(a,g(y_2))) = t(f(a),y_2) = t(b,y_2).
    $$
    Определим объект $w: B \times \{\star,y_0\} \to C$ и морфизмы $(1_B,g_1,1_C), (1_B,g_2,1_C): t \to w$ категории $Chu_{\widetilde{Set}}$ следующим образом: $w(b,\star) = \star$, $w(b,y_0) = t(b,y_1)$, $g_1(\star) = g_2(\star) = \star$, $g_1(y_0) = y_1$, $g_2(y_0) = y_2$ для любых $b \in B$. Корректность определения морфизмов $(1_B,g_1,1_C), (1_B,g_2,1_C)$ следует из равенств:
    $$
        w(b,\star) = t(b,g_1(\star)) = t(b,g_2(\star)) = t(b,\star) = \star,
    $$
    $$
        w(b,y_0) = t(b,g_1(y_0)) = t(b,y_1) = t(b,y_2) = t(b,g_2(y_0))
    $$
    для любого $b \in B$. Так как
    $$
        (g \circ g_1)(\star) = g(g_1(\star)) = \star = g(g_2(\star)) = (g \circ g_2)(\star),
    $$
    $$
        (g \circ g_1)(y_0) = g(g_1(y_0)) = g(y_1) = g(y_2) = g(g_2(y_0)) = (g \circ g_2)(y_0),
    $$
    то $(1_B,g_1,1_C) \circ (f,g,h) = (1_B,g_2,1_C) \circ (f,g,h)$. Поскольку $(f,g,h)$ --- эпиморфизм категории $Chu_{\widetilde{Set}}$, то $g_1 = g_2$. Противоречие.

    \textbf{Достаточность.} Пусть $(f,g,h): r \to t$ --- преобразование Чу категории $Chu_{\widetilde{Set}}$, где $f$ --- эпиморфизм, $g$ --- мономорфизм категории $Set_{\star}$. Предположим, что $(f_1,g_1,h_1), (f_2,g_2,h_2): t \to w$ --- преобразования Чу категории $Chu_{\widetilde{Set}}$ такие, что $(f_1,g_1,h_1) \circ (f,g,h) = (f_2,g_2,h_2) \circ (f,g,h)$, где $w: E \times Z \to P$ --- пространство Чу категории $Chu_{\widetilde{Set}}$. Тогда $f_1 \circ f = f_2 \circ f$, $g \circ g_1 = g \circ g_2$ и для любого $d \in D$ если $d = r(a,(g \circ g_1)(z)) = r(a,(g \circ g_2)(z))$, то $(h_1 \circ h)(d) = (h_2 \circ h)(d) = w((f_1 \circ f)(a), z) = w((f_2 \circ f)(a), z)$. Покажем, что для любого $c \in C$ если $c = t(b,g_1(z))$, то $h_1(c) = h_2(c) = w(f_1(b),z)$. Пусть $c \in C$ и $c = t(b,g_1(z))$ для некоторых $b \in B$, $z \in Z$. Поскольку $f$ --- эпиморфизм, $g$ --- мономорфизм категории $Set_{\star}$, то $f_1 = f_2$, $g_1 = g_2$. Так как $f$ --- эпиморфизм, то $b = f(a)$ для некоторого $a \in A$. Тогда 
    \begin{multline*}
        h_1(c) = h_1(t(b,g_1(z))) = h_1(t(f(a),g_1(z))) = w(f_1(f(a)),z) =\\=
        w(f_2(f(a)),z) = h_2(t(f(a),g_2(z))) = h_1(t(b,g_1(z))) = h_2(c).
    \end{multline*}
    Следовательно, $(f_1,g_1,h_1) = (f_2,g_2,h_2)$ в категории $Chu_{\widetilde{Set}}$ и преобразование $(f,g,h): r \to t$ является эпиморфизмом категории $Chu_{\widetilde{Set}}$.
\end{proof}

\begin{theorem}\label{monomorphism}
    Пусть $r: A \times X \to D$ и $t: B \times Y \to C$ --- объекты категории $Chu_{\widetilde{Set}}$. Преобразование $(f,g,h): r \to t$ категории $Chu_{\widetilde{Set}}$ является мономорфизмом тогда и только тогда, когда $f: A \to B$ --- мономорфизм, $g: Y \to X$ --- эпиморфизм категории $Set_{\star}$.
\end{theorem}
\begin{proof}
    \textbf{Необходимость.} Пусть $(f,g,h): r \to t$ --- мономорфизм категории $Chu_{\widetilde{Set}}$.

    Покажем, что $g$ --- эпиморфизм категории $Set_{\star}$. Предположим, что $X_1 \ne X$, где $X_1 = g(Y)$. Через $X_0$ обозначим фактормножество множества $X$ по отношению эквивалентности $\sim$ такому, что $x \sim x' \Leftrightarrow x,x' \in X_1$. Определим объект $w: A \times (X_0 \times X) \to D$ и морфизмы $(1_A,g_1,1_D), (1_A,g_2,1_D): w \to r$ категории $Chu_{\widetilde{Set}}$ следующим образом: $w(a,(x_0,x)) = r(a,x)$, $g_1(x) = (X_1,x)$, $g_2(x) = (x/\sim, x)$, для любых $a \in A$, $x \in X$, $x_0 \in X_0$. Из определения объекта $w$ категории $Chu_{\widetilde{Set}}$ следует равенство
    $$
        w(a,g_1(x)) = w(a,g_2(x)) = r(a,x),
    $$ 
    где $a \in A$, $x \in X$, что доказывает корректность определения морфизмов $(1_A,g_1,1_D)$, $(1_A,g_2,1_D)$. Если $y \in Y$, то $g(y) \in X_1$ и $g_1(g(y)) = g_2(g(y))$, то есть $g_1 \circ g = g_2 \circ g$. Тогда $(f,g,h) \circ (1_A,g_1,1_D) = (f,g,h) \circ (1_A,g_2,1_D)$. Поскольку $(f,g,h)$ --- мономорфизм категории $Chu_{\widetilde{Set}}$, то $g_1 = g_2$. Противоречие.

    Покажем, что $f$ --- мономорфизм категории $Set_{\star}$. Предположим, что существуют различные $a_1, a_2 \in A$ такие, что $f(a_1) = f(a_2)$. Определим объект $w: \{\star,a',a''\} \times X \to D'$, где $D' = r(\{a_1,a_2\} \times X)$ и морфизмы $(f_1,1_X,1_D), (f_2,1_X,1_D): w \to r$ категории $Chu_{\widetilde{Set}}$ следующим образом $w(\star,x) = \star$, $w(a',x) = r(a_1,x)$, $w(a'',x) = r(a_2,x)$, $f_1(\star) = f_2(\star) = \star$, $f_1(a') = f_2(a'') = a_1$, $f_1(a'') = f_2(a') = a_2$, $h_1(r(a_1,x)) = h_2(r(a_2,x)) = r(a_1,x)$, $h_1(r(a_2,x)) = h_2(r(a_1,x)) = r(a_2,x)$ для любых $x \in X$. Корректность определения морфизмов $(f_1,1_X,h_1), (f_2,1_X,h_2)$ следует из равенств
    $$
        h_1(w(\star,x)) = h_2(w(\star,x)) = \star = r(f_1(\star),x) = r(f_2(\star),x),
    $$
    $$
        h_1(w(a',x)) = h_1(r(a_1,x)) = r(a_1,x) = r(f_1(a'),x),
    $$
    $$
        h_2(w(a',x)) = h_2(r(a_1,x)) = r(a_2,x) = r(f_2(a'),x),
    $$
    $$
        h_1(w(a'',x)) = h_1(r(a_2,x)) = r(a_2,x) = r(f_1(a''),x),
    $$
    $$
        h_2(w(a'',x)) = h_2(r(a_2,x)) = r(a_1,x) = r(f_2(a''),x)
    $$
    для любых $x \in X$.
    Так как $(f \circ f_1)(\star) = \star = (f \circ f_2)(\star)$ и $f(f_1(a')) = f(f_2(a'')) = f(a_1) = f(a_2) = f(f_1(a'')) = f(f_2(a'))$,
    то $f \circ f_1 = f \circ f_2$. Из того, что $g$ --- эпиморфизм, следует, что $x = g(y)$ для некоторого $y \in Y$ для любого $x \in X$. Тогда из равенств
    $$
        (h \circ h_1)(w(\star,x)) = \star = (h \circ h_2)(w(\star,x)),
    $$
    \begin{multline*}
        (h \circ h_1)(w(a',x)) = h(h_1(r(a_1,x))) = h(r(a_1,g(y))) = t(f(f_1(a')),x)  =\\= 
        t(f(a_1),y) = t(f(a_2),y) = h(r(a_2,g(y))) = h(h_2(r(a_1,x))) = (h \circ h_2)(w(a',x)),
    \end{multline*}
    \begin{multline*}
        (h \circ h_1)(w(a'',x)) = h(h_1(r(a_2,x))) = h(r(a_2,g(y))) = t(f(f_1(a'')),x)  =\\= 
        t(f(a_2),y) = t(f(a_1),y) = h(r(a_1,g(y))) = h(h_2(r(a_2,x))) = (h \circ h_2)(w(a'',x))
    \end{multline*}
    для любых $x \in X$ следует, что для любого $d' \in D'$ если $d' = w(a,x)$, то $(h \circ h_1)(d') = (h \circ h_2)(d') = t((f \circ f_1)(a),x)$. Следовательно, $(f,g,h) \circ (f_1,1_X,h_1) = (f,g,h) \circ (f_2,1_X,h_2)$. Поскольку $(f,g,h)$ --- мономорфизм категории $Chu_{\widetilde{Set}}$, то $f_1 = f_2$, то есть $a_1 = a_2$. Противоречие.

    \textbf{Достаточность.} Пусть $(f,g,h): r \to t$ --- преобразование Чу категории $Chu_{\widetilde{Set}}$, где $f$ --- мономорфизм, $g$ --- эпиморфизм категории $Set_{\star}$. Предположим, что $(f_1,g_1,h_1), (f_2,g_2,h_2): w \to r$ --- преобразования Чу категории $Chu_{\widetilde{Set}}$ такие, что $(f,g,h) \circ (f_1,g_1,h_1) = (f,g,h) \circ (f_2,g_2,h_2)$, где $w: E \times Z \to P$ --- пространство Чу категории $Chu_{\widetilde{Set}}$. Тогда $f \circ f_1 = f \circ f_2$, $g_1 \circ g = g_1 \circ g$ и для любого $p \in P$ если $p = w(e,(g_1 \circ g)(y)) = w(e,(g_2 \circ g)(y))$, то $(h \circ h_1)(p) = (h \circ h_2)(p) = t((f \circ f_1)(e), y) = t((f \circ f_2)(e), y)$. Покажем, что для любого $p \in P$ если $p = w(e,g_1(x))$, то $h_1(p) = h_2(p) = r(f_1(e),x)$. Пусть $p \in P$ и $p = w(e,g_1(x))$ для некоторых $e \in E$, $x \in X$. Поскольку $f$ --- мономорфизм, $g$ --- эпиморфизм категории $Set_{\star}$, то $f_1 = f_2$, $g_1 = g_2$. Тогда 
    $$
        h_1(p) = h_1(w(e,g_1(x))) = r(f_1(e),x) = r(f_2(e),x) = h_2(w(e,g_2(x))) = h_2(p).
    $$
    Следовательно, $(f_1,g_1,h_1) = (f_2,g_2,h_2)$ в категории $Chu_{\widetilde{Set}}$ и преобразование $(f,g,h): r \to t$ является мономорфизмом категории $Chu_{\widetilde{Set}}$.
\end{proof}

\begin{theorem}\label{coproduct}
    Пусть $r_i: A_i \times X_i \to D_i$ --- объекты категории $Chu_{\widetilde{Set}}$, $i \in I$. Объект $r: \coprod_{i \in I} A_i \times \prod_{i \in I} X_i \to \coprod_{i \in I} D_i$ категории $Chu_{\widetilde{Set}}$ вместе с семейством $\{(q_{A_i},p_{X_i},q_{D_i}): r \to r_i \mid i \in I\}$ преобразований Чу категории $Chu_{\widetilde{Set}}$, где $r(a,x) = r_i(a,p_{X_i}(x)) = r_i(a, x_i)$, $q_{A_i}: A_i \to \coprod_{i \in I} A_i$, $q_{D_i}: D_i \to \coprod_{i \in I} D_i$ --- естественные вложения, $p_{X_i}: \prod_{i \in I} X_i \to X_i$ --- каноническая проекция для любых $i \in I$, $a \in A_i$, $x \in \prod_{i \in I} X_i$, является копроизведением объектов $r_i$, $i \in I$, категории $Chu_{\widetilde{Set}}$.
\end{theorem}
\begin{proof}
    Пусть условия теоремы выполнены. Поскольку $q_{D_i}(r_i(a,p_{X_i}(x))) = r_i(a,x_i) = r(a,x) = r(q_{A_i}(a),x)$ для любых $a \in A_i$, $x \in \prod_{i \in I} X_i$, $i \in I$, то $(q_{A_i}, p_{X_i}, q_{D_i})$ --- преобразование Чу категории $Chu_{\widetilde{Set}}$ для любого $i \in I$.

    Пусть $t: B \times Y \to D$ --- объект категории $Chu_{\widetilde{Set}}$, $(f_i,g_i,h_i): r_i \to t$ --- преобразование Чу категории $Chu_{\widetilde{Set}}$ для любого $i \in I$. Так как $\coprod_{i \in I} A_i$ и $\coprod_{i \in I} D_i$ --- копроизведения $A_i$, $ i \in I$, и $D_i$, $i \in I$, соответственно, а $\prod_{i \in I} X_i$ --- произведение $A_i$, $i \in I$ в категории $Set_{\star}$, то существуют единственные морфизмы $\tilde{f}: \coprod_{i \in I} A_i \to B$, $\tilde{g}: Y \to \prod_{i \in I} X_i$, $\tilde{h}: \coprod_{i \in I} D_i \to D$ категории $Set_{\star}$ такие, что $f_i = \tilde{f} \circ q_{A_i}$, $g_i = p_{X_i} \circ \tilde{g}$ и $h_i = \tilde{h} \circ q_{D_i}$ для любого $i \in I$, то есть $\tilde{f}(a) = f_i(a)$, $\tilde{g}(y)_i = g_i(y)$ и $\tilde{h}(d) = h_i(d)$ для любых $a \in A_i$, $y \in Y$, $d \in D_i$. Следовательно, $(\tilde{f},\tilde{g},\tilde{h}) \circ (q_{A_i}, p_{X_i}, q_{D_i}) = (f_i, g_i, h_i)$ для любого $i \in I$ в категории $Chu_{\widetilde{Set}}$.

    Покажем, что $(\tilde{f},\tilde{g},\tilde{h})$ --- преобразование Чу категории $Chu_{\widetilde{Set}}$, то есть для любых $a \in \coprod_{i \in I} A_i$, $y \in Y$ имеет место равенство $\tilde{h}(r(a,\tilde{g}(y))) = t(\tilde{f}(a),y)$. Так как $(q_{A_i},p_{X_i},q_{D_i})$ --- преобразование Чу категории $Chu_{\widetilde{Set}}$, то $r(q_{A_i}(a),\tilde{g}(y)) = q_{D_i}(r_i(a,(p_{X_i} \circ \tilde{g})(y)))$ для любых $a \in A_i$, $y \in Y$. Из того, что $h_i = \tilde{h} \circ q_{D_i}$ и $g_i = p_{X_i} \circ \tilde{g}$ следует, что $\tilde{h}(r(a,\tilde{g}(y))) = (\tilde{h} \circ q_{D_i})(r_i(a,(p_{X_i} \circ \tilde{g})(y))) = h_i(r_i(a,g_i(y)))$ для любых $a \in A_i$, $y \in Y$. Так как $(f_i,g_i,h_i)$ --- преобразование Чу категории $Chu_{\widetilde{Set}}$, то $h_i(r_i(a,g_i(y))) = t(f_i(a),y)$ для любых $a \in A_i$, $y \in Y$. Поскольку $\tilde{f}(a) = f_i(a)$, то $t(f_i(a),y) = t(\tilde{f}(a),y)$ для любых $a \in A_i$, $y \in Y$. Таким образом, для любых $a \in \coprod_{i \in I} A_i$, $y \in Y$ имеет место равенство $\tilde{h}(r(a,\tilde{g}(y))) = t(\tilde{f}(a),y)$.
\end{proof}

\begin{theorem}\label{iso}
    Обекты $r: A \times X \to D$ и $\tilde{r}: A \times X \to r(A \times X)$ категории $Chu_{\widetilde{Set}}$, где $\tilde{r}(a,x) = r(a,x)$ для любых $a \in A$, $x \in X$, изоморфны.
\end{theorem}
\begin{proof}
    Построим морфизм $(f,g,h): r \to \tilde{r}$ следующим образом: $f = id_A$, $g = id_X$, $h(d) = 
    \begin{cases}
        r(a,x),& \text{если } d \in r(A \times X),\\
        \star,& \text{иначе}.
    \end{cases}$
    Из равенства $h(r(a,x)) = r(a,x)$ для любых $a \in A$, $x \in X$ следует, что $(f,g,h)$ --- преобразование Чу категории $Chu_{\widetilde{Set}}$. Построим морфизм $(\tilde{f},\tilde{g},\tilde{h}): \tilde{r} \to r$ следующим образом: $\tilde{f} = id_A$, $\tilde{g} = id_X$, $\tilde{h}$ --- естественное вложение. Из равенства $\tilde{h}(r(a,x)) = r(a,x)$ для любых $a \in A$, $x \in X$ следует, что $(\tilde{f},\tilde{g},\tilde{h})$ --- преобразование Чу категории $Chu_{\widetilde{Set}}$.

    Поскольку $(\tilde{f},\tilde{g},\tilde{h}) \circ (f,g,h) = (1_A,1_X,\tilde{h} \circ h)$ и для любого $d \in D$ если $d = r(a,x)$, то $\tilde{h}(d) = h(d) = r(a,x)$, то $(\tilde{f},\tilde{g},\tilde{h}) \circ (f,g,h) = 1_r$. Поскольку $(f,g,h) \circ (\tilde{f},\tilde{g},\tilde{h}) = (1_A,1_X,h \circ \tilde{h})$ и для любого $d \in r(A \times X)$ если $d = r(a,x)$, то $h(d) = \tilde{h}(d) = r(a,x)$, то $(f,g,h) \circ (\tilde{f},\tilde{g},\tilde{h}) = 1_{\tilde{r}}$.
\end{proof}

\begin{theorem}\label{product}
    Пусть $r_i: A_i \times X_i \to D_i$ --- объекты категории $Chu_{\widetilde{Set}}$, $i \in I$. Объект $r: \prod_{i \in I} A_i \times \coprod_{i \in I} X_i \to \prod_{i \in I} r_i(A_i \times X_i)$ категории $Chu_{\widetilde{Set}}$ вместе с семейством $\{(p_{A_i},q_{X_i},p_i): r \to r_i \mid i \in I\}$ преобразований Чу категории $Chu_{\widetilde{Set}}$, где $\tilde{p}_j(r(a,x_i)) = 
    \begin{cases}
        r_i(p_{A_i}(a),x_i),& \text{если } i = j,\\
        \star,& \text{если } i \ne j,
    \end{cases}$
    $\tilde{p}_i: \prod_{i \in I} r_i(A_i \times X_i) \to r_i(A_i \times X_i)$ --- каноническая проекция, $p_{A_i}: \prod_{i \in I} A_i \to A_i$ --- каноническая проекция, $q_{X_i}: X_i \to \coprod_{i \in I} X_i$ --- естественное вложение, $p_i = m_i \circ \tilde{p}_i$, $m_i: r_i(A_i \times X_i) \to D_i$ --- естественное вложение, для любых $a \in \prod_{i \in I} A_i$, $x_i \in X_i$, $i,j \in I$, является произведением объектов $r_i$, $i \in I$, категории $Chu_{\widetilde{Set}}$.
\end{theorem}
\begin{proof}
    Пусть условия теоремы выполнены. Поскольку $p_i(r(a,q_{X_i}(x_i))) = m_i(\tilde{p}_i(r(a,q_{X_i}(x_i)))) = \tilde{p}_i(r(a,q_{X_i}(x_i))) = r_i(p_{A_i}(a),x_i)$ для любых $a \in \prod_{i \in I} A_i$, $x_i \in X_i$, $i \in I$, то $(p_{A_i},q_{X_i},p_i)$ --- преобразование Чу категории $Chu_{\widetilde{Set}}$ для любого $i \in I$.

    Пусть $t: B \times Y \to D$ --- объект категории $Chu_{\widetilde{Set}}$, $(f_i,g_i,h_i): t \to r_i$ --- преобразование Чу категории $Chu_{\widetilde{Set}}$ для любого $i \in I$. Для любого $h_i: D \to D_i$, $i \in I$, построим морфизм $\tilde{h}_i: D \to r_i(A_i \times X_i)$ категории $Set_\star$ следующим образом: $\tilde{h}_i(d) = 
    \begin{cases}
        h_i(d),& \text{если } d = t(b,g_i(x_i)) \text{ для некоторых } b \in B, x_i \in X_i,\\
        \star,& \text{иначе},
    \end{cases}$
    для любого $d \in D$. Поскольку $(f_i,g_i,h_i)$ --- преобразование Чу категории $Chu_{\widetilde{Set}}$ для любого $i \in I$, то $m_i(\tilde{h}_i(t(b,g_i(x_i)))) = h_i(t(b,g_i(x_i))) = r_i(f_i(b),x_i)$ для любых $b \in B$, $x_i \in X_i$. Следовательно, $(f_i,g_i,m_i \circ \tilde{h}_i): t \to r_i$ --- преобразование Чу категории $Chu_{\widetilde{Set}}$ для любого $i \in I$.

    Так как $\prod_{i \in I} A_i$ и $\prod_{i \in I} r_i(A_i \times X_i)$ --- произведения $A_i$, $ i \in I$, и $r_i(A_i \times X_i)$, $i \in I$, соответственно, а $\coprod_{i \in I} X_i$ --- копроизведение $X_i$, $i \in I$, в категории $Set_{\star}$, то существуют единственные морфизмы $\tilde{f}: B \to \prod_{i \in I} A_i$, $\tilde{g}: \coprod_{i \in I} X_i \to Y$, $\tilde{h}: D \to \prod_{i \in I} r_i(A_i \times X_i)$ категории $Set_{\star}$ такие, что $f_i = p_{A_i} \circ \tilde{f}$, $g_i = \tilde{g} \circ q_{X_i}$ и $\tilde{h}_i = \tilde{p}_i \circ \tilde{h}$ для любого $i \in I$, то есть $p_{A_i}(\tilde{f}(b)) = f_i(b)$, $\tilde{g}(x_i) = g_i(x_i)$ и $\tilde{p}_i(\tilde{h}(d)) = \tilde{h}_i(d)$ для любых $b \in  B$, $x_i \in X_i$, $d \in D$, $i \in I$.

    Покажем, что $(\tilde{f},\tilde{g},\tilde{h})$ --- преобразование Чу категории $Chu_{\widetilde{Set}}$, то есть для любых $b \in B$, $x_i \in X_i$, $i \in I$, имеет место равенство $\tilde{h}(t(b,\tilde{g}(x_i))) = r(\tilde{f}(b),x_i)$. Поскольку $\prod_{i \in I} r_i(A_i \times X_i)$ вместе с семейством $\{\tilde{p}_i: \prod_{i \in I} r_i(A_i \times X_i) \to r_i(A_i \times X_i) \mid i \in I\}$ морфизмов категории $Set_\star$ является произведением объектов $r_i(A_i \times X_i)$, $i \in I$, то для любых $b \in B$, $x \in X_i$ равенство $\tilde{h}(t(b,\tilde{g}(x_i))) = r(\tilde{f}(b),x_i)$ равносильно равенству $\tilde{p}_j(\tilde{h}(t(b,\tilde{g}(x_i)))) = \tilde{p}_j(r(\tilde{f}(b),x_i))$ для любых $b \in B$, $x_i \in I$, $i,j \in I$. Тогда из равенств
    \begin{multline*}
        \tilde{p}_j(\tilde{h}(t(b,\tilde{g}(x_i)))) = \tilde{h}_j(t(b,g_i(x_i))) =\\= 
        \begin{cases}
            h_i(t(b,g_i(x_i))) = r_i(f_i(b),x_i) = r_j(f_j(b),x_i),& \text{если } i = j\\
            \star,& \text{если } i \ne j
        \end{cases} =\\=
        \tilde{p}_j(r(\tilde{f}(b),x_i))
    \end{multline*}
    для любых $b \in B$, $x_i \in X_i$, $i,j \in I$ следует, что $(\tilde{f},\tilde{g},\tilde{h})$ --- преобразование Чу категории $Chu_{\widetilde{Set}}$.

    Покажем, что $(p_{A_i},q_{X_i},p_i) \circ (\tilde{f},\tilde{g},\tilde{h}) = (f_i,g_i,h_i)$ для любого $i \in I$, то есть $p_{A_i} \circ \tilde{f} = f_i$, $\tilde{g} \circ q_{X_i} = g_i$ и для любого $d \in D$ если $d = t(b,g_i(x_i))$, то $(p_i \circ \tilde{h})(d) = h_i(d) = r_i(f_i(b),x_i)$. Из определения $\tilde{f}$ и $\tilde{g}$ следует, что $p_{A_i} \circ \tilde{f} = f_i$ и $\tilde{g} \circ q_{X_i} = g_i$ для любого $i \in I$. Пусть $d \in D$ и $d = t(b,g_i(x_i))$ для некоторых $b \in B$, $x_i \in X_i$, $i \in I$. Тогда
    \begin{multline*}
        (p_i \circ \tilde{h})(d) = (m_i \circ \tilde{p}_i \circ \tilde{h})(d) = (m_i \circ \tilde{h}_i)(d) = m_i(\tilde{h}_i(d)) = \tilde{h}_i(d) =\\=
        \tilde{h}_i(t(b,g_i(x_i))) = h_i(t(b,g_i(x_i))) = h_i(d) = r_i(f_i(b),x_i).
    \end{multline*}
    Следовательно, $(p_{A_i},q_{X_i},p_i) \circ (\tilde{f},
    \tilde{g},\tilde{h}) = (f_i,g_i,h_i)$ для любого $i \in I$.

    Покажем, что $(\tilde{f},\tilde{g},\tilde{h})$ --- единственный морфизм такой, что $(p_{A_i},q_{X_i},p_i) \circ (\tilde{f},\tilde{g},\tilde{h}) = (f_i,g_i,h_i)$ для любого $i \in I$. Пусть $(f',g',h'): t \to r$ --- преобразование Чу категории $Chu_{\widetilde{Set}}$ такое, что $(p_{A_i},q_{X_i},p_i) \circ (f',g',h') = (f_i,g_i,h_i)$ для любого $i \in I$, то есть $p_{A_i} \circ f' = f_i$, $g' \circ q_{X_i} = g_i$ и для любого $d \in D$ если $d = t(b,g_i(x_i))$, то $(p_i \circ h')(d) = h_i(d) = r_i(f_i(b),x_i)$ для любого $i \in I$. Из определения $\tilde{f}$ и $\tilde{g}$ следует, что $f' = \tilde{f}$ и $g' = \tilde{g}$. Осталось показать, что $(\tilde{f},\tilde{g},h') = (\tilde{f},\tilde{g},\tilde{h})$, то есть для любого $d \in D$ если $d = t(b,g_i(x_i))$, то $h'(d) = \tilde{h}(d) = r(\tilde{f},x_i)$. Поскольку $\prod_{i \in I} r_i(A_i \times X_i)$ вместе с семейством $\{\tilde{p}_i: \prod_{i \in I} r_i(A_i \times X_i) \to r_i(A_i \times X_i) \mid i \in I\}$ морфизмов категории $Set_\star$ является произведением объектов $r_i(A_i \times X_i)$, $i \in I$, то для любых $b \in B$, $x \in X_i$ равенства $h'(d) = \tilde{h}(d) = r(\tilde{f}(b),x_i)$ равносильны равенствам $\tilde{p}_j(h'(d)) = \tilde{p}_j(\tilde{h}(d)) = \tilde{p}_j(r(\tilde{f}(b),x_i))$ для любых $b \in B$, $x_i \in I$, $i,j \in I$.  Пусть $d \in D$ и $d = t(b,g_i(x_i))$. Тогда если $i = j$, то
    $$
        \tilde{p}_j(h'(d)) = m_i(\tilde{p}_i(h'(d))) = p_i(h'(d)) = p_i(\tilde{h}(d)) = h_i(d) = r_i(f_i(b),x_i),
    $$
    иначе, если $i \ne j$, то
    $$
        \tilde{p}_j(h'(d)) = \tilde{p}_j(h'(t(b,g'(x_i)))) = \tilde{p}_j(r(f'(b),x_i)) = \star = \tilde{p}_j(r(\tilde{f}(b),x_i)) = \tilde{p}_j(\tilde{h}(t(b,\tilde{g}(x_i)))).
    $$
    Следовательно, $(\tilde{f},\tilde{g},\tilde{h}) = (f',g',h')$ в категории $Chu_{\widetilde{Set}}$.
\end{proof}

\section*{Категория $Chu(\GAct_{\star})$}

\begin{theorem}\label{epimorphism-gact}
    Пусть $r: A \otimes X \to D$ и $t: B \otimes Y \to C$ --- объекты категории $Chu_{\widetilde{\GAct}}$. Преобразование $(f,g,h): r \to t$ категории $Chu_{\widetilde{\GAct}}$ является эпиморфизмом тогда и только тогда, когда $f: A \to B$ --- эпиморфизм, $g: Y \to X$ --- мономорфизм категории $\GAct_{\star}$.
\end{theorem}
\begin{proof}
    \textbf{Необходимость.} Пусть $(f,g,h): r \to t$ --- эпиморфизм категории $Chu_{\widetilde{\GAct}}$.

    Покажем, что $f$ --- эпиморфизм категории $\GAct_{\star}$. Предположим, что $B_1 \ne B$, где $B_1 = f(A)$. Через $B_0$ обозначим фактор-полигон полигона $B$ по конгруэнции Риса $\rho_{D_1}$. Определим объект $w: (B_0 \times B) \otimes Y \to C$ и морфизмы $(f_1,1_Y,1_C), (f_2,1_Y,1_C): t \to w$ категории $Chu_{\widetilde{\GAct}}$ следующим образом: $w((b_0,b) \otimes y) = t(b \otimes y)$, $f_1(b) = (B_1,b)$, $f_2(b) = (b/\rho_{B_1},b)$ для любых $b \in B$, $b_0 \in B_0$, $y \in Y$. Корректность определения морфизмов $(f_1,1_Y,1_C)$, $(f_2,1_Y,1_C)$ следует из равенств:
    $$
        t(b \otimes y) = w(f_1(b) \otimes y) = w(f_2(b) \otimes y),
    $$
    где $b \in B$, $y \in Y$. Если $a \in A$, то $f(a) \in B_1$ и $f_1(f(a)) = f_2(f(a))$, то есть $f_1 \circ f = f_2 \circ f$. Тогда $(f_1,1_Y,1_C) \circ (f,g,h) = (f_2,1_Y,1_C) \circ (f,g,h)$. Поскольку $(f,g,h)$ --- эпиморфизм категории $Chu_{\widetilde{\GAct}}$, то $f_1 = f_2$. Противоречие.

    Покажем, что $g$ --- мономорфизм категории $\GAct_{\star}$. Предположим, что существуют различные $y_1, y_2 \in Y$ такие, что $g(y_1) = g(y_2)$. Пусть $b \in B$. Покажем, что $t(b \otimes y_1) = t(b \otimes y_2)$. Так как $f$ --- эпиморфизм, то существует $a \in A$ такой, что $f(a) = b$. Тогда
    $$
        t(b \otimes y_1) = t(f(a) \otimes y_1) = h(r(a \otimes g(y_1))) = h(r(a \otimes g(y_2))) = t(f(a) \otimes y_2) = t(b \otimes y_2).
    $$
    Определим объект $w: B \otimes (G \sqcup \{\star\}) \to C$ и морфизмы $(1_B,g_1,1_C), (1_B,g_2,1_C): t \to w$ категории $Chu_{\widetilde{\GAct}}$ следующим образом: $w(b \otimes \star) = \star$, $w(b \otimes s) = st(b \otimes y_1)$, $g_1(\star) = g_2(\star) = \star$, $g_1(s) = sy_1$, $g_2(s) = sy_2$ для любых $b \in B$, $s \in G$. Корректность определения морфизмов $(1_B,g_1,1_C), (1_B,g_2,1_C)$ следует из равенств:
    $$
        w(b \otimes \star) = t(b \otimes g_1(\star)) = t(b \otimes g_2(\star)) = t(b \otimes \star) = \star,
    $$
    $$
        w(b \otimes s) = t(b \otimes g_1(s)) = st(b \otimes y_1) = st(b \otimes y_2) = t(b \otimes g_2(s))
    $$
    для любого $b \in B$, $s \in G$. Так как
    $$
        (g \circ g_1)(\star) = g(g_1(\star)) = \star = g(g_2(\star)) = (g \circ g_2)(\star),
    $$
    $$
        (g \circ g_1)(s) = g(g_1(s)) = sg(y_1) = sg(y_2) = g(g_2(s)) = (g \circ g_2)(s),
    $$
    то $(1_B,g_1,1_C) \circ (f,g,h) = (1_B,g_2,1_C) \circ (f,g,h)$. Поскольку $(f,g,h)$ --- эпиморфизм категории $Chu_{\widetilde{\GAct}}$, то $g_1 = g_2$. Противоречие.

    \textbf{Достаточность.} Пусть $(f,g,h): r \to t$ --- преобразование Чу категории $Chu_{\widetilde{\GAct}}$, где $f$ --- эпиморфизм, $g$ --- мономорфизм категории $\GAct_{\star}$. Предположим, что $(f_1,g_1,h_1), (f_2,g_2,h_2): t \to w$ --- преобразования Чу категории $Chu_{\widetilde{\GAct}}$ такие, что $(f_1,g_1,h_1) \circ (f,g,h) = (f_2,g_2,h_2) \circ (f,g,h)$, где $w: E \otimes Z \to P$ --- пространство Чу категории $Chu_{\widetilde{\GAct}}$. Тогда $f_1 \circ f = f_2 \circ f$, $g \circ g_1 = g \circ g_2$ и для любого $d \in D$ если $d = r(a \otimes (g \circ g_1)(z)) = r(a \otimes (g \circ g_2)(z))$, то $(h_1 \circ h)(d) = (h_2 \circ h)(d) = w((f_1 \circ f)(a) \otimes z) = w((f_2 \circ f)(a) \otimes z)$. Покажем, что для любого $c \in C$ если $c = t(b \otimes g_1(z))$, то $h_1(c) = h_2(c) = w(f_1(b) \otimes z)$. Пусть $c \in C$ и $c = t(b \otimes g_1(z))$ для некоторых $b \in B$, $z \in Z$. Поскольку $f$ --- эпиморфизм, $g$ --- мономорфизм категории $\GAct_{\star}$, то $f_1 = f_2$, $g_1 = g_2$. Так как $f$ --- эпиморфизм, то $b = f(a)$ для некоторого $a \in A$. Тогда 
    \begin{multline*}
        h_1(c) = h_1(t(b \otimes g_1(z))) = h_1(t(f(a) \otimes g_1(z))) = w(f_1(f(a)) \otimes z) =\\=
        w(f_2(f(a)) \otimes z) = h_2(t(f(a) \otimes g_2(z))) = h_1(t(b \otimes g_1(z))) = h_2(c).
    \end{multline*}
    Следовательно, $(f_1,g_1,h_1) = (f_2,g_2,h_2)$ в категории $Chu_{\widetilde{\GAct}}$ и преобразование $(f,g,h): r \to t$ является эпиморфизмом категории $Chu_{\widetilde{\GAct}}$.
\end{proof}

\begin{theorem}\label{monomorphism-gact}
    Пусть $r: A \otimes X \to D$ и $t: B \otimes Y \to C$ --- объекты категории $Chu_{\widetilde{\GAct}}$. Преобразование $(f,g,h): r \to t$ категории $Chu_{\widetilde{\GAct}}$ является мономорфизмом тогда и только тогда, когда $f: A \to B$ --- мономорфизм, $g: Y \to X$ --- эпиморфизм категории $\GAct^{\star}$.
\end{theorem}
\begin{proof}
    \textbf{Необходимость.} Пусть $(f,g,h): r \to t$ --- мономорфизм категории $Chu_{\widetilde{\GAct}}$.

    Покажем, что $g$ --- эпиморфизм категории $\GAct_{\star}$. Предположим, что $X_1 \ne X$, где $X_1 = g(Y)$. Через $X_0$ обозначим фактор-полигон полигона $X$ по конгруэнции Риса $\rho_{X_1}$. Определим объект $w: A \otimes (X_0 \times X) \to D$ и морфизмы $(1_A,g_1,1_D), (1_A,g_2,1_D): w \to r$ категории $Chu_{\widetilde{\GAct}}$ следующим образом: $w(a \otimes (x_0,x)) = r(a \otimes x)$, $g_1(x) = (X_1,x)$, $g_2(x) = (x/\rho_{X_1}, x)$, для любых $a \in A$, $x \in X$, $x_0 \in X_0$. Из определения объекта $w$ категории $Chu_{\widetilde{\GAct}}$ следует равенство
    $$
        w(a \otimes g_1(x)) = w(a \otimes g_2(x)) = r(a \otimes x),
    $$ 
    где $a \in A$, $x \in X$, что доказывает корректность определения морфизмов $(1_A,g_1,1_D)$, $(1_A,g_2,1_D)$. Если $y \in Y$, то $g(y) \in X_1$ и $g_1(g(y)) = g_2(g(y))$, то есть $g_1 \circ g = g_2 \circ g$. Тогда $(f,g,h) \circ (1_A,g_1,1_D) = (f,g,h) \circ (1_A,g_2,1_D)$. Поскольку $(f,g,h)$ --- мономорфизм категории $Chu_{\widetilde{\GAct}}$, то $g_1 = g_2$. Противоречие.

    Покажем, что $f$ --- мономорфизм категории $\GAct_{\star}$. Предположим, что существуют различные $a_1, a_2 \in A$ такие, что $f(a_1) = f(a_2)$. Определим объект $w: (\{\star\} \sqcup G \sqcup G') \otimes X \to D'$, где $G'$ --- копия $G$, $D' = r(\{a_1,a_2\} \otimes X)$ и морфизмы $(f_1,1_X,1_D), (f_2,1_X,1_D): w \to r$ категории $Chu_{\widetilde{\GAct}}$ следующим образом $w(\star \otimes x) = \star$, $w(s \otimes x) = sr(a_1 \otimes x)$, $w(s' \otimes x) = s'r(a_2 \otimes x)$, $f_1(\star) = f_2(\star) = \star$, $f_1(s) = sa_1$, $f_1(s') = s'a_2$, $f_2(s) = sa_2$, $f_2(s') = s'a_1$, $h_1(r(a_1 \otimes x)) = h_2(r(a_2 \otimes x)) = r(a_1 \otimes x)$, $h_1(r(a_2 \otimes x)) = h_2(r(a_1 \otimes x)) = r(a_2 \otimes x)$ для любых $x \in X$, $s \in G$, $s' \in G'$. Корректность определения морфизмов $(f_1,1_X,h_1), (f_2,1_X,h_2)$ следует из равенств
    $$
        h_1(w(\star \otimes x)) = h_2(w(\star \otimes x)) = \star = r(f_1(\star) \otimes x) = r(f_2(\star) \otimes x),
    $$
    $$
        h_1(w(s \otimes x)) = h_1(sr(a_1 \otimes x)) = r(sa_1 \otimes x) = r(f_1(s) \otimes x),
    $$
    $$
        h_2(w(s \otimes x)) = h_2(sr(a_1 \otimes x)) = r(sa_2 \otimes x) = r(f_2(s) \otimes x),
    $$
    $$
        h_1(w(s' \otimes x)) = h_1(s'r(a_2 \otimes x)) = r(s'a_2 \otimes x) = r(f_1(s') \otimes x),
    $$
    $$
        h_2(w(s' \otimes x)) = h_2(s'r(a_2 \otimes x)) = r(s'a_1 \otimes x) = r(f_2(s') \otimes x)
    $$
    для любых $x \in X$, $s \in G$, $s' \in G'$.
    Так как $f(f_1(s)) = f(sa_1) = sf(a_1) = sf(a_2) = f(sa_2) = f(f_2(s))$, $f(f_1(s')) = f(s'a_2) = s'f(a_2) = s'f(a_1) = f(s'a_1) = f(f_2(s'))$ и $(f \circ f_1)(\star) = \star = (f \circ f_2)(\star)$ для любых $s \in G$, $s' \in G'$, то $f \circ f_1 = f \circ f_2$. Из того, что $g$ --- эпиморфизм, следует, что $x = g(y)$ для некоторого $y \in Y$ для любого $x \in X$. Тогда из равенств
    $$
        (h \circ h_1)(w(\star \otimes x)) = \star = (h \circ h_2)(w(\star \otimes x)),
    $$
    \begin{multline*}
        (h \circ h_1)(w(s \otimes x)) = h(sh_1(r(a_1 \otimes x))) = sh(r(a_1 \otimes g(y))) = st(f(f_1(s)) \otimes x)  =\\= 
        st(f(a_1) \otimes y) = st(f(a_2) \otimes y) = sh(r(a_2 \otimes g(y))) = h(sh_2(r(a_1 \otimes x))) = (h \circ h_2)(w(s \otimes x)),
    \end{multline*}
    \begin{multline*}
        (h \circ h_1)(w(s' \otimes x)) = h(s'h_1(r(a_2 \otimes x))) = s'h(r(a_2 \otimes g(y))) = s't(f(f_1(s')) \otimes x)  =\\= 
        s't(f(a_2) \otimes y) = s't(f(a_1) \otimes y) = s'h(r(a_1 \otimes g(y))) = h(s'h_2(r(a_2 \otimes x))) = (h \circ h_2)(w(s' \otimes x))
    \end{multline*}
    для любых $x \in X$, $s \in G$, $s' \in G'$ следует, что для любого $d' \in D'$ если $d' = w(a \otimes x)$, то $(h \circ h_1)(d') = (h \circ h_2)(d') = t((f \circ f_1)(a) \otimes x)$. Следовательно, $(f,g,h) \circ (f_1,1_X,h_1) = (f,g,h) \circ (f_2,1_X,h_2)$. Поскольку $(f,g,h)$ --- мономорфизм категории $Chu_{\widetilde{\GAct}}$, то $f_1 = f_2$, то есть $a_1 = a_2$. Противоречие.

    \textbf{Достаточность.} Пусть $(f,g,h): r \to t$ --- преобразование Чу категории $Chu_{\widetilde{\GAct}}$, где $f$ --- мономорфизм, $g$ --- эпиморфизм категории $\GAct_{\star}$. Предположим, что $(f_1,g_1,h_1), (f_2,g_2,h_2): w \to r$ --- преобразования Чу категории $Chu_{\widetilde{\GAct}}$ такие, что $(f,g,h) \circ (f_1,g_1,h_1) = (f,g,h) \circ (f_2,g_2,h_2)$, где $w: E \otimes Z \to P$ --- пространство Чу категории $Chu_{\widetilde{\GAct}}$. Тогда $f \circ f_1 = f \circ f_2$, $g_1 \circ g = g_1 \circ g$ и для любого $p \in P$ если $p = w(e \otimes (g_1 \circ g)(y)) = w(e \otimes (g_2 \circ g)(y))$, то $(h \circ h_1)(p) = (h \circ h_2)(p) = t((f \circ f_1)(e) \otimes y) = t((f \circ f_2)(e) \otimes y)$. Покажем, что для любого $p \in P$ если $p = w(e \otimes g_1(x))$, то $h_1(p) = h_2(p) = r(f_1(e) \otimes x)$. Пусть $p \in P$ и $p = w(e \otimes g_1(x))$ для некоторых $e \in E$, $x \in X$. Поскольку $f$ --- мономорфизм, $g$ --- эпиморфизм категории $\GAct_{\star}$, то $f_1 = f_2$, $g_1 = g_2$. Тогда 
    $$
        h_1(p) = h_1(w(e \otimes g_1(x))) = r(f_1(e) \otimes x) = r(f_2(e) \otimes x) = h_2(w(e \otimes g_2(x))) = h_2(p).
    $$
    Следовательно, $(f_1,g_1,h_1) = (f_2,g_2,h_2)$ в категории $Chu_{\widetilde{\GAct}}$ и преобразование $(f,g,h): r \to t$ является мономорфизмом категории $Chu_{\widetilde{\GAct}}$.
\end{proof}

\end{document}

\begin{theorem}[критерий существования уравнителя]
    Пусть $r: A \times X \to D_1$, $s: B \times Y \to D_2$ --- пространства Чу категории $Chu_{\widetilde{Set}}$, $(f_1,g_1,h_1), (f_2,g_2,h_2): r \to s$ --- преобразования Чу категории $Chu_{\widetilde{Set}}$. Уравнитель преобразований Чу $(f_1,g_1,h_1)$ и $(f_2,g_2,h_2)$ существует тогда и только тогда существует элемент $a \in A$ такой, что
    \begin{enumerate}
        \item $f_1(a) = f_2(a)$;
        \item $r(a,g_1(y)) = r(a,g_2(y))$ для любого $y \in Y$;
        % \item $h_1(r(a,x)) = h_2(r(a,x))$ для любого $x \in X$.
    \end{enumerate}
\end{theorem}
\begin{proof}
    \textbf{Необходимость.} Пусть $t: C \times Z \to D_0$ --- пространство Чу категории $Chu_{\widetilde{Set}}$, $(f,g,h): t \to r$ --- уравнитель преобразований Чу $(f_1,g_1,h_1)$ и $(f_2,g_2,h_2)$, то есть $(f_1,g_1,h_1) \circ (f,g,h) = (f_2,g_2,h_2) \circ (f,g,h)$. В этом случае $f_1 \circ f = f_2 \circ f$, $g \circ g_1 = g \circ g_2$ и для любого $d_0 \in D_0$ если $d_0 = t(c, (g \circ g_1)(y))$, то $(h_1 \circ h)(d_0) = (h_2 \circ h)(d_0) = s((f_1 \circ f)(c),y)$. Определим $a \in A$ как произвольный элемент образа $f(C)$, то есть $a = f(c)$ для некоторого $c \in C$. Проверим выполнение условий 1),2) теоремы. Выполнение условия 1) следует из равенств
    $$
        f_1(a) = f_1(f(c)) = f_2(f(c)) = f_2(a).
    $$
    Выполнение условия 2) следует из равенств
    \begin{multline*}
        r(a,g_1(y)) = r(f(c),g_1(y)) = h(t(c,g(g_1(y)))) =\\= 
        h(t(c,g(g_1(y)))) = r(f(c),g_2(y)) = r(a,g_2(y))
    \end{multline*}
    для любых $y \in Y$. 
    % Выполнение условия 3) следует из равенств
    % \begin{multline*}
    %     h_1(r(a,x)) = h_1(r(f(c),x)) = h_1(h(t(c,g(x)))) =\\=
    %     h_2(h(t(c,g(x)))) = h_2(r(f(c),x)) = h_2(r(a,x)).
    % \end{multline*}

    \textbf{Достаточность.} Пусть существует элемент $a \in A$ такой, что выполнены условия 1),2) теоремы. Определим следующие множества:
    $$
        C_1 = \{c \in A \mid f_1(c) = f_2(c)\},
        C_2 = \{c \in A \mid \forall y \in Y\ r(c,g_1(y)) = r(c,g_2(y))\}.
    $$
    Определим пространство Чу $t: C \times X/\nu(g_1,g_2) \to D_1$, где $C = C_1 \cap C_2$, $t(c,\overline{x}) = r(c,x)$ для любых $c \in C, \overline{x} \in X/\nu(g_1,g_2)$. Так как $a \in C$, то $C \ne \varnothing$. Поскольку для любого $c \in C \subseteq C_2$, то $t(c,\overline{g_1(y)}) = r(c,g_1(y)) = r(c,g_2(y)) = t(c,\overline{g_2(y)})$, то есть значение $t(c,\overline{x})$ не зависит от выбора представителя $\overline{x} \in X/\nu(g_1,g_2)$. Определим уравнитель преобразований Чу $(f_1,g_1,h_1)$ и $(f_2,g_2,h_2)$ как морфизм $(f,g,h): t \to r$ следующим образом: $f(c) = c$, $g(x) = \overline{x}$, $h(d) = d$ для любых $c \in C$, $x \in X$, $d \in D$. Корректность определения $(f,g,h)$ следует из равенств $h(t(c,g(x))) = h(r(c,x)) = r(c,x) = r(f(c),x)$ для любых $c \in C$, $x \in X$. Из того, что для любого $c \in C \subseteq C_1$ следует, что $f_1 \circ f = f_2 \circ f$. Так как $g \circ g_1 = g \circ g_2$ и для любого $d \in D$ если $d = t(c,(g \circ g_1)(y))$, то $(h_1 \circ h)(d) = (h_2 \circ h)(d) = s((f_1 \circ f)(c),y)$, следовательно $h_1(d) = h_2(d)$. Тогда $(f_1, g_1, h_1) \circ (f, g, h) = (f2, g_2, h_2) \circ (f, g, h)$, то есть $(f,g,h)$ -- уравнитель $(f_1,g_1,h_1)$ и $(f_2,g_2,h_2)$.
\end{proof}

\begin{theorem}[существование коуравнителя]
    Пусть $r: A \times X \to D_1$ и $s: B \times Y \to D_2$ --- объекты категории $Chu_{\widetilde{Set}}$ и $(f_1,g_1,h_1), (f_2,g_2,h_2): r \to s$ --- преобразования категории $Chu_{\widetilde{Set}}$. Тогда коуравнитель морфизмов $(f_1,g_1,h_1), (f_2,g_2,h_2)$ --- это пространство Чу $t: Q \times E \to W$, где $Q = B/\nu(f_1,f_2)$, $W = D/\nu(h_1,h_2)$, $E = \{y \in Y \mid g_1(y) = g_2(y)\}$, $t(b/\nu(f_1,f_2),y) = s(b,y)/\nu(h_1,h_2)$ для любых $b \in B$, $y \in E$, с морфизмом $(f,g,h): s \to t$, где $f,h$ --- канонические эпиморфизмы, $g$ --- естественное вложение.
\end{theorem}
\begin{proof}
    Пусть условия теоремы выполнены.

    Введем обозначения: $\overline{b} = b/\nu(f_1,f_2)$, $\overline{d} = d/\nu(h_1,h_2)$ для любых $b \in B$, $d \in D$. Корректность определения $t$ следует из равенств $h_1(r(a,g_1(y))) = s(f_1(a),y)$, $h_2(r(a,g_2(y))) = s(f_2(a),y)$ для любых $a \in A$, $y \in E$.

    Покажем, что $(f,g,h): s \to t$ такое, что $f(b) = \overline{b}$, $g = 1_E$ и $h(d_2) = \overline{d_2}$ для любых $b \in B$, $d_2 \in D_2$, является преобразованием Чу категории $Chu_{\widetilde{Set}}$.

    Пусть $t': Q' \times E' \to W'$ объект категории $Chu_{\widetilde{Set}}$ и $(f',g',h'): s \to t$ преобразование Чу категории $Chu_{\widetilde{Set}}$ такое, что $(f',g',h') \circ (f_1,g_1,h_1) = (f',g',h') \circ (f_2,g_2,h_2)$. Тогда $f' \circ f_1 = f' \circ f_2$, $g_1 \circ g' = g_2 \circ g'$ и для любого ... ПРО h' НАПИСАТЬ. Так как $Q,W$ являются коуравнителями и $E$ является уравнителем в категории $Set_{\star}$, то существуют единственные морфизмы $u: Q \to Q'$, $w: W \to W'$ категории $Set_{\star}$ такие, что $f' = u \circ f$, ХЗ ТАК ЭТО ИЛИ НЕТ НО $h' = w \circ h$ и существует единственный морфизм $v: E' \to E$ категории $Set_{\star}$ такой, что $g' = g \circ v$. Покажем, что $(u,v,w): t \to t'$ --- преобразование Чу категории $Chu_{\widetilde{Set}}$. Пусть $b \in B$, $e' \in E'$. Так как $u(\overline{b}) = (u \circ f)(b)$ и $f' = u \circ f$, то $t'(u(\overline{b}),e') = t'((u \circ f)(b),e') = t'(f'(b),e')$. Из того, что $(f',g',h')$ --- преобразование Чу категории $Chu_{\widetilde{Set}}$ следует, что $t'(f'(b),e') = h'(s(b,g'(e')))$. С другой стороны, из того, что $\overline{b} = f(b)$ следует, что $w(t(\overline{b},v(e'))) = w(t(f(b),v(e')))$. Так как $(f,g,h)$ --- преобразование Чу категории $Chu_{\widetilde{Set}}$, то $w(t(f(b),v(e'))) = (w \circ h)(s(b,(g \circ v)(e')))$. Поскольку $h' = w \circ h$ и $g' = g \circ v$, то $(w \circ h)(s(b,(g \circ v)(e'))) = h'(s(b,g'(e')))$. Таким образом, $t'(u(\overline{b}),e') = w(t(\overline{b},v(e')))$. Следовательно, $(u,v,w)$ --- преобразование Чу категории $Chu_{\widetilde{Set}}$. Ясно, что преобразование $(u,v,w)$ --- единственное преобразование, удовлетворяющее равенству $(f',g',h') = (u,v,w) \circ (f,g,h)$.
\end{proof}