\documentclass[a4paper,12pt]{article}
\usepackage{framed}
\usepackage[top=2cm, bottom=2cm]{geometry}
\usepackage[utf8]{inputenc}
\usepackage[english,russian]{babel}
\usepackage{amsthm,amssymb,amsfonts,amsmath,cite,enumerate}
\usepackage[all]{xy}

\newtheorem{statement}{Утверждение}
\newtheorem{lemma}{Лемма}
\newtheorem{theorem}{Теорема}
\newtheorem{consequence}{Следствие}
\newtheorem{remark}{Замечание}

\begin{document}

Зафиксируем категорию $C$ и объект $\star \in Ob(C)$. Определим следующую категорию, называемую \textit{пунктированной категорией $C$}:
\begin{itemize}
    \item Объектами категории являются пары $(k,u)$, где $k \in Ob(C)$, $u \in Mor(\star,k)$;
    \item Морфизмом из $(k,u)$ в $(l,v)$ является элемент $f \in Mor(k,l)$ такой, что $f \circ u = v$,
\end{itemize}
где комозиция наследуется из категори $C$.

Пусть $(C,\otimes,\star)$ --- моноидальная, замкнутая, биполная категория с терминальным объектом $\star$ и пусть $C_\star$ --- пунктированная категория $C$. Для объектов $k$ и $l$ из $C_*$ определим их \textit{скрещенное произведение} $k \wedge l$ как объект из $C_\star$, задаваемый следующим расслоенным произведением в $C$:
$$\xymatrix{
    (k \otimes \star) \sqcup (\star \otimes l) \ar[r] \ar[d] & k \otimes l \ar[d]\\
    \star \ar[r] & k \wedge l.
}$$

Везде ниже $(C,\times,\star)$ означает моноидальную замкнутую биполную категорию с терминальным объектом $\star$, объектами $C$ являются алгебраические системы и для любый подсистемы $B$ произвольной алгебраической системы $A \in Ob(C)$ существует подсистема $C$ системы $A$ такая, что $A = B \sqcup C$. Скрещенное произведение объектов $(A,u), (B,v) \in Ob(C_\star)$ изоморфино объекту $((A \sqcup B)/\Theta, w)$, где $\Theta = \{(a,\star) \mid a \in A\} \cup \{(\star,b) \mid a \in A\} \cup \{(c,c) \mid c \in A \cup B\}$ и $w(\star)/\Theta = \{(a,\star) \mid a \in A\} \cup \{(\star,b) \mid b \in B\}$. В пунктированной категории $C_\star$ объекты $(A,u)$ будет обозначаться через $A_\star$ или просто $A$ и $\star$ будет рассматриваться как элемент $A$.

ДОПИСАТЬ ОПРЕДЕЛЕНИЯ

Ниже будем рассматривать пунктированную категорию множеств $Set_\star$. Объектом $\star$ в этой категории является одноэлементное множество $\{\star\}$.

\begin{theorem}\label{epimorphism}
    Пусть $r: A \times X \to D$ и $t: B \times Y \to C$ --- объекты категории $Chu_{\widetilde{Set}}$. Преобразование $(f,g,h): r \to t$ категории $Chu_{\widetilde{Set}}$ является эпиморфизмом тогда и только тогда, когда $f: A \to B$ --- эпиморфизм, $g: Y \to X$ --- мономорфизм категории $Set^*$.
\end{theorem}
\begin{proof}
    \textbf{Необходимость.} Пусть $(f,g,h): r \to t$ --- эпиморфизм категории $Chu_{\widetilde{Set}}$.

    Покажем, что $f$ --- эпиморфизм категории $Set^*$. Предположим, что $B_1 \ne B$, где $B_1 = f(A)$. Через $B_0$ обозначим фактормножество множества $B$ по отношению эквивалентности $\sim$ такому, что $b \sim b' \Leftrightarrow b, b' \in B_1$. Определим объект $w: (B_0 \times B) \times Y \to C$ и морфизмы $(f_1,1_Y,1_C), (f_2,1_Y,1_C): t \to w$ категории $Chu_{\widetilde{Set}}$ следующим образом: $w((b_0,b),y) = t(b,y)$, $f_1(b) = (B_1,b)$, $f_2(b) = (b/\sim,b)$ для любых $b \in B$, $b_0 \in B_0$, $y \in Y$. Корректность определения морфизмов $(f_1,1_Y,1_C)$, $(f_2,1_Y,1_C)$ следует из равенств:
    $$
        t(b,y) = w(f_1(b),y) = w(f_2(b),y),
    $$
    где $b \in B$, $y \in Y$. Если $a \in A$, то $f(a) \in B_1$ и $f_1(f(a)) = f_2(f(a))$, то есть $f_1 \circ f = f_2 \circ f$. Тогда $(f_1,1_Y,1_C) \circ (f,g,h) = (f_2,1_Y,1_C) \circ (f,g,h)$. Поскольку $(f,g,h)$ --- эпиморфизм категории $Chu_{\widetilde{Set}}$, то $f_1 = f_2$. Противоречие.

    Покажем, что $g$ --- мономорфизм категории $Set^*$. Предположим, что существуют различные $y_1, y_2 \in Y$ такие, что $g(y_1) = g(y_2)$. Пусть $b \in B$. Покажем, что $t(b,y_1) = t(b,y_2)$. Так как $f$ --- эпиморфизм, то существует $a \in A$ такой, что $f(a) = b$. Тогда
    $$
        t(b,y_1) = t(f(a),y_1) = h(r(a,g(y_1))) = h(r(a,g(y_2))) = t(f(a),y_2) = t(b,y_2).
    $$
    Определим объект $w: B \times \{\star,y_0\} \to C$ и морфизмы $(1_B,g_1,1_C), (1_B,g_2,1_C): t \to w$ категории $Chu_{\widetilde{Set}}$ следующим образом: $w(b,\star) = \star$, $w(b,y_0) = t(b,y_1)$, $g_1(\star) = g_2(\star) = \star$, $g_1(y_0) = y_1$, $g_2(y_0) = y_2$ для любых $b \in B$. Корректность определения морфизмов $(1_B,g_1,1_C), (1_B,g_2,1_C)$ следует из равенств:
    $$
        w(b,\star) = t(b,g_1(\star)) = t(b,g_2(\star)) = t(b,\star) = \star,
    $$
    $$
        w(b,y_0) = t(b,g_1(y_0)) = t(b,y_1) = t(b,y_2) = t(b,g_1(y_0))
    $$
    для любого $b \in B$. Так как
    $$
        (g \circ g_1)(\star) = g(g_1(\star)) = \star = g(g_2(\star)) = (g \circ g_2)(\star),
    $$
    $$
        (g \circ g_1)(y_0) = g(g_1(y_0)) = g(y_1) = g(y_2) = g(g_2(y_0)) = (g \circ g_2)(\star),
    $$
    то $(1_B,g_1,1_C) \circ (f,g,h) = (1_B,g_2,1_C) \circ (f,g,h)$. Поскольку $(f,g,h)$ --- эпиморфизм категории $Chu_{\widetilde{Set}}$, то $g_1 = g_2$. Противоречие.

    \textbf{Достаточность.} Пусть $(f,g,h): r \to t$ --- преобразование категории $Chu_{\widetilde{Set}}$, где $f$ --- эпиморфизм, $g$ --- мономорфизм категории $Set^*$. Предположим, что $(f_1,g_1,h_1), (f_2,g_2,h_2): t \to w$ --- преобразования Чу категории $Chu_{\widetilde{Set}}$ такие, что $(f_1,g_1,h_1) \circ (f,g,h) = (f_2,g_2,h_2) \circ (f,g,h)$, где $w: E \times Z \to P$ --- пространство Чу категории $Chu_{\widetilde{Set}}$. Тогда $f_1 \circ f = f_2 \circ f$, $g \circ g_1 = g \circ g_2$ и для любого $d \in D$ если $d = r(a,(g \circ g_1)(z)) = r(a,(g \circ g_2)(z))$, то $(h_1 \circ h)(d) = (h_2 \circ h)(d) = w((f_1 \circ f)(a), z) = w((f_2 \circ f)(a), z)$. Покажем, что для любого $c \in C$ если $c = t(b,g_1(z))$, то $h_1(c) = h_2(c) = w(f_1(b),z)$. Пусть $c \in C$ и $c = t(b,g_1(z))$ для некоторых $b \in B$, $z \in Z$. Поскольку $f$ --- эпиморфизм, $g$ --- мономорфизм категории $Set^*$, то $f_1 = f_2$, $g_1 = g_2$. Так как $f$ --- эпиморфизм, то $b = f(a)$ для некоторого $a \in A$. Тогда 
    \begin{multline*}
        h_1(c) = h_1(t(b,g_1(z))) = h_1(t(f(a),g_1(z))) = w(f_1(f(a)),z) =\\=
        w(f_2(f(a)),z) = h_2(t(f(a),g_2(z))) = h_1(t(b,g_1(z))) = h_2(c).
    \end{multline*}
    Следовательно, $(f_1,g_1,h_1) = (f_2,g_2,h_2)$ в категории $Chu_{\widetilde{Set}}$ и преобразование $(f,g,h): r \to t$ является эпиморфизмом категории $Chu_{\widetilde{Set}}$.
\end{proof}

\begin{theorem}\label{monomorphism}
    Пусть $r: A \times X \to D$ и $t: B \times Y \to C$ --- объекты категории $Chu_{\widetilde{Set}}$. Преобразование $(f,g,h): r \to t$ категории $Chu_{\widetilde{Set}}$ является мономорфизмом тогда и только тогда, когда $f: A \to B$ --- мономорфизм, $g: Y \to X$ --- эпиморфизм категории $Set^*$.
\end{theorem}
\begin{proof}
    \textbf{Необходимость.} Пусть $(f,g,h): r \to t$ --- мономорфизм категории $Chu_{\widetilde{Set}}$.

    % Покажем, что для любых $d_1,d_2 \in r(A,X)$ если $h(d_1) = h(d_2)$, то  $d_1 = d_2$. Предположим, что существуют различные $d_1,d_2 \in r(A,X)$ такие, что $h(d_1) = h(d_2)$. Поскольку $d_1,d_2 \in r(A,X)$, то $d_1 = r(a_1,x_1)$ и $d_2 = r(a_2,x_2)$ для некоторых $a_1,a_2 \in A$, $x_1,x_2 \in X$. Пусть $h(d_1) = h(d_2)$, то есть $h(r(a_1,x_1)) = h(r(a_2,x_2))$. Так как $g$ --- эпиморфизм, то $r(a_1,x_1) = r(a_1,g(y_1))$ и $r(a_2,x_2) = r(a_2,g(y_2))$ для некоторых $y_1,y_2 \in Y$. Тогда
    % $$
    %     t(f(a_1),y_1) = h(r(a_1,g(y_1))) = h(r(a_1,x_1)) = h(r(a_2,x_2)) = h(r(a_2,g(y_2))) = t(f(a_2),y_2)
    % $$
    
    % Предположем, что существуют различные $a_1,a_2 \in A$ и $x \in X$ такие, что $h(r(a_1,x)) = h(r(a_2,x))$. Определим объект $w: A \times X \to (\star \sqcup r(A,X))$ и морфизмы $(1_A,1_X,h_1),(1_A,1_X,h_2): w \to r$ категории $Chu_{\widetilde{Set}}$ следующим образом: $w(a,x) = r(a,x)$. $h_1(\star) = r(a_1,x)$, $h_2(\star) = r(a_2,x)$, $h_1(r(a,x)) = h_2(r(a,x)) = r(a,x)$ для любых $a \in A$, $x \in X$. Корректность определения морфизмов $(1_A,1_X,h_1)$, $(1_A,1_X,h_2)$ следует из равенств:
    % $$
    %     h_1(w(a,x)) = h_2(w(a,x)) = r(a,x),
    % $$
    % где $a \in A$, $x \in X$. Так как $h(h_1(r(a,x))) = h(h_2(r(a,x)))$ для любых $a \in A$, $x \in X$ и
    % $$
    %     h(h_1(\star)) = h(r(a_1,x)) = h(r(a_2,x)) = h(h_2(\star)),
    % $$
    % то $h \circ h_1 = h \circ h_2$. Следовательно, $(f,g,h) \circ (1_A,1_X,h_1) = (f,g,h) \circ (1_A,1_X,h_2)$. Поскольку $(f,g,h)$ --- мономорфизм категории $Chu_{\widetilde{Set}}$, то $(1_A,1_X,h_1) = (1_A,1_X,h_2)$, то есть для любого $d \in \star \sqcup r(A,X)$ если $d = w(a,x)$, то $h_1(d) = h_2(d) = r(a,x)$. В частности, для $d = \star = w(...,...)$ выполняется $h_1(\star) = r(a_1,x) = r(a_1,x) = h_2(\star) = r(...,...)$. Противоречие.

    Покажем, что $g$ --- эпиморфизм категории $Set^*$. Предположим, что $X_1 \ne X$, где $X_1 = g(Y)$. Через $X_0$ обозначим фактормножество множества $X$ по отношению эквивалентности $\sim$ такому, что $x \sim x' \Leftrightarrow x,x' \in X_1$. Определим объект $w: A \times (X_0 \times X) \to D$ и морфизмы $(1_A,g_1,1_D), (1_A,g_2,1_D): w \to r$ категори $Chu_{\widetilde{Set}}$ следующим образом: $w(a,(x_0,x)) = r(a,x)$, $g_1(x) = (X_1,x)$, $g_2(x) = (x/\sim, x)$, для любых $a \in A$, $x \in X$, $x_0 \in X_0$. Из определения объекта $w$ категории $Chu_{\widetilde{Set}}$ следует равенство
    $$
        w(a,g_1(x)) = w(a,g_2(x)) = r(a,x),
    $$ 
    где $a \in A$, $x \in X$, что доказывает корректность определения морфизмов $(1_A,g_1,1_D)$, $(1_A,g_2,1_D)$. Если $y \in Y$, то $g(y) \in X_1$ и $g_1(g(y)) = g_2(g(y))$, то есть $g_1 \circ g = g_2 \circ g$. Тогда $(f,g,h) \circ (1_A,g_1,1_D) = (f,g,h) \circ (1_A,g_2,1_D)$. Поскольку $(f,g,h)$ --- мономорфизм категории $Chu_{\widetilde{Set}}$, то $g_1 = g_2$. Противоречие.

    Покажем, что $f$ --- мономорфизм категории $Set^*$. Предположим, что существуют различные $a_1, a_2 \in A$ такие, что $f(a_1) = f(a_2)$. Пусть $x \in X$. Покажем, что $r(a_1,x) = r(a_2,x)$. Так как $g$ --- эпиморфизм, то существует $y \in Y$ такой, что $g(y) = x$. Тогда
    $$
        h(r(a_1,x)) = h(r(a_1,g(y))) = t(f(a_1),y) = t(f(a_2),y) = h(r(a_2,g(y))) = h(r(a_2,x)).
    $$
    Определим объект $w: \star \times X \to D$ и морфизмы $(f_1,1_X,1_D), (f_2,1_X,1_D): w \to r$ категории $Chu_{\widetilde{Set}}$ следующим образом $w(\star,x) = r(a_1,x)$, $f_1(\star) = a_1$, $f_2(\star) = a_2$ для любых $x \in X$. ДЛЯ КОРРЕКТНОСТИ ОПРЕДЕЛЕННИЯ МОРФИЗМОВ НУЖНО РАВЕНСТВО $w(\star,x) = r(f_1(\star), x) = r(f_2(\star),x) = r(a_1,x)$, КОТОРОЕ БЫЛО БЫ, ЕСЛИ $h$ БЫЛО МОНОМОРФИЗМОМ. Поскольку
    $$
        (f \circ f_1)(\star) = f(f_1(\star)) = f(a_1) = f(a_2) = f(f_2(\star)) = (f \circ f_2)(\star),
    $$
    то $f \circ f_1 = f \circ f_2$. Следовательно, $(f,g,h) \circ (f_1,1_X,1_D) = (f,g,h) \circ (f_2,1_X,1_D)$. Поскольку $(f,g,h)$ --- мономорфизм категории $Chu_{\widetilde{Set}}$, то $f_1 = f_2$, то есть $a_1 = a_2$. Противоречие.

    \textbf{Достаточность.} Пусть $(f,g,h): r \to t$ --- преобразование категории $Chu_{\widetilde{Set}}$, где $f$ --- мономорфизм, $g$ --- эпиморфизм категории $Set^*$. Предположим, что $(f_1,g_1,h_1), (f_2,g_2,h_2): w \to r$ --- преобразования Чу категории $Chu_{\widetilde{Set}}$ такие, что $(f,g,h) \circ (f_1,g_1,h_1) = (f,g,h) \circ (f_2,g_2,h_2)$, где $w: E \times Z \to P$ --- пространство Чу категории $Chu_{\widetilde{Set}}$. Тогда $f \circ f_1 = f \circ f_2$, $g_1 \circ g = g_1 \circ g$ и для любого $p \in P$ если $p = w(e,(g_1 \circ g)(y)) = w(e,(g_2 \circ g)(y))$, то $(h \circ h_1)(p) = (h \circ h_2)(p) = t((f \circ f_1)(e), y) = t((f \circ f_2)(e), y)$. Покажем, что для любого $p \in P$ если $p = w(e,g_1(x))$, то $h_1(p) = h_2(p) = r(f_1(e),x)$. Пусть $p \in P$ и $p = w(e,g_1(x))$ для некоторых $e \in E$, $x \in X$. Поскольку $f$ --- мономорфизм, $g$ --- эпиморфизм категории $Set^*$, то $f_1 = f_2$, $g_1 = g_2$. Тогда 
    $$
        h_1(p) = h_1(w(e,g_1(x))) = r(f_1(e),x) = r(f_2(e),x) = h_2(w(e,g_2(x))) = h_2(p).
    $$
    Следовательно, $(f_1,g_1,h_1) = (f_2,g_2,h_2)$ в категории $Chu_{\widetilde{Set}}$ и преобразование $(f,g,h): r \to t$ является мономорфизмом категории $Chu_{\widetilde{Set}}$.
\end{proof}

\begin{theorem}[критерий существования уравнителя]
    Пусть $r: A \times X \to D_1$, $s: B \times Y \to D_2$ --- пространства Чу категории $Chu_{\widetilde{Set}}$, $(f_1,g_1,h_1), (f_2,g_2,h_2): r \to s$ --- преобразования Чу категории $Chu_{\widetilde{Set}}$. Уравнитель преобразований Чу $(f_1,g_1,h_1)$ и $(f_2,g_2,h_2)$ существует тогда и только тогда существует элемент $a \in A$ такой, что
    \begin{enumerate}
        \item $f_1(a) = f_2(a)$;
        \item $r(a,g_1(y)) = r(a,g_2(y))$ для любого $y \in Y$;
        % \item $h_1(r(a,x)) = h_2(r(a,x))$ для любого $x \in X$.
    \end{enumerate}
\end{theorem}
\begin{proof}
    \textbf{Необходимость.} Пусть $t: C \times Z \to D_0$ --- пространство Чу категории $Chu_{\widetilde{Set}}$, $(f,g,h): t \to r$ --- уравнитель преобразований Чу $(f_1,g_1,h_1)$ и $(f_2,g_2,h_2)$, то есть $(f_1,g_1,h_1) \circ (f,g,h) = (f_2,g_2,h_2) \circ (f,g,h)$. В этом случае $f_1 \circ f = f_2 \circ f$, $g \circ g_1 = g \circ g_2$ и для любого $d_0 \in D_0$ если $d_0 = t(c, (g \circ g_1)(y))$, то $(h_1 \circ h)(d_0) = (h_2 \circ h)(d_0) = s((f_1 \circ f)(c),y)$. Определим $a \in A$ как произвольный элемент образа $f(C)$, то есть $a = f(c)$ для некоторого $c \in C$. Проверим выполнение условий 1),2) теоремы. Выполнение условия 1) следует из равенств
    $$
        f_1(a) = f_1(f(c)) = f_2(f(c)) = f_2(a).
    $$
    Выполнение условия 2) следует из равенств
    \begin{multline*}
        r(a,g_1(y)) = r(f(c),g_1(y)) = h(t(c,g(g_1(y)))) =\\= 
        h(t(c,g(g_1(y)))) = r(f(c),g_2(y)) = r(a,g_2(y))
    \end{multline*}
    для любых $y \in Y$. 
    % Выполнение условия 3) следует из равенств
    % \begin{multline*}
    %     h_1(r(a,x)) = h_1(r(f(c),x)) = h_1(h(t(c,g(x)))) =\\=
    %     h_2(h(t(c,g(x)))) = h_2(r(f(c),x)) = h_2(r(a,x)).
    % \end{multline*}

    \textbf{Достаточность.} Пусть существует элемент $a \in A$ такой, что выполнены условия 1),2) теоремы. Определим следующие множества:
    $$
        C_1 = \{c \in A \mid f_1(c) = f_2(c)\},
        C_2 = \{c \in A \mid \forall y \in Y\ r(c,g_1(y)) = r(c,g_2(y))\}.
    $$
    Определим пространство Чу $t: C \times X/\nu(g_1,g_2) \to D_1$, где $C = C_1 \cap C_2$, $t(c,\overline{x}) = r(c,x)$ для любых $c \in C, \overline{x} \in X/\nu(g_1,g_2)$. Так как $a \in C$, то $C \ne \varnothing$. Поскольку для любого $c \in C \subseteq C_2$, то $t(c,\overline{g_1(y)}) = r(c,g_1(y)) = r(c,g_2(y)) = t(c,\overline{g_2(y)})$, то есть значение $t(c,\overline{x})$ не зависит от выбора представителя $\overline{x} \in X/\nu(g_1,g_2)$. Определим уравнитель преобразований Чу $(f_1,g_1,h_1)$ и $(f_2,g_2,h_2)$ как морфизм $(f,g,h): t \to r$ следующим образом: $f(c) = c$, $g(x) = \overline{x}$, $h(d) = d$ для любых $c \in C$, $x \in X$, $d \in D$. Корректность определения $(f,g,h)$ следует из равенств $h(t(c,g(x))) = h(r(c,x)) = r(c,x) = r(f(c),x)$ для любых $c \in C$, $x \in X$. Из того, что для любого $c \in C \subseteq C_1$ следует, что $f_1 \circ f = f_2 \circ f$. Так как $g \circ g_1 = g \circ g_2$ и для любого $d \in D$ если $d = t(c,(g \circ g_1)(y))$, то $(h_1 \circ h)(d) = (h_2 \circ h)(d) = s((f_1 \circ f)(c),y)$, следовательно $h_1(d) = h_2(d)$. Тогда $(f_1, g_1, h_1) \circ (f, g, h) = (f2, g_2, h_2) \circ (f, g, h)$, то есть $(f,g,h)$ -- уравнитель $(f_1,g_1,h_1)$ и $(f_2,g_2,h_2)$.
\end{proof}

\end{document}

\begin{theorem}[существование коуравнителя]
    Пусть $r: A \times X \to D_1$ и $s: B \times Y \to D_2$ --- объекты категории $Chu_{\widetilde{Set}}$ и $(f_1,g_1,h_1), (f_2,g_2,h_2): r \to s$ --- преобразования категории $Chu_{\widetilde{Set}}$. Тогда коуравнитель морфизмов $(f_1,g_1,h_1), (f_2,g_2,h_2)$ --- это пространство Чу $t: Q \times E \to W$, где $Q = B/\nu(f_1,f_2)$, $W = D/\nu(h_1,h_2)$, $E = \{y \in Y \mid g_1(y) = g_2(y)\}$, $t(b/\nu(f_1,f_2),y) = s(b,y)/\nu(h_1,h_2)$ для любых $b \in B$, $y \in E$, с морфизмом $(f,g,h): s \to t$, где $f,h$ --- канонические эпиморфизмы, $g$ --- естественное вложение.
\end{theorem}
\begin{proof}
    Пусть условия теоремы выполнены.

    Введем обозначения: $\overline{b} = b/\nu(f_1,f_2)$, $\overline{d} = d/\nu(h_1,h_2)$ для любых $b \in B$, $d \in D$. Корректность определения $t$ следует из равенств $h_1(r(a,g_1(y))) = s(f_1(a),y)$, $h_2(r(a,g_2(y))) = s(f_2(a),y)$ для любых $a \in A$, $y \in E$.

    Покажем, что $(f,g,h): s \to t$ такое, что $f(b) = \overline{b}$, $g = 1_E$ и $h(d_2) = \overline{d_2}$ для любых $b \in B$, $d_2 \in D_2$, является преобразованием Чу категории $Chu_{\widetilde{Set}}$.

    Пусть $t': Q' \times E' \to W'$ объект категории $Chu_{\widetilde{Set}}$ и $(f',g',h'): s \to t$ преобразование Чу категории $Chu_{\widetilde{Set}}$ такое, что $(f',g',h') \circ (f_1,g_1,h_1) = (f',g',h') \circ (f_2,g_2,h_2)$. Тогда $f' \circ f_1 = f' \circ f_2$, $g_1 \circ g' = g_2 \circ g'$ и для любого ... ПРО h' НАПИСАТЬ. Так как $Q,W$ являются коуравнителями и $E$ является уравнителем в категории $Set^*$, то существуют единственные морфизмы $u: Q \to Q'$, $w: W \to W'$ категории $Set^*$ такие, что $f' = u \circ f$, ХЗ ТАК ЭТО ИЛИ НЕТ НО $h' = w \circ h$ и существует единственный морфизм $v: E' \to E$ категории $Set^*$ такой, что $g' = g \circ v$. Покажем, что $(u,v,w): t \to t'$ --- преобразование Чу категории $Chu_{\widetilde{Set}}$. Пусть $b \in B$, $e' \in E'$. Так как $u(\overline{b}) = (u \circ f)(b)$ и $f' = u \circ f$, то $t'(u(\overline{b}),e') = t'((u \circ f)(b),e') = t'(f'(b),e')$. Из того, что $(f',g',h')$ --- преобразование Чу категории $Chu_{\widetilde{Set}}$ следует, что $t'(f'(b),e') = h'(s(b,g'(e')))$. С другой стороны, из того, что $\overline{b} = f(b)$ следует, что $w(t(\overline{b},v(e'))) = w(t(f(b),v(e')))$. Так как $(f,g,h)$ --- преобразование Чу категории $Chu_{\widetilde{Set}}$, то $w(t(f(b),v(e'))) = (w \circ h)(s(b,(g \circ v)(e')))$. Поскольку $h' = w \circ h$ и $g' = g \circ v$, то $(w \circ h)(s(b,(g \circ v)(e'))) = h'(s(b,g'(e')))$. Таким образом, $t'(u(\overline{b}),e') = w(t(\overline{b},v(e')))$. Следовательно, $(u,v,w)$ --- преобразование Чу категории $Chu_{\widetilde{Set}}$. Ясно, что преобразование $(u,v,w)$ --- единственное преобразование, удовлетворяющее равенству $(f',g',h') = (u,v,w) \circ (f,g,h)$.
\end{proof}